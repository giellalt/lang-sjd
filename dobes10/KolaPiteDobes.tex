\documentclass[a4paper,12pt]{article}

\usepackage{fontspec}
\usepackage{xunicode}
\usepackage{textcomp}
\usepackage{graphicx}

\usepackage[english]{babel}

\usepackage{url}

\usepackage{colortbl}
%tags: %\rowcolor[gray]{0.8} %\columncolor[gray]{0.8}

\usepackage{subscript}

\usepackage{linguex}

\usepackage{lscape}

\usepackage{longtable}

\usepackage{tabularx}

\setromanfont{Arial}

\usepackage{natbib}
\bibpunct[: ]{(}{)}{;}{a}{}{;}

\begin{document}
\urlstyle{same}
\newfontinstance\scshape[Letters=SmallCaps,Numbers=Uppercase]{Hoefler Text}

%just temporary, for easier browsing the PDF
\tableofcontents
%just temporary, for easier browsing the PDF

\begin{flushleft}
\begin{Large}
\textit{Funding Initiative\\
\textbf{“Documentation of Endangered Languages”}}\\\bigskip
\end{Large}

\textbf{VolkswagenStiftung}\\
Dr.\,Vera Szőllősi-Brenig\\
Kastanienallee 35\\
30519 Hannover\\
GERMANY
\end{flushleft}

\begin{flushright}
\today
\end{flushright}

\begin{flushleft}
\begin{tabularx}{\textwidth}{ l | X }
\hline
\multicolumn{2}{>{\large\columncolor[gray]{0.8}} c }{Personal data and adresses}\\
\multicolumn{2}{>{\large\columncolor[gray]{0.8}} c }{Applicant(s), cooperation partner, grant recipient}\\
\hline
\multicolumn{2}{>{\large\columncolor[gray]{0.8}} l }{Principal applicant}\\
\hline
\hline
\textbf{Family Name} & {\textbf{Trosterud}}\\
\hline
\textbf{First Name} & {\textbf{Trond}}\\
\hline
Female / Male & {Male}\\
\hline
Titel & {PhD}\\
\hline
Field of Study & {Saami linguistics}\\
\hline
\hline
\textbf{Institution} & {\bf{Universitetet i Tromsø}}\\
\hline
\textbf{Department} & {\textbf{Institutt for språkvitskap}}\\
\hline
Postcode & {9037}\\
\hline
City & {Tromsø}\\
\hline
Country & {Norway}\\
\hline
Phone No & {+47-77644763}\\
\hline
E-Mail Address & {trond.trosterud@uit.no}\\
\hline
Homepage & {http://www.hum.uit.no/a/trond/}\\
\hline
\end{tabularx}
\end{flushleft}

\noindent The application has not been / will not be submitted to other funding institutions.\\

Signature\\
%\begin{figure}[htbp]
%\begin{center}
%\includegraphics[width=3cm]{signature.jpg}
%\end{center}
%\end{figure}

\noindent \textit{\textbf{Please note that the Volkswagen Foundation – in accordance with regulations safeguarding data privaty – records electronically your personal data as well as the project proposal.}}

\newpage

\begin{flushleft}
\begin{tabularx}{\textwidth}{ l | X }
\hline
\multicolumn{2}{>{\large\columncolor[gray]{0.8}} l }{Co-applicant and grant recipient}\\
\hline
\textbf{Family Name} & {\textbf{Rießler}}\\
\hline
\textbf{First Name} & {\textbf{Michael}}\\
\hline
Female / Male & {Male}\\
\hline
Titel & {M.A. (Dr.\,phil expected 2010)}\\
\hline
Field of Study & {Scandinavian linguistics}\\
\hline
\hline
\textbf{Institution} & \textbf{Albrecht-Ludwigs-Universität Freiburg i.\,Br.}\\
\hline
\textbf{Department} & \textbf{Skandinavisches Seminar}\\
\hline
Street & {Platz der Universität 3}\\
\hline
Postcode & {79085}\\
\hline
City & {Freiburg}\\
\hline
Country & {Germany}\\
\hline
Phone No & {+49-761-203-3300}\\
\hline
Mobile Phone No & {+49-179-9441585}\\
\hline
Fax No & {+49-761-203-3366}\\
\hline
E-Mail Address & {michael.riessler@skandinavistik.uni-freiburg.de}\\
\hline
Homepage & \url{www.skandinavistik.uni-freiburg.de/institut/mitarbeiter/riessler/}\\
\hline
\end{tabularx}
\end{flushleft}

\newpage

\begin{flushleft}
\begin{tabularx}{\textwidth}{ l | X }
\hline
\multicolumn{2}{>{\large\columncolor[gray]{0.8}} l }{Co-applicant}\\
\hline
\textbf{Family Name} & {\textbf{Gerstenberger}}\\
\hline
\textbf{First Name} & {\textbf{Ciprian}}\\
\hline
Female / Male & {Male}\\
\hline
Field of Study & {Computer linguistics}\\
\hline
\hline
\textbf{Institution} & {\bf{Universitetet i Tromsø}}\\
\hline
\textbf{Department} & {\textbf{Institutt for språkvitskap}}\\
\hline
Postcode & {9037}\\
\hline
City & {Tromsø}\\
\hline
Country & {Norway}\\
\hline
Phone No & {…}\\
\hline
Fax No & {…}\\
\hline
E-Mail Address & {…}\\
\hline
Homepage & {…}\\
\hline
\end{tabularx}
\end{flushleft}

\newpage

\begin{flushleft}
\begin{tabularx}{\textwidth}{ l | X }
\hline
\multicolumn{2}{>{\large\columncolor[gray]{0.8}} l }{Co-applicant}\\
\hline
\textbf{Family Name} & {\textbf{Wilbur}}\\
\hline
\textbf{First Name} & {\textbf{Joshua Karl}}\\
\hline
Female / Male & {Male}\\
\hline
Field of Study & {Documentary Linguistics}\\
\hline
\hline
\textbf{Institution} & {\bf{Humboldt-Universität zu Berlin}}\\
\hline
\textbf{Department} & {\textbf{Nordeuropa-Institut}}\\
\hline
Postcode & {10099}\\
\hline
City & {Berlin}\\
\hline
Country & {Germany}\\
\hline
Phone No & {+49-30-2093-4850}\\
\hline
Fax No & {+49-30-2093-9626}\\
\hline
E-Mail Address & {wilburjk@staff.hu-berlin.de}\\
\hline
Homepage & {\url{http://www.ni.hu-berlin.de/personal/jwil/jwil_html}}\\
\hline
\end{tabularx}
\end{flushleft}

\newpage

\section*{Project proposal}

\section{Basic information}

\begin{tabbing}
LLLLLLLLinks \= Mitte \= Rechts \kill
Project title: \>\textbf{Language Technology for Small Saamic Languages}\\
%alternative titles:
%Multimedia corpora and language technology for highly endangered Saami languages
%Computer-aided analyses of Saami language corpora
%Spoken corpora and computer linguistics for small Saami languages
Total budget: \>\textbf{250,000 €}\\
Project period: \>\textbf{October 2011 – September 2014}\\
\end{tabbing}
Keywords: {\it Documentary Linguistics, Saami Linguistics, Computer Linguistics, Applied Linguistics, Corpus Linguistics, Revitalization; Kildin Saami, Ter Saami, Ume Saami, Pite Saami, ISO 639-3: sje, sjd, sjt, sju}

\section{Abstract}%up to 2 pages / is rather finished, but has to by checked for consistency before finalizing the application 
The proposed project will use existing DoBeS and ELAR archives to create a state-of-the-art corpus infrastructure as well as language technology tools for four endangered Saami languages: Ume, Pite, Kildin and Ter Saami. This shall involve a variety of aspects relevant to the field of modern documentary linguistics: endangered language documentation and archiving practices, automatic annotation, and computer-aided tools for creating lexica and educational materials. Ultimately, the project intends to develop new methods for processing already extant archived linguistic data from endangered Saami languages in order to improve subsequent/future corpus linguistic research and to support language revitalization efforts with language technology.

The main goal of the project is twofold: 1) to apply language technology tools to efficiently supplement corpora of digitized, transliterated and translated texts with consistent annotations, and 2) using the resulting annotated corpora, to then create practical applications such as paradigm and word-form generators, interactive teaching materials and electronic lexica. As a result, the needs of both the academic community and the speech communities concerned will be served.

Specifically, the project shall result in the following outcomes:
\begin{itemize}
\item a searchable corpus of spoken language for these four endangered Saami languages which shall be linked to multimedia files and intended not only %particularly 
for use in linguistic research, but also as a gateway to the corpus for the language communities;
\item automatic annotation of the corpus;
\item creation of language technology tools for both linguists and revitalization activists such as lemmatizers%TROND, CIPRIAN – is this the english word?
, intelligent dictionaries, translation tools, educational materials, etc. (again based on tools currently used by Giellatekno for North Saami); 
\item development of methods, workflows, conventions and best-practice guidelines for such projects;
\item dissemination of these outcomes to be utilized in the future as at least a template or source of inspiration for other DoBeS, ELAR and other endangered language projects for developing corpus linguistics and language technologies.
\end{itemize}
Pite Saami and Ume Saami are spoken in Sweden, while Kildin Saami and Ter Saami are spoken in Russia; they belong genealogically to distinct subbranches of the Saamic group inside Uralic: Ume is in the southern branch of West-Saamic, Pite in the central branch of West-Saamic, and Kildin and Ter belong to the Peninsula branch of East-Saamic. All four languages are highly endangered or moribund due to ongoing language shift to the regional suprastrate Indo-European languages Swedish and Russian.

For three of the four languages in question, annotated multimedia corpora already exist thanks to the groundwork carried out by principle applicant Michael Rießler (partially with the help of co-applicants E.\,Karvovskaya and J.\,Wilbur) for the Kola Saami Documentation Project (a DoBeS project) and by co-applicant J.\,Wilbur for the Pite Saami Documentation Project (a Hans Rausing Endangered Languages Project). These corpora covering Kildin, Ter and Pite Saami will form the initial core set of resources for the project's work, but will be supplemented with more data and more elaborate annotations in the course of the project, as well as with addition of the first annotated corpus of Ume language materials. While automatic generation of annotations and corpora is nothing new, the application of such computer-aided technologies to spoken, multimedia archived materials of endangered languages is a first.

Community-driven revitalization efforts are underway to varying extents for all four languages. Thus, there is local interest in working on these languages while speakers are still living, and the proposed project will work closely with the language communities in producing practical products in tune with the respective communities' wishes and expectations.

The project will be carried out within the framework of contemporary documentary linguistics and as a joint effort between the Scandinavian Department at Freiburg University and the center for Saami language technology (Giellatekno) at the University of Tromsø.
The team shall consist of experienced documentary and computational linguists as well as programmers, all of whom are familiar with Saamic languages in general and have excellent working knowledge of the Saami languages in question and the relevant \textit{lingua franca}. Team members are also well versed in corpus- and computer linguistics, in carrying out documentation fieldwork, and in practical revitalization issues. The project will also profit from the symbioses already in place between researchers familiar with the Giellatekno infrastructure (including tools for other northern languages) and researchers acquainted with the DoBeS and ELAR infrastructures (documentary linguistics of spoken language, archiving of annotated multimedia).

The two leaders of the project are computer linguist Trond Trosterud (Tromsø) and documentary linguist Michael Rießler (Freiburg), both of whom are specialized in Saami linguistics. The latter applicant is a junior researcher who also led a DoBeS project on the Kola Saami languages. The other two applicants are Joshua Wilbur, currently coordinator/researcher for the Pite Saami Documentation Project (ELDP), and Giellatekno's programmer Ciprian Gerstenberger. Funding for the proposed project would go towards a full principle researcher position for M.\,Rießler and a half-time principle researcher position for J.\,Wilbur at the University of Freiburg, two student research assistants as well as material support and travel expenditures. 

\section{Detailed project description}%up to 25 pages
\subsection{Introduction}% rather finished

The \textit{Dokumentation bedrohter Sprachen} (DoBeS) and the \textit{Hans Rausing Endangered Language Project} (which ELAR is a part of) programs have the aim to create data-orientated, multifunctional, and generally accessible documentations of endangered languages in order to support languages in danger of extinction and to help protect endangered linguistic and other cultural data from disappearing. In this, archiving recordings of spoken languages alone is not considered sufficient; instead, archived recordings must include metadata and annotations describing their contents. This should at least consist of cataloging metadata (concerning e.g. participants, recording location, etc.) and transcriptions of the recordings with translations into a stable \textit{lingua franca}. However, annotations can provide much more detailed information, such as the specifics provided by detailed linguistic analyses\footnote{Detailed linguistic analyses typically contain such information as part of speech, glosses, and/or word or morpheme boundaries.}. Such analyses are interesting and useful in and of themselves for linguistics research, but they also can be utilized to produce lexica, translation tools and teaching materials for use by linguists and, at least as important, by the relevant endangered language community. However, producing such detailed analyses is exceptionally time and resource consuming, and as a result, only a portion of archived materials typically include such detailed information. Fortunately, with the help of current computer-based technology, the availability of such detailed linguistic analyses in a computer-readable format can result in:
\begin{itemize}
\item easier and more effective searches of documentation corpora for both corpus linguists and language community members, including linking search results to the actual primary recordings;
\item the creation of lexica, translation tools and educational materials utilizing modern web-based technologies.
\end{itemize}
It is thus the logical next step from the perspective of both language and research communities to supplement archived materials with as much detailed linguistic analyses as possible and then, with the help of computers, to create practical products as a result. This task is particularly urgent for endangered languages because 1) access to native speaker knowledge is limited and potentially no longer possible in a relatively short time, and 2) because those involved in revitalization efforts want to take advantage of the practical tools that result in counteracting wide-spread language death at a local level as soon as these become available.

\subsection{Preliminary considerations}%almost finished, some formulations and the English are unfinished though
\paragraph{What are corpora?} In principle, all transcriptions of recorded texts result in searchable corpora. If at least one translation is provided with the transcription we have a parallel corpus. In fact, most parts of Saami language documentation are still available as parallel corpora in books or other printed media and are thus only manually searchable. If such text corpora are scanned they can be made machine-readable but can still be used only for raw text search. In accordance with the central aims of the contemporary documentary linguistics, i.e. data orientation, multifunctionality and general accessibility, DoBeS corpora (similar to ELAR or other archives) do not only provide a direct link to the original recordings but in most cases these corpora include independent linguistic (and even other non-linguistic) annotation. Normally, these annotations are created as the result of tedious manual labour or by using semi-automatic (morphological) parsers like Toolbox.

\paragraph{What is “corpus linguistics”?} Corpus linguistics means the study of linguistic phenomena as expressed in “real world” text, such as in the mentioned printed text representations or, much more efficient, in machine-readable corpora. The problem by creating a “good” machine-readable corpus is the consequent annotation …???… one kind of information in one annotation tier (e.g. parts-of-speech), consequent use of certain tags for similar phenomena (e.g. fixed list of morphological glosses, kinds of morpheme boundaries, non-linguistic phenomena, etc.) Im Gegensatz zum fehleranfälligen manuellen Annotieren würde Toolbox, prinzipiell, konsequentes taggen möglich machen. In der Praxis jedoch scheinen viele in Toolbox angefertigten Corpora auch nur mit einer besseren Volltextsuche zu funktionieren. Aufgrund von Inkonsequenzen beim Taggen. Damit kann man in solchen Koprora inhaltsmäßig nur relativ beschränkt suchen.

\paragraph{What is computer linguistics and language technology?} Computer linguistics means the machinable%is this english??
 processing of natural language. This normally includes, among other things, the Entwicklung von Analyse- und Generierungsverfahren für natürlich-sprachliche Texte, Programme zur Sammlung und statistischen Auswertung großer Mengen von Sprachdaten (like (Lemmatisierung, Häufigkeitswortlisten, Konkordanzen)%this is not :-)
 Computer linguistics also aims at the derivation of language corpora largely by automated processes (normally morphological+syntactic disambiguation). Language technology could be seen as an (functioning) application of computer linguistics as it is aimed at analyzing and generating natural language in various ways and for multiple mostly practical purposes. Praktische Anwendungen sind z.B. maschinelle Übersetzung, computergestützter Sprachunterricht.%no-o Jussi no cry! … this is hier no englisch … 
 
\paragraph{What are multimedia spoken vs. written corpora?}
Giellatekno already has all this alles hier oben%DE->EN
 for Northern Saami, partly also for Lule and South Saami, very little even for Kildin Saami. However, these are all written languages and the Textkorpora auf denen die Arbeit von Giellatekno basiert sind geschriebene (einsprachige oder mehrsprachige Texte). Unser Projekt will aber ausschließlich mit gesprochenen Korpora arbeiten. Prinzipiell ist es identisch, weil die gesprochenen Texte ja zuerst orthographisch repräsentiert werden (transkribiert) werden müssen. Aber es gibt ein paar dinge zu beachten:
%das hatte ich alles irgendwo schon mal aufgeschrieben

Außerdem sind viele Korpora (wie auch die existierenden Giellateknokorpora) auschließlich schriftlich repräsentiert. Wir dagegen haben die Links zum alignierten Multimedia. Das ist auch kein Novum, z.B. gibt es Childis (Kindersprache) oder verschiedene Dialektkorpora mit Multimedia (z.B. in Schweden, Norwegen). Aber unser Korpus wird Multimedia haben und automatisch annotiert sein. Wir verbinden also Dokumentationslinguistik mit Korpuslinguistik sowie mit Sprachtechnologie. It is precisely these considerations which are at the source of motivation for the proposed project. Unser hauptziel ist eine korpuslinguistische infrastruktur zu schaffen, damit man später aus DoBeS/ELAR für Saami noch mehr machen kann.%DE->EN

\paragraph{Why choosing specifically those four Saamic languages?}
-because computer linguistic infrastructure already available for closely related (linguistically and culturally) written Northern Saami
-because computer linguistic know-how already available for simple and parallel written corpora of Northern Saami

Diese vier saamischen Sprachen bieten darüber hinaus interessante variation mit spezifischen schwierigkeiten. um die ergebnisse des projekts blablabla auch für generelle überlegungen andere projekte mit sprachtechnology und koproro für endangered spoaken languages zu nutzen
\begin{itemize}
\item Kildin (highly endangered, with an orthography, with some description and documentation)
\item Ter (moribund, without an [established] orthography, with little description and documentation)
\item Pite (moribund, without an [established] orthography, with little description and documentation)
\item Ume (practically extinct, without an [established] orthography, with very little description and documentation)
\end{itemize}

\subsection{Urgency of documentation}%the Kola Saami part (MR) ist still missing
The Saamic languages (Uralic) form a dialect continuum across an area extending from central Scandinavia to the Kola Peninsula in the Russian Federation. Traditionally, the Saami were half-nomadic peoples who migrated annually between summer and winter settlements and survived on hunting and fishing. Herding reindeer was not a typical occupation until relatively recent times, but has become so at least in the central mountainous areas of Sápmi %das stimmte so nicht, was Josh geschrieben hatte: Rentierzucht gibt es erst seit dem ausgehenden Mittelalter und ist auch nur für einen Teil Saami typisch; vor allem waren sie bis vor kurzem halbsesshafte Fischer, Jäger und hatten oft sogar etwas Hausvieh
; nowadays most Saami live modern lives and blend in with the rest of the predominantly non-Saami population. Due to the influence of North Germanic, Finnic and Russian languages and cultures, all ethnic Saami are fluent in their respective contact languages, while younger generations are typically monolingual in these. As a result, the Saami languages are today either extinct, moribund or endangered. The least endangered Saamic language is Northern Saami with approx. ?? speakers in Norway, Sweden and Finland out of ?? ethnic Saami.%MR:I'll check the figures
The proposed project will focus on four particularly endangered Saami languages: Ume, Pite, Kildin and Ter Saami.

Ume is a southern Saamic language of West-Saamic and spoken in central Swedish Lapland in and around the modern towns of Arvidsjaur, Malå, Ammarnäs and Tärna. Today, Ume is practically extinct, as the estimated number of speakers is less than five.\footnote{This is J.\,Wilbur's own estimate, based partly on hear-say; Henrik Barruk (ca.~40 years old) taught himself Ume as a second language, but speaks it with his two kids, who could be considered Ume speakers; according to my contacts in Arjeplog, there are 2–3 elderly non-active speakers in Arvidsjaur.} Although the first Saami language book ever was a bible translation (New Testament) in Ume Saami from 1755, no standardized orthography has been developed, nor is there any significant set of literature in Ume. However, one Ume language activist (Henrik Barruk) has recently developed some educational materials and texts on his own.

The situation for Pite Saami is not quite as desperate, but can hardly be described as positive. Pite (also known as Arjeplog Saami) is in the central branch of West-Saami and is currently spoken by approximately 30 speakers in and around the community of Arjeplog in central Swedish Lapland. Of these speakers, the vast majority are 50 years old or older and have neglected to teach their children Pite. There is one exception, a 33 year old reindeer herder who actively speaks Pite on a daily basis in his work with his family's reindeer, and speaks Pite with his two young sons (6 and 4 years old). As with Ume, Pite does not have an established orthography; however the local Saami association in Arjeplog is currently completing a project called \textit{Insamling av pitesamiska ord}\footnote{Expected completion in December 2010.} to collect a wordlist consisting of several thousand words, and is creating an orthography in the process. Co-applicant J.\,Wilbur has worked closely with the project as an unofficial “linguistics consultant” and in many conversations with Nils-Henrik Bengtsson, the project coordinator, has heard about the project's interest in using the wordlist to create a dictionary and pedagogical materials.

Historically, Ume and Pite were also spoken in the adjacent parts of central Norway, but today, the territory in which they are spoken has been reduced to a small area in Swedish Lapland due to repressive language policies in both Norway and Sweden, particularly during the first half of the 20th century. Today, Swedish language and culture dominate the lives of the Ume and Pite Saami to such an extent that their languages have practically been wiped off the map due to language shift to Swedish. Indeed, \citet[123]{blokland-etal2003} point out that Swedish laws protecting the status of Saami minorities lag behind those in Norway and Finland; this has contributed to a faster decline of Saami languages and culture in Sweden. Current Swedish (and Norwegian) language legislation only applies to South, Lule and North Saami, and ignores Ume and Pite as separate languages \cite[180]{kulonen-etal2005}, despite the fact that they are considered independent languages by linguists \citep[cf.][]{gordon 2005,sammallahti1998b}. Even a recent language law introduced in Sweden in 2009 and intended to protect and support minority languages only recognizes Saami as one single language, despite the significant linguistic and social divides between the various Saami groups. Effectively, the other three Saami languages spoken in Sweden (North, Lule and South Saami), but particularly North Saami (with the most robust group of speakers in Sweden, Norway and Finland, as well as with radio and television support) block any chances that Ume or Pite might have for gaining public recognition. 

While knowledge of Pite lexical items is common particularly among reindeer herders, the language is rarely used as a means of communication. No classes are taught in school, not even as a foreign language, however there have been sporadic intensive courses over the last two decades, targeted mostly at teaching reading and writing using the Lule Saami orthography.

Ter and Kildin Saami spoken in Russia together form the Peninsula branch of East-Saamic. Ter Saami is nearly extinct and spoken by no more than 30 speakers or semi- speakers living in different locations within and outside of the Murmansk region, such as in Murmansk, Lovozero, Gremicha, Revda, Krasnoščel'e, Umba, and also in Saint Petersburg. The mean age of the youngest speakers of Ter is over 50.%MR: I'll write more here soon

In addition to the fact that Ume and Pite are in desperate need of documentation (revitalization seems unlikely at present despite an increase in interest among younger ethnic Saami), from a purely linguistic point of view, they prove to be interesting both cross-linguistically and within Saami studies. Unlike the other Indo-European languages in the area, Saami languages are almost exclusively suffixing and mostly have postpositions. Phrases are generally head-final, although SVO has become grammatical as well, just as postponed relative constructions are acceptable (most likely a result of language contact). Saami case systems consist of between seven and nine cases. Verbs and pronouns are inflected for singular, dual and plural number categories. In Pite and Ume, all inflectional case suffixes have been retained, and accusative and genitive cases are still differentiated morphologically; they are thus considered more archaic than other Saami languages to the north and east that have lost some case endings and have come to lack an accusative-genitive distinction. Phonological and phonetic studies of Pite and Ume from the project's data should reveal more detailed information about another Saami characteristic: preaspiration (a rarity in the world's languages).

Additionally, Saami languages are known for their complicated non-linear morphology realized as umlaut, consonant gradation (phonological alternations in a stem's consonants triggered by the morphological environment) and/or palatalization of noun and verb stems. In general, regressive vowel harmony in Saami languages is found in the initial stressed syllable of a stem with its allophone being adjusted to the vowel in the second and unstressed syllable. Pite is especially interesting in this respect because it has progressive vowel harmony concerning lip rounding in the same environment \cite[272]{korhonen2005a}. Ume is interesting because it has a large variety of umlauts and the least amount of consonant gradation, with the exception of neighboring South Saami, the only Saami language lacking consonant gradation. Whether proto-Saami also had consonant gradation is a controversial issue \cite[cf.][154–55]{svonni2006}, and further information about this phenomenon in Ume and Pite could help inform this debate. Furthermore, Ume is unique among Saami languages in having trisyllabic stems in the nominative case \cite[421]{korhonen2005b}; the role nonlinear morphology plays in words of this length is unclear. Moreover, the linguistic topics of information structure, focus ETC ETC ETC ETC ETC ETC ETC ETC have been completely neglected in Saami linguistics in the past.
The resulting corpus from the proposed project could help Saami linguistics to come to a more detailed understanding of %the complicated nonlinear morphological processes in Ume and Pite, these processes' history
these topics and these specific Saami languages' places in Saami studies and linguistics in general. Finally, completion of such elaborately detailed corpora will be an asset for corpus linguistics and provide more date for future research by both Saami and general linguists on these and other topics.

\subsection{State of research}%to be finished by JW, MR
\subsubsection{Language documentation}
%Pite Saami: annotated corpus available at ELAR, linguist (for programming morphological (and syntactic) automators) available
%Kildin, Ter Saami: annotated corpus available at DoBeS, linguists (for programming morphological (and syntactic) automators) available
%Ume Saami: the most endangered Saamic language (probably only one living L1 speaker); almost no documentation; revitalization activity by speaker community 
For three of the four languages in question, annotated multimedia corpora already exist thanks to the groundwork carried out by principle applicant Michael Rießler (partially with the help of co-applicants E.\,Karvovskaya and J.\,Wilbur) for the Kola Saami Documentation Project (a DoBeS project) and by co-applicant J.\,Wilbur for the Pite Saami Documentation Project (a Hans Rausing Endangered Languages Project). %All annotations from these projects' corpora can be viewed in alignment with the respective audio/video recordings on a local computer using the program ELAN. Furthermore, the materials in the DoBeS archive can be viewed on-line via the IMDI browser. Elaborate searches can also be performed on annotations locally in ELAN or on-line with the help of the ANNEX/TROVA programs. %%JW: I don't think this is relevant for the Introduction (too detailed)

grammars, dictionaries, text collections

INFO ABOUT KILDIN/TER here...

Slightly more linguistic work has been done on Pite than on Ume. Juhani Lehtiranta published a Pite grammar in Finnish in 1992, but in fact this grammar is based on earlier studies done between 1890 and 1950, and thus does not necessarily reflect current usage. There is also an old Pite grammar in German from 1926 written by Eliel Lagercrantz. There is no dictionary of Pite.

For Ume, on the other hand, an Ume-German dictionary based on the speech of the village Malå (a dialect which is now extinct) was published by Wolfgang Schlachter in German in 1958. In 1738 a dictionary and the only grammar of Ume appeared (Fjellström 1738a, 1738b). 

Aside from discussions of Ume and Pite in various articles, that was the extent of linguistic work on these languages. However, the Pite Saami Documentation Project has now begun archiving annotated materials at the Endangered Languages Archive in London; %muss/soll ich genauer erwähnen, wieviel schon archiviert ist, noch archiviert wird? schwer zu quantifizieren.
various papers and talks are underway, %eine sehr vage Angabe!
and a PhD-dissertation including a sketch grammar and a description of morpho-phonological phenomena shall also be completed by the end of 2012 by co-applicant Joshua Wilbur. There is a considerably larger amount of material available for Lule and South Saami (the two neighboring languages to the north and south), which can prove useful in analyzing the data.

%??DAUM: Umesaamische Aufnahmen%wir nehmen DAUM nicht als Partner, schreiben nur eifach (ganz frech) dass Wennstedt zugesichert hat diese aufnahmen verwednen zu dürfen, das hat er doch mündlich tatsächlich mal getan oder?

\subsubsection{The DoBeS archive for Kola Saami}%to be finished by MR, JW

All text-based digital resources found in the DoBeS archive, including the Kildin and Ter recordings and annotations, can be searched efficiently with the help of corpus tools developed by the Technical Group at the Max-Planck Institute for Psycholinguistics in Nijmegen, the Netherlands, where the main DoBeS servers are located. All tools are available free of charge and with an open-source license, and the Technical Group intends to continue maintaining these tools into the foreseeable future, ensuring accessibility to archived data to future generations of linguists and speech community members; even if support for the tools should cease to exist, archived files are written as simple text in the open format XML, which is both readable by both machines and humans\footnote{While considered to be readable by humans, more complicated XML files are not necessarily a pleasure to read for the layperson.}. These tools are available online via the Language Archiving Technology website at \url{www.lat-mpi.eu}.%JW: of course, the people reading this application will know this - is this necessary?
The usefulness of these tools for the intents of the proposed project are summarized below.

\paragraph{Corpus search tools at DoBeS / MPI}\label{dobesTools}%to be finished by MR, JW
\begin{itemize}
\item ELAN (EUDICO Linguistic Annotator) allows audio and video recordings to be time aligned with detailed transcriptions, translations and further annotations. Because ELAN permits one to link digital text-based annotations with the actual recordings of linguistic events themselves, it is the platform on which the project's corpora will be based. ELAN files are simple text files written in XML, a current archiving standard because it is an open format which is machine and (somewhat) human readable. Elaborate searching, including use of regular expressions, is possible within a single ELAN file or across several files, all stored on a local computer or network; search results can be shown in a \textit{key-words in context} format, in other words, a context of up to eight words on either side of the search term are visible; and a complex multiple-layer search limiting the context of neighboring tiers can also be performed (same software/GUI as with ANNEX/TROVA); finally, search results can be saved.
\item IMDI-Browser (ISLE Metadata Initiative Browser) is a web-based interface which allows one to search throughout the MPI corpora, including the DoBeS resources. With this browser, cataloging metadata as well as transcriptions, translations and other annotations can be searched in. Using the same software and graphic user interface as in ELAN, a multiple layer search can also be performed and results can also be organized in a \textit{key-words in context} format; finally, search results can be downloaded and saved.
\item ANNEX/TROVA is the software and graphic user interface behind the search engines mentioned above for ELAN and the IMDI-browser. However, a complex linguistic search across archived materials using the IMDI-Browser is unfortunately not very feasible because most of the data is only translated and lacking grammatical or more detailed linguistic analysis, and even the detailed linguistic data present in the various annotations has used a variety of different annotation conventions. Partially in an attempt to increase inter-searchability between the proposed project's annotations and other annotations in the MPI corpora, but also in following good documentation practices, the project will create and document transparent glossing practices based on accepted conventions, such as those described in the Leipzig glossing rules\footnote{\url{www.eva.mpg.de/lingua/resources/glossing-rules.php}}. Due to the very nature of automatic annotation, all project transcriptions are guaranteed to be glossed consistently.
\item ARBIL fulfills a variety of functions concerning the organization of cataloging metadata for the MPI corpora when working offline, and is thus only useful for locally-stored data; once an internet connection is available, data can then be transferred to the remote servers via LAMUS. The search function not yet integrated. %JW: vielleicht brauchen wir ARBIL gar nicht erwähnen - ist höchstens nützlich wenn man off-line ist, sonst nur am rande relevant als "search tool", oder?
\end{itemize}

%KWIC-Suche (key word(s) in context)
%ich glaube diese kexwords kann man über imdi eingeben
%KWIC ist aber für  linguistische und anthropologische Forschung (sowie für Revitalizierung) nur sehr beschränkt nützlich

\subsubsection{The ELAR deposit for Pite Saami}%to be finished by Josh
In August 2010, the first Pite materials resulting from the Pite Saami Documentation Project (PSDP) went online at the Endangered Languages Archive (ELAR) in London. This initial deposit consists of six audio/video recordings and the respective ELAN annotation files, as well as related images and metadata definition files. These recordings cover a variety of genres (from descriptions of traditional handicrafts and reindeer herding practices, reindeer vocabulary, and everyday conversations). As the project will be concluded in July 2011, more recordings and transcriptions/annotations will be deposited and made available online\footnote{Access to archived materials is only available to registered users and each file may have even more limiting permission levels; however, at this point, all files are at least available to the Pite Saami community members and the linguistic research community}, including elicitation sessions, in the near future. The materials can be accessed via the ELAR website at \url{www.hrelp.org/archive/}.

The first ELAR resources from a variety of HRELP-funded projects only went online in summer 2010, and at this point, searching through ELAR materials online is limited to a very basic character-string search of basic cataloging metadata from a single project's deposits only. However, all PSDP recordings archived at ELAR include annotations in ELAN and are thus in a format which is searchable using the available Language Archiving Technology tools mentioned above in section \ref{dobesTools}. Copies of these ELAN files will be stored on the DoBeS/MPI servers to allow these to be searched as well, and, in cooperation with ELAR, the respective audio and video materials in ELAR will be virtually connected to the DoBeS archive to allow linking of search results to multimedia. %JW: oder, falls David Nathan nie ragiert: Since Joshua Wilbur as project coordinator and linguist for PSDP has copy right privilidges for PSDP materials, these will also be archived as part of the DoBeS archive in order to allow them to be searched and for search results to be linked to the relevant multimedia.
%
%\paragraph{Corpus search tools at ELAR} 
%\begin{itemize}
%\item very basic character-string search (not even as advanced as a google search!) of a single project's deposit only. I assume more will be possible in the future?
%\item ELAN annotations can be searched (off-line) using the DoBeS tools
%\end{itemize}

\subsubsection{The Saami language technology project}%not finished, needs input from GT
% TROND … write specifically, what is preliminary already there or in the works for Pite and Ume and Kildin and Ter
%TROND, CIPRIAN – we need some more text here, start with general things; mention also prestigious things for Northern Saami

The language technology project Giellatekno already has an infrastructure setup for building language analysers. This infrastructure has been set up for Pite and Kildin Saami, and for both languages there is an embryonic analyser, containing some items from each part of speech, and some of the basic morphological and morphophonological processes for nouns and verbs. The Kildin Saami analyser has also been included on the Giellatekno home page, cf. \url{http://giellatekno.uit.no/cgi/index.sjd.eng.html}.  Nothing has been done for Ume or Ter Saami yet.

\subsection{Research aims and methodologies}%rather finished, but needs better formulations and English

%3 areas:
%formal linguistic description (morphology, syntax as basis for parsers)
%JW, MR + TT
%computer linguistics (parsers)
%TT + JW, MR
%programming (tools, lexica)
%CG + TT, JW, MR

The main goal of the project is twofold: 1) to apply language technology tools (more precisely: finite-state transducers and constraint grammar) to efficiently supplement DoBeS and ELAR multimedia corpora of digitized, transliterated and translated texts with consistent annotations, and 2) using the resulting annotated corpora, to then create practical applications such as paradigm and word-form generators, interactive teaching materials and electronic lexica. 

As a result, the needs of both the academic community and the Saami speech communities concerned will be served. Furthermore, the workflows and tools established in our project can %?bereichern?
general methodology in Documentary Linguistics as our results can be helpful to other documentation projects specifically aimed at the creation of better corpus linguistic infrastructure as well as practical applications for endangered spoken languages.

Ein bedeutender Unterschied zur common practice in DoBeS und anderen Dokumentationsprojekten: wir können nur die orthographischen Repräsentationen annotieren und korpuslinguistisch aufarbeiten. Ansonsten könnten wir ja keine praktischen Produkte (Wörterbücher, Lehrmaterialien) erstellen. Die meisten anderen Dokumentationsprojekte dagegen machen eine präzise aber zeitaufwendige und fehleranfällige morphologische Annotation auf Grundlage einer phonologischen Transkription oder auf einer auf phonemischen Prinzipien beruhenden (oft von den Forschern selber hergestellten) Orthographie. Die Annotationen dieser Korpora 

Wir machen ein Korpus mit identischen Prinzipien für mehrere saamische Sprachen. Prinzipiell ist er erweiterbar für alle saamischen und auch weitere Sprachen. Außerdem von Vorteil: unsere Korpora gesprochener Sprache können später mit Korpora geschriebener Sprache zusammengeführt werden (mindestens für Kildin gibt es inzwischen auch nicht wenige schriftliche Texte) und unsere Tools müssten für geschriebene Sprache genauso funktionieren. Und natürlich: wir bringen ja unsere automatisch auf Grundlage der Orthographie erstellten Annotationen in das mit den Originalaufnahmen alignierte ELAN zu DOBES und ELAR zurück. Entsprechend der Hauptprinzipien von DoBeS (and language documenation in general): data orientation, multifunctionality and general accessibility können andere Forscher diese Annotationen ergänzen, z.B. mit morphem-breakup, phonetic annotations, etc.%DE->EN

The principal methodology applied will be (after preparing/cleaning/unifying the text corpora) to first programming analyzers and generators. With the basic grammatical analyzers and generators in place, a wide range of possibilities opens up for language analysis and development. Linguistic analysis will benefit from lemmatization and grammatical analysis: From a given corpus of collected texts, analyzers may keep track of which lemmata are covered in a dictionary and which are not. For revitalization work, …???

Given the different quantity and quality of their documentation, the degree of successful revitalization efforts inside their speaker communities (including the existence of established orthographies) and the existence of previous language technological work the project will devote its … to the four languages to different steps from the creation of a basic corpus and language technological infrastructure for Ume Saami to a full range of … and tools for Kildin Saami.

Generally, we can build on Giellatekno's already existing and functioning infrastructure and tools for written Northern Saami (and partly also other written Saami languages). With the help of this know-how the project will start working with Kildin Saami, which has an established orthography, which has some successful revitalization efforts, for which most documentation is available and which already got small corpora and a few tools at Giellatekno.

Biggest challenge (by using the Northern Saami know-how for Kildin) beside different morphological and syntactic system and different (Cyrillic) script is to establish useful and consistent annotation conventions for phenomena specific for spoken text.

Once a sufficient large Kildin test corpus is created, and morphological parsing is running successfully we will work in a similar way with Pite. Giellatekno's existing ?things? for Northern Saami but especially for the closely related Lule Saami will make it easier to write the parsers. However, Pite Saami has a lesser amount of transcribed text recordings available (as compared to Kildin). Furthermore, there is no established orthography (?jedenfalls bis heute nicht, die Situation kann in einem Jahr anders sein?). We will thus use the preliminary orthography which is also applied for ?Nisses Wörterbuchprojekt?. Eventuell muss das dann später nochmal angepasst werden.

Gleichzeitig entwickeln wir Kildin Saami weiter und beginnen mit dem syntaktischen Parsen (Disambiguieren)

Similar to Pite there is no established orthography for Ter Saami. Amount of available transcribed texts is even smaller than those for Pite.  On the other hand, Ter Saami is very closely related to Kildin and the Kildin morphological and syntactic parsers can probably be used and adjusted.

Ume finally, no orthography, almost no documentation, and is linguistically relatively distant to Pite (and the other Saami languages). We are thus planning only to create a rather small corpus of transcribed and translated text and write a morphological parser.

Die gleichzeitige Bearbeitung dieser vier Sprachen mit unterschiedlichen Voraussetzungen wird deswegen auch methodisch wertvolle generelle Fragen aufwerfen: (korpuslinguistische) Annotation gesprochener Sprache, Fragen von Orthographien und orthographischer Variation usw und deshalb werden die Resultate des Projektes für spätere Forschung sehr nützlich sein blabla

\subsubsection{Language technology} %to be finished by Trond, Ciprian
%HERE WE SHOULD MENTION GIELLATEKNO'S GENERAL RESEARCH AIMS AND METHODOLOGIES
%TROND, CIPRIAN – Could you please complete missing and/or wrong things 
Currently, the dominating paradigm within language technology is based upon statistical methods, upon teaching the computer the behavior of natural language by means of presenting it for vast amount of un-analyzed and manually analyzed data. This approach achieves results superior to the most common grammar-based approaches. For the majority of the world's languages, and especially for the Saamic languages treated here, this is not a possibility. The amounts of text needed (analyzed or not) is simply not available. The competing paradigm is grammar-based which is applied successfully by Giellatekno for (written) Northern Saami and which will also by applied by this project for (spoken) Kildin usw.

Giellatekno's approach (finite-state transducers and constraint grammar) differs from most grammar-based systems in that it gives robust analyses for unconstrained text input.

\paragraph{Two-Level-Morphology}
Zur computerlinguistischen Verarbeitung der sprachlichen Daten wird vorerst die s.g. Zwei-Ebenen-Morphologie (TWOL) angewendet. In TWOL wird die Zuordnung von Morphemen auf zwei Ebenen beschrieben:
\begin{enumerate}
\item morphologische/lexikalische Ebene: Struktur eines Lexems, Morphe, Morphemgrenzen
\item morphographischen Ebene: Oberflächenrealisierung des Lexems
\end{enumerate}
Die erlaubten bzw. verbotenen Zuordnungen von Morphemen auf beiden Ebenen erfolgt mittels (sprachspezifischer) Regeln. Jede Regel wird von einem Transducer in die Realisierung auf der jeweils anderen morphologischen Ebene übersetzt. Somit können morphologische Formen nicht nur analysiert (Übersetzungsrichtung morphographemische Ebene $\rightarrow$ morphologische/lexikalische Ebene), sondern auch generiert werden (Übersetzungsrichtung morphologische/lexikalische Ebene (Stamm+Affixe) $\rightarrow$ morphographemische Ebene (Oberflächenrealisierung)).

Der TWOL-Ansatz eignet sich besonders zur Beschreibung von Sprachen mit einem komplexen und sich durch Irregularitäten und Stammallomorphie auszeichnenden Formensystem – wie z.B. Saamisch. 

\paragraph{Zwei-Ebenen-Morphologie} (two-level morphology, TWOL) ist ein theoretischer Ansatz, in dem die Struktur eines Lexems auf der morphologisch-lexikalischen Ebene in Relation zu dessen Oberflächenrealisierung, d.h. der Kodierung des Lexems auf der morphographischen Ebene, gesetzt wird. Erlaubte bzw. verbotene Korrespondenzen zwischen Zeichen auf den zwei Ebenen werden durch abstrakte Regeln erfasst.

Um Übergeneralisierung bei der durch einen Automaten (Transduktor) angefertigten Übersetzung zu vermeiden, müssen die Regeln durch die Anbindung eines (Morphem-)Lexikons sowie durch Angaben über Morphemkombinatorik auf der morpho-lexikalischen Ebene eingeschränkt werden.

Das Lexikon stellt eine Liste aller gebundenen und ungebundenen Morpheme dar. Ein vorläufiges Morpheminventar wird aus den bereits vorhandenen Toolboxprojekten von KSDP und PSDP importiert. Hauptziel nach abgeschlossener Arbeit mit Toolbox ist es aber weitere Texte mit Hilfe der Automatoren zu parsen und das Morpheminventar damit automatisch zu erweitern.

Die Listen mit Stämmen und der ihnen zugeordneten morphologischen Markierungen werden in TWOL mittels eines mit Indizes versehener Listenverkettungsmodells (indexed concatenation model) miteinander verbunden. Dies bedeutet, dass phonologische oder graphemische Morphemvarianten eines Stamms mit unterschiedlichen Indizes versehen werden, die der Art der zugelassenen Folgekategorien (d.h. möglichen Suffixen für einen gegebenen Stamm) entsprechen. So erhält ein Stamm, dem ein bestimmte Gruppe Suffixe folgen kann, eine andere Indizierung als ein Stamm, dem eine andere Gruppe Suffixe folgen kann. Auch die Suffixe im Lexikon erhalten Indizes, die angeben, mit welcher Art indizierter Stämmen sie kombiniert werden dürfen. Die Aufgabe des Listenverkettungsmodells ist es, die Liste der Wortstämme mit der möglicher Suffixe entsprechend der Indizes einzelner Elemente in diesen Listen miteinander zu verketten.

\textit{Giellatekno} wendet TWOL bei der Erstellung von sprachtechnologischen Anwendungen wie interaktive pädagogische Programme oder Rechtschreibprüfprogramme für morphologisch komplexe Sprachen mit Stamm- und Flexionsvariation bereits an, darunter v.\,a. Nordsaamisch, aber auch andere saamische Sprachen sowie auch Grönländisch, Kveeni, Komi und weitere

\paragraph{Syntactic disambiguation}
%TROND, CIPRIAN

\paragraph{Parallel corpora}
%TROND, CIPRIAN – We have not mentioned this so far, but our Pite and Kola Elan files include in fact parallel corpora because the Saami texts are translated consequently with two languages: Pite-Shwedish-English resp. Kildin/Ter-Russian-English; can we do more out of this for the present project? what exactly? and how can we put it into this application?

\paragraph{Practical applications}
%TROND, CIPRIAN – only mention perhaps, we have to list and explain them under results anyway

%\paragraph{Wortformgenerator} Die bereits im Laufe des Pilotprojektes zu erstellenden \textbf{Paradigmen-} und \textbf{Wortformengeneratoren} können eine praktische Funktion im Sprachunterricht übernehmen. In Korpora lassen sich i.d.R. nicht alle Formen eines Paradigmas finden. Mithilfe von TWOL (aber im Unterschied zu Programmen wie Toolbox) können Wortformen jedoch nicht nur analysiert, sondern auch generiert werden. Komplette Paradigmen lassen sich somit für jeden Eintrag in der Datenbank in Echtzeit generieren oder z.B. in das geplante interaktive Lexikon einbinden.

%Dementsprechend gibt der Wortformengenerator bei vorgegebener Grundform eines Lexems und der gewünschten morphologischen Markierungen die entsprechende grammatische Form aus (z.B. input: hestur+ACC $\rightarrow$ output: hest).%besser saamisches Beispiel\\

%Sobald in einer späteren Projektphase auch ein syntaktischer Automat programmiert worden ist, könnte zusätzlich sogar ein \textbf{Interaktives Syntax-Lernprogramm} erstellt werden. Dabei liefert zuerst die morphologische Analyse auf Wortebene die jeweilige Form. Ein syntaktischer Disambiguator eliminiert daraufhin auf Satzebene die inkorrekten Analysen und versieht die korrekten Analysen mit entsprechenden syntaktischen Funktionen (Subjekt, Objekt usw.). Schließlich könnte eine Phrasenstrukturgrammatik der linearen Darstellung eine hierarchische Struktur zuordnen.

\paragraph{LEXUS} as interface for multimedia annotation (done by assistants) and web-based presentation, export to off-line dictionary applications, export to wikipedia
%MR: I had some ideas, unfortunately they are not written down yet

\subsubsection{Methodological challenges}%to be finished
The project will use annotation and presentation tools (ELAN and LEXUS) as well as corpus tools (??) created by DoBeS. These tools are created specifically to meets the needs of quite divergent linguistic and anthropological documentation projects. Crucial – in accordance with current methods in Documentary Linguistics – is that annotations are linked to multimedia, are accessible with rich metada and through user-friendly interfaces.

On the other hand, we will use methodologies and language technology created by Giellatekno specifically for written Saami languages (though they have also been successfully applied to other (written) languages). The existing Giellatekno written corpora are thus of very different character … mostly because not linked to multimedia.

Die Kombination zwischen dem dokumentationslinguistischen Ansatz (DOBES) und dem Korpus- und Computerlinguistischen Ansatz (Giellatekno) macht einige methodologische Überlegungen notwendig. Die Lösung dieser Fragen ist dabei nicht nur für unser Projekt von Bedeutung sondern auch im Generellen für Dokumentationslinguistik die bessere Korpuslinguistik sein will bzw. für Korpuslinguistik die sich mit gesprochenen, bedrohten Sprachen beschäftigen will.

%Ich weiß noch nicht wo das hin muss
%Korpus ist eine Sache, das Interface zum Präsentieren und Suchen ist eine andere Sache
%Im Prinzip kann man mit Unix oder irgendwelchen Programmen in Quelltexten mehr oder weniger komplizierte Suchalgorythmen laufen lassen
%ELAN ist ein Programm zum Präsentieren, es hat auch sehr gute Suchfunktionen (lokal)
%TRO???oder wie das heißt hat die entsprechenden Funktionen webbasiert/cross-corpus search blabla

%Lena: should we mention somewhere the differences between documentation and  a corpus? Thus what DoBeS now has is not a corpus. However this data could be organized and made search-able (this makes a corpus). However some work is required to do it. The methodology is extremely important. Not every corpus is search-able, even if it is called "corpus". We have a good example - giellatekno, we can use the methodology which already exists.  

%DoBeS Spracharchive sind eigentlich im strengeren Sinne meist gar keine Korpora (im Sinne von Korpuslinguistik), sondern sie sind entstanden als Multi-Tier-Annotationen, die möglichst viel linguistische und nicht-linguistische Information enthalten sollen, aber die man in der Regel erst korpuslinguistisch aufarbeiten muss, um tatsächlich ein bestimmtes Phänomen untersuchen zu können. Dazu kommt, dass in der Praxis nur ein Teil der Aufnahmen tatscächlich mehr als nur ein Übersetzungs- oder Transkriptionstier bekommt, weil einfach die Zeit und manpower nicht ausreicht alles manuaell zu annotieren.
%als beispiel G.Haighs Projekt [das aber nicht expliziet hier benennen!] (siehe seine Dokumente und DoBeS-Workshopvorträge zu referential density): er bittet möglichst viele dobeslinguisten auf seine art zusätzlich zu annotieren, was die meisten nicht machen, weil es zeit kostet und sie vielleicht nicht interessiert; wenn er aber schon merhrere automatisch gut vorannotierte korpora hätte, könnte er wahrscheinlich selber mit seinen handannotationen zu seinem spezifischen korpuslinguistischen projekt weitermachen und erfolgreich sein
%unser projekt will also dokumentationslinguistik und korpuslinguistik sinnvoll verbinden:
%-wir machen eine dokumentation (mit multimedia und annotationen)
%-dann machen wir eine automatische linguistische annotation (morphology, syntax), mit deren Hilfe relativ schnell große corpusmassen zur verfügung gestellt werden können
%-diese annotation geht zurück in den verlinkten multimediakorpus, der damit noch hochwertiger wird, für uns selber, für andere forscher
%-außerdem können wir sozusagen als nebenprodukt sprachtechnologische werkzeuge herstellen, die sind auch weider praktisch für die forsachung aber auch fpür die revitalizierung

\paragraph{written vs. spoken coprora} 
CORPUS OF WRITTEN LANGUAGE

we have many and large corpora of written languages

-CORPUS OF SPOKEN LANGUAGE

--we have a lot fewer and smaller corpora of spoken language (e.g. dialects, sociolects)

--written representation of text

--different units than in text

--linguistic features characteristic of spoken language (intonation, hesitation, false start, etc.)

---CORPUS OF AN ENDANGERED SPOKEN LANGUAGE

---we have very few and only very small corpora of endangered languages

---written representation of text

---different units than in text

---linguistic features characteristic of spoken language (intonation, hesitation, false start, etc.)

---code switching to other languages

---language attrition

\paragraph{The “spoken corpus” issue}
…this is mostly an issue regarding annotation conventions
-gesprochene Sprache (hohe Variation durch Dialekte und durch language attrition/language loss; sehr viel code-switching; sehr viel hezitations, false starts, self-corrections usw.; Intonationseinheiten anstatt Sätze)

\paragraph{The “orthography issue”}
… since we are working with the orthographic representation of our texts and we are producing practical tools for language users … we have to discuss in some detail for what languages orthographies are already established (Kildin is established but in several variants; Pite is under development by community members in collaboration with Josh; Ter could use the Kildin orthography but community members have to decide on this; Ume is unclear) … and how we deal with orthography issues like variants and possible later changes originating from inside the communities … I know that this is all no big deal from the Giellatekno perspective, but we have to address this for our referees 

\paragraph{Interfaces}
Different kind of users:
-speaker community
-documentary linguist, incl. anthropological linguist
-general and corpus
-other interested users
different kind of portals to our created infrastructure and tools is needed!

%!!IMPORTANT: somewhere to mention: interoperability, open source, open access, etc. is no question for us 

\subsection{Work flow}%I need input from all of you
svn/ichat/subethaedit/etc. (see below) mention in brief again

manually annotated texts linked to audio and video (ELAN) already available at DoBeS and ELAR, these texts have (at least) some sort of orthographic transcription and (at least) one translation (Swedish, Russian, English)
several texts have additional grammatical annotations (done manually or semi-manually with Toolbox): parts-of-speech, morpheme break-up, glossing

Workflow (corpus)
1. create consistent transcription in standard orthography (+using conventions for features characteristic of spoken language) – manually in ELAN (wilbur, rießler, karvovskaya)
2. create morphological parser (wilbur, riessler, trosterud)
3. create syntactic parser (wilbur, riessler, trosterud)
4. automatic corpus annotation (wilbur, riessler, trosterud, gerstenberger)
5. re-import of automatic annotation into ELAN and archiving at DoBeS/ELAR (wilbur, rießler, karvovskaya)

Workflow (tools)
?? (gerstenberger)

Workflow (dictionaries)
1. lemmatization of automatically annotated corpus (wilbur, rießler, gerstenberger)
2. dictionary programming (wilbur, rießler, gerstenberger)
3. xml converting between different formats (oahpa, gt, lexus, etc.) (riessler, trosterud, gerstenberger)

%create a monolingual multimedia dictionary after automatic lemmatization]

\subsection{Expected outcome of the proposed project}%needs to be finished
Generally, the project shall result in the following outcomes:
\begin{itemize}
\item a searchable corpus of spoken language for these four endangered Saami languages which shall be linked to multimedia files and intended not only %particularly 
for use in linguistic research (similar to the Giellatekno corpus for North Saami in Tromsø, but with the addition of multimedia), but also as a gateway to the corpus for the language communities;
\item automatic annotation of the corpus (as currently possible for North Saami in the Giellatekno infrastructure);
\item creation of language technology tools for both linguists and revitalization activists such as lemmatizers%is this the english word?
, intelligent dictionaries, translation tools, educational materials, etc. (again based on tools currently used by Giellatekno for North Saami); 
\item development of methods, workflows, conventions and best-practice guidelines for such projects;
\item dissemination of these outcomes to be utilized in the future as at least a template or source of inspiration for other DoBeS, ELAR and other endangered language projects for developing corpus linguistics and language technologies.
\end{itemize}

\subsubsection{Corpus linguistics}
\paragraph{Searchable, consequently, completely annotated multimedia Saami corpora at DoBeS/ELAR}
%nochmal expliziet aufschreiben, dass wir komplett die gesamten Kildin, Ter, Pitetexte die es jetzt schon Bei DOBES/ELAR gibt parsen werden und sie in ELAN zurückexportieren; damit sind die jetzt schon existierenden Dokumentationen korpuslinguistisch ebesser (einheitlich/konsequent) annotiert und können für linguitische Untersuchungen (auch vergleichend über diese saamische Sprachen hinweg, sowie mit einschluss des bereits existierenden nordsaamischen, lulesaamischen Materials bei Giellatekno) viel besser verwendet werden.
...will be available and searchable using DoBeS tools
\paragraph{Searchable, consequently, completely annotated plain text Saami corpora at DoBeS/ELAR}
...will be available and searchable using the Giellatekno search interface, similar to the Northern Saami interactive text corpus available from \url{http://giellatekno.uit.no/text.en.html}.

\subsubsection{Language technology}
The following specific language technological outcomes are planned to be %???
\paragraph{Tools for grammar and analysis}
These tools will include 1) a text analysis tool (which analyzes, disambiguates and hyphenates text, and gives dependency analysis; it also converts text into IPA), 2) a paradigm generator (which generates paradigms of different sizes for any word), 3) a word generator, and 4) a number word generator (both generate wordforms using lemma and grammatical tags). The interface for these tools will be available (in different language localizations) from the Giellatekno website, similar the the already existent North Saami tools at \url{http://giellatekno.uit.no/cgi/index.sme.eng.html}. It will also be linked from the other project portals at DoBeS and ELAR.

\paragraph{Pedagogical program \textit{Oahpa!}}
Oahpa! (North Saami for ‘learn!’) is a collection of interactive games that were originally developed for North Saami learners. Similar to North Saami our Oahpa! will include vocabulary trainers, morphology trainers and question-answer trainers. The Oahpa! platform will also include short grammatical overviews of the single Saami languages written in Russian (for Kildin and Ter Saami) and Swedish (for Pite and Ume Saami). The interface for the Oahpa! program will be available (in different language localizations) from the Giellatekno website, similar the the already existent North Saami tools at \url{http://giellatekno.uit.no/oahpa/}. It will also be linked from the other project portals at DoBeS and ELAR.

\paragraph{Integrated electronic dictionaries}
We will create integrated dictionaries which may be used for offline use in traditional dictionary applications, such as “Dictionary” (Mac) or StarDict (Windows, Linux). Some applications, like Safari or several text editors, even allow lookup in locally running programs by using the right-mouse click. The dictionaries even contains all basic inflected forms of each lemma (and these inflected forms are automatically generated by use of the ??) similar to the already existing and running dictionary for North and South Saami (cf.~\url{http://giellatekno.uit.no/words/dicts/index.eng.html}.)

\paragraph{Machine translation}%to be finished by Trond or Ciprian
… cf. Preliminary machine translation programs for several language pairs \url{http://victorio.uit.no/cgi-bin/francis/index.php?lang=eng}

\paragraph{Online Lexica – LEXUS}%to be finished by Micha

\paragraph{Lexical resources} are created automatically and will include, e.g., wordform frequency lists (most and less frequent words, hapaxes, tergo lists), letter frequency lists, and parts-of-speech frequency lists. The interface for these lexical resources will be available (in different language localizations) from the Giellatekno website, similar to the already existent site at \url{http://giellatekno.uit.no/lex.en.html}. It will also be linked from the other project portals at DoBeS and ELAR.

\paragraph{Spell checker and hyphenation}%has to be finished
… use Divvun experience with Northern Saami

\subsubsection{User portals}
Access to the data/tools as well as information about the project, project documentation from three different partals:
-Giellatekno in multilingual (Norwegian/Swedish, Russian, English, other) localizations is already existent
-Project portal, under creation \url{www2.hu-berlin.de/saami/} (would move to Freiburg's server) community friendly multilingual portal to ELAR and DoBeS archives, to PSDP, KSDP, Tools, etc. 
-ELAR: general project info (PSDP) links
-DoBeS: general project info (KSDP) links

\subsection{Cooperation partners}
%Reihenfolge der Kooperationspartner vielleicht ändern?

{\bf Kola Saami Language Center in Lovozero} … former Saami project assistants and main consultants: Sharshina, Danilova, Antonova, Vinogradova … coordinated by former KSDP researcher Scheller

Kola Saami Youth organisation \textit{\textbf{Sām' nūraš}}  led by Anna Afanasyeva. A.\,Afanasyeva … formerly Saami student assistant in {\it Kola Saami Documentation Project} … currently enrolled as M.A. student in Native Studies at Tromsø University. … native Kildin Saami and Kildin Saami speaker and political and language activist … {\it Sām' nūraš} works among other things with language revitalization of Kildin, Ter and Skolt in Russia … especially modernization of the language, das prestige der Sprache erhöhen und sie z.B. im Internet sichtbar und nutzbar zu machen … {\it Sām' nūraš} plant unter anderem ein Kildinsaamisches Wikipedia  zu gründen. Sie wollen auf die Expertise unseres Projekts zurückgreifen.

Giellatekno already collaborates closely with \textit{\textbf{Divvun}} (\url{http://www.divvun.no/}), led by the computer linguist Sjur Nøstebø Moshagen. {\textit{Divvun} is a language technological project administered by the Norwegian Saami Parliament. The aim of {\textit{Divvun} is to create spell checkers and hyphenation programs for the Norwegian Saami language (Northern. Lule and South Saami). The tools for Northern and Lule Saami already run on Linux, MacOS X and Windows for the most common office applications. The tools for South Saami are in the works. Since the same language technological infrastructure is shared by {\textit{Divvun} and {\textit{Giellatekno}, our project can refer to already existent know-how when working with spell checkers and hyphenation programs for KIldin and Pite Saami.

The project will be an active group member of \textit{\textbf{SaamiDocNet}} (\url{http://saamidocnet.uit.no/}) as are Giellatekno and the current {\it Pite Saami Documentation Project} and {\it Kola Saami Documentation Project} already. {\it SaamiDocNet} is a research network for Saami documentation and revitalization funded by the Nordic NordForsk (preliminary until 2012%check with Bruce
 ). It s aim is to enhance Saami language research, documentation, maintenance and (re)vitalization by keying these directly to one another. The network is developing a web-based, open-access, information portal to ensure that Saami language resources are easily found and accessible. It also funds membership travel to workshops, conferences, training courses and other relevant activities aimed at improving the knowledge-base of young researchers and giving them unique opportunities to build collaborative networks of their own. {\it SaamiDocNet} is coordinated by Bruce Morén-Duolljá at Tromsø university.
 
The project will cooperate with \textbf{Peter Steggo} who agreed to be an external specialist for Ume and Pite Saami. Being a native Pite Saami from Arjeplog, P.\,Steggo has studied Saami linguistics at Umeå university and is now employed by Arvidsjaur municipality as a language consultant for Saami in Arvidsjaur. Till arbetsuppgifterna hör bland annat att utbilda personal i samiska samt jobba med språkutvecklande aktiviteter gentemot förskola, skola och barnomsorg.%SW->EN  

Another Saami cooperation partner from the Swedish side is ...?Arjeplog project/?... lead by Nisse Bengtsson. Both N.\,Bengtsson and P.\,Steggo are familiar with ...?documentary linguistics in particular with Giellatekno's practical approach to Saami language technology and creation of teaching programms etc.?... They have both participated at training provided by the DoBeS Winter School in Saami Language Documentation and Revitalization (Bodø 2010) and ...?work together/consults?... with {\it Pite Saami Documentation Project}. 

… {\bf Elisabeth Scheller} is a Ph.D. student in Saami linguistics at Tromsø University. She has a background in both the academic and practical issues involved in Saami languages and cultures - particularly with respect to Kola Saami language revitalization. E.\,Scheller is a former KSDP researcher … has completed a large scale sociological survey of language use und languege loss among Kola Saami. In cooperation with the {\it Kola Saami Documentation Project}, she has been instrumental in establishing a Kola Saami language center in Lovozero and has is active and central in current Kildin Saami revitalization and maintenance initiatives. She has organized and administrate activities, such as language courses, language circles, language camps and other practical measures. She will consult our project in issues related to collaborative work with the Kola Saami communities
 
\bibliographystyle{linquiry2.bst}
\bibliography{KolaPiteDobes}

\section{Community consent}
%Josh – wie können wir das am besten schreiben?
%PSDP and KSDP are already working closely together with the Saami communities and will use the already existing networks and will work together with the same Saami assistants as before
%Giellatekno is in fact a Saami project, employs several Saami, working language is partly Northern Saami partly Norwegian
%All cooperation partners/external specialist are either Saami themselves or institutions working closely with or inside Saami community

\section{Binding indication}

%AND SOURCE CODE habe ich zur Originalformulierung (von VW vorgegeben) dazugefügt
The applicants declare that all data will be recorded and processed in accordance with the linguistic, technical and juridical framework of the program group “Documentation of Endangered Languages (DoBeS)” and the outcoming language documentation and source code will be transferred to the archives of the central data bank project.

\section{Key project participants, their responsibilities in the project and previous work on the subject}

\subsection{Project participants' previous work on the subject}
The team consists of experienced documentary and computer linguists and programmers well familiar with Saamic languages.

\paragraph{Michael Rießler} completed a \textit{Magister} degree in Scandinavian Linguistics at Humboldt-Universität in Berlin. His research has focused on both historical and synchronic-typological issues, and has covered a variety of topics including (but not limited to) areal linguistics in Northern Europe, Saami linguistics, and Documentary linguistics. He submitted his doctoral dissertation in General Linguistics in July 2010, and his defense at the Universität Leipzig is expected in November 2010.

During the last six years, Michael Rießler has conducted extensive fieldwork on Kola Saami languages. Due to his work as principle investigator and coordinator of the “Kola Saami Documentation Project”, a DoBeS project, he is familiar with the objectives and methods of contemporary documentary linguistics in general and with the framework of the DoBeS program group in particular. He has also organized a DoBeS Winter School on “Saami Language Documentation and Revitalization” (together with Ulrike Mosel, Jurij Kusmenko, Bruce Morén, and Ida Toivonen) which took place in 2010 in Bodø/Norway.

In connection with the documentation of Kola Saami M.\,Rießler has extensively collaborated with the Saami language technology group (Giellatekno) at Tromsø university and has helped creating Web-based teaching tools for Kildin and Skolt Saami. He is thus well familiar with practical computer linguistics applications and the methods and workflows applied at Giellatekno.

\paragraph{Trond Trosterud} … Saami linguist, Computer linguist, project administrator of Giellatekno, employed in Tromsø

\paragraph{Ciprian Gerstenberger} … Computer linguist, programmer, employed in Tromsø
%bei tronds und cips bios müssen wir genau specifizieren, wieviel ihrer arbeitszeit sie unserem projekt widmen dürfen/können

\paragraph{Joshua Wilbur} completed a \textit{Magister} degree in General Linguistics and American Studies at the Universität Leipzig in March 2008. Since April 2008, he has been a doctoral student, initially at the General Linguistics department at Humboldt-Universtität in Berlin under the supervision of Manfred Krifka, but has now transferred to Christian-Albrechts-Universtität in Kiel to complete his PhD in Documentary Linguistics under the supervision of Ulrike Mosel. His initial contact with endangered languages took place as a student assistant in the Chintang and Puma Documentation Project (DoBeS), as well as during an internship fieldwork trip to collect an initial wordlist for the Gurung dialect (Tibeto-Burman) spoken in the Manang District of Nepal. Joshua's first contact with Saami languages came as a student research assistant in the Kola Saami Documentation Project (DoBeS) and included two field trips to the Russian Federation to work on Kildin and Ter Saami; as a result of these trips and his work on the project, he completed his MA thesis on syllable structures and stress patterns in Kildin Saami. He has also written an article (together with M.\,Rießler) on “Documenting the endangered {K}ola {S}aami languages” \citep{riesler-etal2007}.

Since June 2008, he has been coordinator of the Pite Saami Documentation Project (PSDP; funded by the Hans Rausing Endangered Languages Project) and recipient of an Individual Graduate Studentship to carry out the project as part of the HRELP funding (extended funding through August 2011). For the project, he has spent 14 months doing fieldwork in and around Arjeplog to collect data for the project. His PhD dissertation will involve the documentation corpus being collected, a sketch grammar of Pite Saami (the first grammatical description in English) and a detailed analysis of morpho-phonological phenomena in Pite Saami (expected completion in summer 2012).

As part of PSDP and his own PhD project, he has compiled an extensive database for his recordings and results from elicitation sessions, and has developed Kildin and Pite keyboards and maintains a webpage on the project website concerning useful project resources for documentary linguists. He has taught a university class on documentary linguistics in the northern European area, and led a fieldwork course and a Toolbox tutorial at the DoBeS Winter School on “Saami Language Documentation and Revitalization”.

\paragraph{Elena Karvovskaya} is a student of General linguistics at Potsdam university. In 2010 she finished her B.Sc. thesis in Kola Saami semantics which was based on fieldwork and corpus research. She has also co-authored conference presentations and a paper on Kola Saami semantics.

Beside her extensive field work on Kildin and Ter Saami (with M.\,Rießler) and Iźva-Komi (with R.\,Blokland and M.\,Rießler) on the Kola Peninsula she has done field work on Pite Saami (with J.\,Wilbur) and earlier also participated at several other field expeditions in the Caucasus (with Yakov Testelets) and Siberia (both with Olga Kazakevich). As the result of her internships and student assistantship for the Kola Saami Documentation Project and the Languages of West Ambrym, her participation at DoBeS training courses and a DoBeS Winter School in Saami Language Documentation she is familiar with the Documentary linguistics framework in general and with the DoBeS workflows in particular.

Due to her studies and her research E.\,Karvovskaya has acquainted herself with general corpus linguistic methods and tools and programming skills. She has also participated at a computer lexicography workshop organized by Giellatekno in Tromsø and has been working with the creation of Web-based teaching tools for Kildin and Skolt Saami. He is thus also well familiar with practical computer linguistics applications and the methods and workflows applied at Giellatekno.

\paragraph{Andrey Dubovcev} … Lovozero, LEXUS … BA in English, continues MA studies in linguistics (Fernstudium) … earlier student assistant, intern in KSDP … have worked with XML and Kildin Saami electronic Dictionaries … 

\subsection{Project participants' responsibilities}

\paragraph{Trond Trosterud} will coordinate the project work including the setting up and maintaining the project's computer linguistics infrastructure at Giellatekno in Tromsø. He will also work (together with other team members) on ... Saami data and tools%, including the training of native speaker assistants …

\paragraph{Michael Rießler} will administer the project and working as principal researcher with special focus on the Kildin and Ter Saami data and tools.%noch etwas mehr muss hier hin, z.B. LEXUS, DoBeS archive infrastructure oder sowas

\paragraph{Joshua Wilbur} will work as second principal researcher with a half-time position. His contribution will focus on the Ume and Pite Saami data and tools. In addition to his collaboration on the project, he will complete the Pite Saami multimedia archive at ELAR and his doctoral dissertation (planed for 2012).

\paragraph{Ciprian Gerstenberger} will work as computer linguist and programmer for the 

\paragraph{Elena Karvovskaya} will be employed as a \textbf{student research assistant} with 10 hours of work per week for 3 years … M.A. thesis in Saami corpus linguistics.\\

For assistance with collecting and understanding the linguistic data, the project will hire members of the Saamic communities in Sweden and Russia as \textbf{research assistants}. %JW: später heißen sie aber "consultants" - das muß konsequenter sein, oder ein unterschied dazwischen gemacht werden
Ideally, at least one regular assistant for Kildin and one for Pite will be found, and other speakers shall be hired as needed and funds are available. By working closely with speakers, the project hopes to stimulate interest and value in the languages in question, and thus encourage the passing on of linguistic knowledge and language within the respective Saami communities.

%\subsection{Infrastructure}
\subsection{Project intercommunication}
On a daily basis, members of the project will communicate with one another on site and in person, whenever possible. M.\,Rießler and J.\,Wilbur will both work from offices at the Department of Scandinavian Studies in Freiburg; T.\,Trosterud and C.\,Gerstenberger will work from the Giellatekno offices in Tromsø; E.\,Karvovskaya will be based at the Linguistics Department at the University of Potsdam. A.\,Dubovcev will work closely with the project from is home in Lovozero.

For long distance communication between project members in completing everyday project business, the communication opportunities offered by the internet will be taken full advantage of; not only email, but also video/audio chat programs (specifically \textit{iChat}), collaborative text editing software (specifically \textit{SubEthaEdit}) and a revision control system (specifically \textit{Apache Subversion (svn)}) will be used, and indeed already are the means of working communication in Giellatekno and are also frequently used today for communication between the other project applicants. Native speaker consultants will be invited and instructed on how to participate in the same ways, but it is likely that long-distance communication concerning consultants will take place via email and on the telephone.

In addition, the project meetings are planned to provide project members and native speaker consultants with a chance to work together on the project face to face; these meetings are described in more detail in Section \ref{meetings} below.

\section{Work program and time schedule}
\subsection{General work program}

In order to achieve the project's goals, a preliminary plan and time schedule have been developed. This includes a number of project meetings, the purpose of which will be described below; this is then followed by a detailed time schedule for the project.

\paragraph{Project Meetings}\label{meetings}
To begin with, an initial project meeting shall take place in December in Murmansk, Russia (part of the Kildin Saami language area), as a supplement to the "Language Technology for East Saami" workshop. This workshop is already planned and will be funded by the Saami Documentation and Revitalization Network\footnote{Also known as SaamiDocNet, the Saami Documentation and Revitalization Network (\url{http://saamidocnet.uit.no/}) is a funding initiative sponsored by NordForsk (\url{http://www.nordforsk.org/index.cfm}) to support travel, meeting and accommodation expenses for individuals interested in documenting and revitalizing the Saami languages; all individuals involved in the current project proposal are members of SaamiDocNet.}. At this workshop, language technology particularly for eastern Saami languages (including Kildin and Ter) will be presented in a practical way in a series of training sessions. This will cover Giellatekno tools, LEXUS training and the KSDP archive. A variety of other interested individuals from the Saami linguistic community and the relevant language communities will attend. Members of the proposed project as well as community assistants and other interested community members from Russia and Sweden will participate in the initial project meeting in connection with the workshop. In this they will have the chance to discuss the proposed project in detail with one another, discuss the project schedule, duties and responsibilities as well as work directly with the project's assistants.

A second, similar meeting shall take place in the spring of 2011 during a similar workshop planned for the southwestern Saami languages (including Pite and Ume), with the same goals and training sessions as in the first meeting mentioned above. It will take place in either Arjeplog (Pite language territory) or Arvidsjaur (Ume language area)\footnote{The workshop venue has not been decided yet, but SaamiDocNet funding has been approved.}. This second meeting will essentially have the same format and goals, and the same opportunities for project members, as the first meeting, but will have a focus on southwestern Saami languages. It will also be funded by SaamiDocNet.

Note that the workshops described above and funded by SaamiDocNet will take place regardless of the outcome of funding for the current project proposal.

Two additional meetings/workshops are planned for the proposed project, both of which shall include participation by all project members and community assistants/external partners.%so ähnlich schreiben

During the first year of the project, a project meeting connected to a documentation and archiving seminar for assistants in the project will take place in Freiburg or (if the DoBeS technical staff agrees to help plan and carry out this workshop) in Nijemegen. Ideally, this meeting can directly be connected to a DoBeS training workshop, which the assistants will than also be able to attend. The project assistants as well as one representative member of each of the four Saami language communities\footnote{Ideally, these individuals will have worked at least occasionally for the project and be somewhat familiar with its activities and goals.} will participate in the documentation and archiving seminar. This seminar should serve as an opportunity for the involved Saami individuals to become more familiar with the archive and to become exposed to and discuss, together with archivists and documentary linguists, the ideals and problems of collaborative documentation and archiving.%wir müssen hier am ende genau ins budget gucken, wieviel kommen können usw.

The final meeting shall take place in Tromsø at the end of the project to present the project results. It shall take place in connection with a conference on Saami language documentation and language technology organized by the University of Tromsø.

Together, daily communication and the project meetings will fit together in order to meet the project goals in a timely and efficient way, as outlined/described in the following section.

\subsection{Time schedule}%JW: die Tabelle sieht scheiße aus, finde ich - zu viel information drin; man könnte das ganze auch mit "paragraphs" machen, wie unten, aber dann würde ich gerne die leeren Zeilen zwischen paragraphs weg machen. MR: ja, keine Tabelle, lieber paragraphs

\paragraph{Dec 2010}Initial project meeting in Murmansk (funded by SaamiDocNet) for all team members including assistants and main native speaker consultants from all four Saami communities; in connection with the "Language Technology for East Saami" workshop.

\paragraph{Spring 2011}meeting in Arjeplog/Arvidsjaur (SaamiDocNet, finanzierung bereits zugesichert)
all team members including assistants and main native speaker consultants from all four Saami communities

\paragraph{Spring–Summer 2011}
checking/completing/unifying existent Kildin Saami orthographic transcriptions and Russian/English translations at DoBeS archive 
(M. Rießler, with help of E. Karvovskaya and Kildin assistant)
checking/completing/unifying existent Pite Saami orthographic transcriptions and Swedish/English translations
(J. Wilbur, with help of E. Karvovskaya and Pite assistant)
agree on special conventions for spoken texts
(all team members)
processing Kildin and Pite multi-tier annotations [MR: ich weiß nicht ob das "korpustechnisch" so heißt] and transferring to Giellatekno corpus structure (es wie die nordsaamischen Texte auf den gtsvn server packen)
(M. Rießler, J. Wilbur, E. Karvovskaya, with help of T. Trosterud and C. Gerstenberger)

\paragraph{Fall-Winter 2011}
programming/testing morphological parsers for Kildin and Pite (die sind jetzt schon angefangen!), test parsing and creation and making available of automators for Kildin and Pite (Kildin ist jetzt schon angefangen und im Internet!)
(T. Trosterud, M. Rießler, J. Wilbur)
start annotating and archiving Ume Saami recordings from DAUM (E. Karvovskya, Saami assistant)

\paragraph{Winter–Spring 2012}
checking/completing/unifying existent Ter Saami orthographic transcriptions and Russian/English translations at DoBeS archive 
(M. Rießler, with help of E. Karvovskaya and Saami assistant)
start programming/testing the syntactic parsers for Kildin Saami (M. Rießler, T. Trosterud) and Pite Saami (J. Wilbur, T. Trosterud)

\paragraph{Spring–Summer 2012}

\paragraph{Summer–Fall 2012}

\paragraph{Fall–Winter 2012}

\paragraph{Winter–Spring 2013}
meeting Tromsø
finalize project
…

automatic lemmatization and dictionary creation Kildin Pite (M. Rießler, J. Wilbur, C. Gerstenberger)

dictionary export to Oahpa!

multimedia annotation of lexica (Saami assistants)

lexica export to wikipedia

\begin{longtable}{ l l l }
\hline Dec 2010		& Initial meeting in Murmansk in connection with		& team members,\\
				& "Language Technology for East Saami" workshop		&native speaker consultants\\%JW: nur die russischen?
				& (funded by SaamiDocNet) 						& \\
\hline Spring 2011	& second meeting in Swedish Lapland in connection with	& team members,\\
				& "Language Technology for Southwestern Saami" workshop	& native speaker consultants from Sweden and Russia\\
				& (funded by SaamiDocNet)						&\\
\hline August 2011	&Project commences officially						&\\
\hline Summer/Fall	&-checking/completing/unifying existent Kildin Saami		& Rießler, Karvovskaya,\\
	2011			& orthographic transcriptions and Russian/English translations& Kildin assistant\\
				& at DoBeS archive								&\\
				&-checking/completing/unifying existent Pite Saami		& Wilbur, Karvovskaya,\\
				& orthographic transcriptions and Swedish/English translations& Pite assistant\\
				& at ELAR										&\\
				&-agree on special conventions for spoken texts		& all team members\\
				&-processing Kildin and Pite multi-tier annotations;		& Rießler, Wilbur, Karvovskaya\\ %MR: ich weiß nicht ob das "korpustechnisch" so heißt
				&transferring to Giellatekno corpus structure			& Trosterud, Gerstenberger\\
\hline Winter/Spring	&-programming/testing morphological parsers			& Trosterud, Rießler,\\
	2011-2012	&for Kildin and Pite								& Wilbur\\
				&-begin annotating/archiving Ume recordings			& Karvovskaya,\\
				&from DAUM									& Ume/Pite assistant\\
\hline Spring/Summer&-checking/completing/unifying Ter orthographies and	& Rießler, Karvovskaya,\\
	2012			&Russian/English translations for DoBeS				& Kildin assistant\\
				&-begin programming parsers for Kildin				& Rießler, Trosterud\\
				&-begin programming parsers for Pite				& Wilbur, Trosterud\\
\hline Summer/Fall	& & \\
	2012			& & \\
\hline Fall/Winter	& & \\
	2012			& & \\
\hline Winter/Spring	& & \\
	2012-2013	& & \\
\hline Spring/Summer&-final meeting in Tromsø	, finalize project				& all team members\\
	2013			&-presenting project results to communities, archives	& all team members\\
				&-official end of project							& \\
\end{longtable}


\section{Budget summary}

\begin{longtable}{| l | r |}
\hline
Cost items & €\\
\hline
\textbf{A. Personnel expenditure}&\\
\hline
Research personnel & 278,497\\
%\hline
%Other personnel &\\
\hline
\hline
\textbf{B. Running non-personnel costs} & \\
\hline
Travel and accommodation & 10,915\\
\hline
Consumables/other & 1,440\\
\hline
\hline
\textbf{C. Non-recurring expenses} & \\
\hline
Equipment & 8,800\\
\hline
\hline
\textbf{Total A+B+C} & \textbf{~299,652}\\
\hline
\multicolumn{2}{| l |}{\textbf{Please note:}}\\
\multicolumn{2}{| l |}{\textbf{Administration overheads and valuable added tax for personnel}}\\
\multicolumn{2}{| l |}{\textbf{costs are not covered by the Volkswagen Foundation.}}\\
\multicolumn{2}{| l |}{\textbf{Personnal funds for foreign partner should be estimated at the}}\\
\multicolumn{2}{| l |}{\textbf{standard rates of the relevant country}}\\
\hline
\end{longtable}

\newpage
\section{Budget details and justification}
\subsection*{A Personnel expenditure}
T.\,Trosterud and C.\,Gersten\-berger both already have permanent positions at the University of Tromsø and would be allowed to work on the project as part of their job responsibilities. %JOSH – bitte schreiben: Rießler=postdoc 34 monate, Bezahlung fängt zwei Monate später an, weil er bis einschließlich September von Uni Freiburg bezahlt wird, teaching hört aber schon im juli auf; wilbur=doktorand; 2 Hilfskräfte, die von zu Hause arbeiten
\noindent \textbf{Salaries}\\
\begin{longtable}{| l | l | r |}
\hline
M.\,Rießler&total costs as&\\
13 TV-L&projected by the&\\
34 month&university's administration&166,600\\%in Freiburg payed until September, project starts already August 
\hline
J.\,Wilbur&total costs as&\\
13 TV-L&projected by the&\\
36 month (50\%)&university's administration&88,200\\
\hline
E.\,Karvovskaya&total costs as&\\
40 working hours/month& projected by the&\\
36 month&university's administration&16,497\\
\hline
A.\,Dobovcev	&Werkvertrag with	&\\
(Lovozero)	&Freiburg university	&\\
			&200 €/month		&7,200\\
\hline
\end{longtable}

\subsection*{B Running non-personnel costs}
\paragraph{Travel and accommodation}
We estimate the following average travel costs:\\

\noindent The regular per diem remunerations based on \textit{Auslandstagegeld} and \textit{Auslandsübernachtungsgeld}\footnote{Based on “Auslandsreisekostenverordnung des Landes Baden-Würtemberg”.} come to € 62 for The Netherlands and € 85 for Norway.
%peter steggo
%nisse
%andrey
%kola-speaker

\begin{longtable}{| l | l | r |}
\hline
\hline
2012		& 2 x Freiburg-Nijmegen&500\\
spring	& 1 x Potsdam-Nijmegen&300\\
		& 2 x Tromsø-Nijmegen&1,000\\
		& 1 x Arvidsjaur-Nijmegen&500\\
		& 1 x Arjeplog-Nijmegen&500\\
		& 2 x Murmansk-Nijmegen&1,000\\
		& 5 days x 9 pers The Netherlands&2,790\\
\hline
2013		& 2 x Freiburg-Tromsø&800\\
summer	& 1 x Potsdam-Tromsø&400\\
		& 2 x Murmansk-Tromsø&600\\
		& 1 x Arvidsjaur-Tromsø&200\\
		& 1 x Arjeplog-Tromsø&200\\
		& 5 days x 5 pers Norway&2,125\\
\hline
\end{longtable}

\noindent \textbf{Consumables/Other}
\begin{longtable}{| l |  r | r | r |}
\hline
Item	&Price	&Qty.		&Total\\
	&	&per unit	&price\\
\hline		
prepaid internet/phone&40€/month&36 months&1,440\\
for assistant in Russia&&&\\
\hline
\end{longtable}

\subsection*{C Non-recurring expenses}
\noindent desktop computer for principal investigators, notebook for assistants; photokameras, videocameras (for multimedia lexicon work done by assistants) are available from PSDP, KSDP; we are using UNIX/open source/etc. and do not need special software, except a few editors such as SubEthaEdit, Oxygene%What software do we need? %do we need other equipment except computers?
\begin{longtable}{| l | l | r | r | r |}
\hline
Item&Model&Price&Qty.&Total\\
	&&per unit&&price\\
\hline
multimedia&iMac 27”&2,000&2&4,000\\
desktop computer&&&&\\
\hline
all-in-one&MacBook Pro&2,000&2&4,000\\
notebook&&&&\\
\hline
software&	  &	&4&800\\
\hline
\end{longtable}

\newpage
\section*{Appendix}
\subsection*{Curriculum vitae and list of publications for the key participants}

\end{document}

%NOTES

%Lena: the urgency of work with the collected data should not be underestimated. Even though we already have a rich documentation, the languages are still endangered. Together with the the language community we can still make a lot with the texts and interviews which we have recorded. We have the community members who are interested and there are still speakers left. However if the speakers are gone, the work on the recordings becomes extremely difficult. It is problematic with the DoBes archives. They should be analyzed as soon as possible, but most of the data is only annotated.

%Lena: does anybody know what these guyes are doing? http://www.rosettaproject.org/ The Rosetta Project:A meaningful survey and near permanent archive of 1000 languages. JOSH: as far as I can tell, they only store printed/image data so far, so they could in theory have our transcriptions/annotations, but we are also very concerned about linking these to the actual recordings; but the Rosetta Project has serious foresight!
%and http://www.endangeredlanguagefund.org/ Endangered Languages Fund: Devoted to the scientific study of endangered languages and to the support of native efforts in maintaining endangered languages. Helping to stem the tide of language dissapearance in the world. JOSH: as far as I can tell, they fund similar projects like HRELP and Dobes, but the archive is purely symbolic and for any serious comparative or corpora linguistic work essentially useless, at least the online access is, and I can't find any information about future plans for the archive - maybe I missed something, but maybe they don't plan to make archived materials available on-line.
%MR: we don't need to mention these projects

%WÄRE ES NICHT SCHÖN, EINE TOLLE NEUE KAMERA ZU KAUFEN? 
%MR: ist leider für dieses Projekt unlogisch sowas im Budget zu haben

%note that an preliminary (not yet established) orthography is used in the case of Pite and Ume  
%Lena: using the spell-checkers we  could also annotate the existing Saami texts which have been published earlier. Like Образцы саамской речи. We could use an OCR (Optical character recognition) programm like Fine Reader. Including older texts in the corpora will be very helpful for future researchers and for the community; JOSH: in theory, definitely a good idea, but practically the question is whether it is realistic to include legacy - I originally planned to work on legacy materials for Ume in my original HRELP application, but I have my hands full with the recordings I've done on my own (of course, I'm also not the most industrious/productive linguist alive) LENA: you are right. but I thought  if you really create the automates you can add legacy materials later.

%The main goal of the projects was (and is) to take the result of academic research in Saami language technology and turn it into useful tools for the language community, and by that giving back to the community that made the academic research possible in the first hand. There is also hope that making such tools will help with (re)vitalizing the languages, and strengthen their use and position in the society in general.

%The feedback from the Saami community has been very positive, and shows that there is a strong need for these products, as well as the importance of doing the research required to enable such projects. The required research covers at least lexicography, linguistics and computational linguistics. To study the effects of making such tools, sociolinguistic research would be required, but so far this is not done. The network applied for here will help initiate research projects to both study the sociolinguistic effect of the tools already made, and to do further linguistic research to enable similar projects for other Saami languages.

