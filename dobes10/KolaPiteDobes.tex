%!!XeLaTeX (not LaTeX)!!
\documentclass[a4paper,12pt]{article}

\usepackage{fontspec}
\usepackage{xunicode}
\usepackage{textcomp}
\usepackage{graphicx}

\usepackage[english]{babel}

\usepackage{url}

\usepackage{colortbl}
%tags: %\rowcolor[gray]{0.8} %\columncolor[gray]{0.8}

\usepackage{subscript}

\usepackage{linguex}

\usepackage{lscape}

\usepackage{longtable}

\usepackage{tabularx}

\setromanfont{Arial}

\usepackage{natbib}
\bibpunct[: ]{(}{)}{;}{a}{}{;}

\begin{document}
\urlstyle{same} %%JW added this because the url-package was changing the url-font to something totally illegible. "same" gives the url the same font as the surrounding text
\newfontinstance\scshape[Letters=SmallCaps,Numbers=Uppercase]{Hoefler Text}

\begin{flushleft}
\begin{Large}
\textit{Funding Initiative\\
\textbf{“Documentation of Endangered Languages”}}\\\bigskip
\end{Large}

\textbf{VolkswagenStiftung}\\
Dr.\,Vera Szőllősi-Brenig\\
Kastanienallee 35\\
30519 Hannover\\
GERMANY
\end{flushleft}

\begin{flushright}
\today
\end{flushright}

\begin{flushleft}
\begin{tabularx}{\textwidth}{ l | X }
\hline
\multicolumn{2}{>{\large\columncolor[gray]{0.8}} c }{Personal data and adresses}\\
\multicolumn{2}{>{\large\columncolor[gray]{0.8}} c }{Applicant(s), cooperation partner, grant recipient}\\
\hline
\multicolumn{2}{>{\large\columncolor[gray]{0.8}} l }{Principal applicant}\\
\hline
\hline
\textbf{Family Name} & {\textbf{Rießler}}\\
\hline
\textbf{First Name} & {\textbf{Michael}}\\
\hline
Female / Male & {Male}\\
\hline
Titel & {M.A. (Dr.\,phil expected 2010)}\\
\hline
Field of Study & {Scandinavian linguistics}\\
\hline
\hline
\textbf{Institution} & \textbf{Albrecht-Ludwigs-Universität Freiburg i.\,Br.}\\
\hline
\textbf{Department} & \textbf{Skandinavisches Seminar}\\
\hline
Street & {Platz der Universität 3}\\
\hline
Postcode & {79085}\\
\hline
City & {Freiburg}\\
\hline
Country & {Germany}\\
\hline
Phone No & {+49-761-203-3300}\\
\hline
Mobile Phone No & {+49-179-9441585}\\
\hline
Fax No & {+49-761-203-3366}\\
\hline
E-Mail Address & \url{mailto:michael.riessler@skandinavistik.uni-freiburg.de}\\
\hline
Homepage & \url{www.skandinavistik.uni-freiburg.de/institut/mitarbeiter/riessler/}\\
\hline
\end{tabularx}
\end{flushleft}

\noindent The application has not been / will not be submitted to other funding institutions.\\

Signature\\
%\begin{figure}[htbp]
%\begin{center}
%\includegraphics[width=3cm]{signature.jpg}
%\end{center}
%\end{figure}

\noindent \textit{\textbf{Please note that the Volkswagen Foundation – in accordance with regulations safeguarding data privaty – records electronically your personal data as well as the project proposal.}}

\newpage

\begin{flushleft}
\begin{tabularx}{\textwidth}{ l | X }
\hline
\multicolumn{2}{>{\large\columncolor[gray]{0.8}} l }{Co-applicant}\\
\hline
\textbf{Family Name} & {\textbf{Trosterud}}\\
\hline
\textbf{First Name} & {\textbf{Trond}}\\
\hline
Female / Male & {Male}\\
\hline
Field of Study & {…}\\
\hline
\hline
\textbf{Institution} & {\bf{Universitetet i Tromsø}}\\
\hline
\textbf{Department} & {\textbf{…}}\\
\hline
Postcode & {…}\\
\hline
City & {…}\\
\hline
Country & {Norway}\\
\hline
Phone No & {…}\\
\hline
Fax No & {…}\\
\hline
E-Mail Address & {…}\\
\hline
Homepage & {…}\\
\hline
\end{tabularx}
\end{flushleft}

\newpage

\begin{flushleft}
\begin{tabularx}{\textwidth}{ l | X }
\hline
\multicolumn{2}{>{\large\columncolor[gray]{0.8}} l }{Co-applicant}\\
\hline
\textbf{Family Name} & {\textbf{Gerstenberger}}\\
\hline
\textbf{First Name} & {\textbf{Ciprian}}\\
\hline
Female / Male & {Male}\\
\hline
Field of Study & {…}\\
\hline
\hline
\textbf{Institution} & {\bf{Universitetet i Tromsø}}\\
\hline
\textbf{Department} & {\textbf{…}}\\
\hline
Postcode & {…}\\
\hline
City & {…}\\
\hline
Country & {Norway}\\
\hline
Phone No & {…}\\
\hline
Fax No & {…}\\
\hline
E-Mail Address & {…}\\
\hline
Homepage & {…}\\
\hline
\end{tabularx}
\end{flushleft}

\newpage

\begin{flushleft}
\begin{tabularx}{\textwidth}{ l | X }
\hline
\multicolumn{2}{>{\large\columncolor[gray]{0.8}} l }{Co-applicant}\\
\hline
\textbf{Family Name} & {\textbf{Wilbur}}\\
\hline
\textbf{First Name} & {\textbf{Joshua Karl}}\\
\hline
Female / Male & {Male}\\
\hline
Field of Study & {Documentary Linguistics}\\
\hline
\hline
\textbf{Institution} & {\bf{Humboldt-Universität zu Berlin}}\\
\hline
\textbf{Department} & {\textbf{Nordeuropa-Institut}}\\
\hline
Postcode & {10099}\\
\hline
City & {Berlin}\\
\hline
Country & {Germany}\\
\hline
Phone No & {+49-30-2093-4850}\\
\hline
Fax No & {+49-30-2093-9626}\\
\hline
E-Mail Address & {\url{mailto:wilburjk@staff.hu-berlin.de}}\\ %%in the typeset pdf, this is displayed as a url with the phrase "mailto:" - looks stupid on paper - can this be changed?
\hline
Homepage & {\url{http://www.ni.hu-berlin.de/personal/jwil/jwil_html}}\\
\hline
\end{tabularx}
\end{flushleft}

\newpage

\section*{Project proposal}

\section{Basic information}

\begin{tabbing}
LLLLLLLLinks \= Mitte \= Rechts \kill
Project title: \>\textbf{Language Technology for Small Saamic Languages}\\
%alternative titles:
%Multimedia corpora and language technology for highly endangered Saami languages
%Computer-aided analyses of Saami language corpora
%Spoken corpora and computer linguistics for small Saami languages
Total budget: \>\textbf{250,000 €}\\
Project period: \>\textbf{October 2011 – September 2014}\\
\end{tabbing}

\section{Abstract}%1–2 pages

The project aims at creating a state-of-the-art corpus infrastructure and language technology tools for Ume, Pite, Kildin and Ter Saami.

Content: Documentation, Archiving, Automatic annotation, Computer-aided tools for language teaching

Hauptziel des Projektes ist es also sprachtechnologische Werkzeuge (Automaten) zu nutzen um die digitalisierten, transliterierten und übersetzten Texte automatisch annotieren zu können und mit Hilfe des annotierten Korpus schließlich praktische Anwendungen, wie Paradigmen- und Wortformengenerator, interaktive Lehrmaterialien oder elektronische Lexika, zu erstellen.

Sprachen im Fokus sind:

-Kildin (highly endangered, with orthography, with some description and documentation)

-Pite und Ter (moribund, without [established] orthography, with little description and documentation)

-Ume (practically extinct, without [established] orthography, with very little description and documentation)

Community driven revitalization efforts are done (to different extents) for all four languages

Für alle diese Sprachen gibt es schon multimedial annotierte Korpora, die Josh in seinem Projekt zu Pite und wir in unserem Projekt zu Kolasaami hergestellt haben. Man kann die Annotationen synchron zum Audio/Video anhören/ansehen (mit dem Programm ELAN). Man kann auch etwas in den Annotationen suchen with the help of ELAN locally or, for the DoBeS materials, online with the help of ANNEX/TROVA. %, aber nur sehr limitiert.

Ziel des Projektes soll sein:
-ein suchbares, für linguistische Forschung nützliches und mit Multimedia verlinktes Korpus gesprochener Sprache herzustellen (wie Giellatekno für Nordsaamisch, aber mit Multimedia)
-dieses Korpus automatisch weiter zu annotieren (ungefähr wie Giellatekno es für Nordsaamisch macht)
-Sprachtechnologische Tools für Linguisten und Revitalisierer zur Verfügung zu stellen (Lemmatisierer, intelligente Wörterbücher, Übersetzungstools usw., so wie Giellatekno es jetzt schon für Nordasaamisch macht)
-Tools, Methoden, Workflows und Konventionen zu entwickeln, wie in Zukunft auch andere DOBES- und ELAR-Archive bedrohter Sprachen für echte Korpuslinguistik und Sprachtechnologie nutzbar gemacht werden können


The languages are all highly endangered or moribund. They are spoken in Sweden and Russia respectively and belong genealogically to distant subbranches of the Saamic group inside Uralic: Ume Saami belongs to the southern branch of West-Saami, Pite Saami belongs to the central branch of West-Saamic, Kildin and Ter Saami belong to the Peninsula branch of East-Saamic.

In the framework of contemporary documentary linguistics … DOBES and ELAR archives under creation … Ume without documentation …

The goals of the project concern … as well as work with the language communities. To achieve these aims, the project will …

Ultimately, the project should result in the creating and … The survival of the very object of such research, the language itself, is of course crucially dependent on an active community of speakers. In order to support these individuals and their community as well as to promote active use of the endangered Saamic languages, the project will actively contribute its energies and resources to the language communities …

The team consists of experienced documentary and computer linguists and programmers familiar with Saamic languages. Team members have excellent knowledge in the Saami languages in questions, in corpus- and computer linguistics, and in practical revitalization issues. The project will also profit from the symbioses between researches familiar with the Giellatekno infrastructure (including tools for Northern languages) and researchers with the DOBES/ELAR infrastructure (Documentary linguistics of spoken language, archiving of annotated multimedia)

Das beantragte Projekt kann auf die Erfahrungen des {Giellatakno-Projektes} (Tromsø, http://giellatekno.uit.no/) zurückgreifen, mit dem M.\,Rießler und J.\,Wilbur bereits zum Pite- und KolaSaamischen zusammen arbeiten. 

\section{Detailed project description}%approx 15 pages
\subsection{Introduction}

DOBES/ELAR programs have the aim to create data orientated, multifunctional, and generally accessible documentations of endangered languages in order to ... against linguistic and cultural extinction … 
 
… a logical continuation from the perspective of both language and research communities is the corpus linguistic upgrading/appreciation/
revaluation of the already existing documentations … making them useful for corpus linguistic research which automatically also results in practical tools (dictionaries, translation tools, etc.) …

THIS IS EXACTLY WHAT WE ARE PLANNING

\paragraph{what is “corpus linguistics”?} Corpus linguistics=study of language as expressed in "real world" text
%korpuslinguist ist jeder der mit hilfe von korpora etwas erforscht. aber unser hauptziel soll es sein eine _korpuslinguistische infrastruktur_ zu schaffen, damit man später aus DOBES mehr machen kann

\paragraph{what is computer linguistics?} Computer linguistics=maschinelle Verarbeitung natürlicher Sprache
%-Entwicklung von Analyse- und Generierungsverfahren für natürlich-sprachliche Texte
%-Programme zur Sammlung und statistischen Auswertung großer Mengen von Sprachdaten (Lemmatisierung, Häufigkeitswortlisten, Konkordanzen)
%-Praktische Anwendungen: maschinelle Übersetzung, computergestützter Sprachunterricht
%-Giellatekno already has all this for Northern Saami, partly also for Lule and South Saami, very little even for Kildin Saami

\paragraph{what are corpora?}
%DOBES enthält Korpora, in denen man aber inhaltsmäßig nur relativ beschränkt suchen kann, wegen inkonsequenter (manueller) Annotation
%Giellatekno corpus largely derived by automated processes (morphological+syntactic automators)

\paragraph{what are multimedia vs. written corpora?}
%DOBES/ELAR corpora largely derived by hand (ELAN, Toolbox)

\paragraph{what is language technology?}
%TROND??

\paragraph{why Saami?}
%-because computer linguistic infrastructure already available for closely related (linguistically and culturally) written Northern Saami
%-because computer linguistic know-how already available for simple and parallel written corpora of Northern Saami

\paragraph{what Saami?}
%Pite Saami: annotated corpus available at ELAR, linguist (for programming morphological (and syntactic) automators) available
%Kildin, Ter Saami: annotated corpus available at DOBES, linguists (for programming morphological (and syntactic) automators) available
%Ume Saami: the most endangered Saamic language (probably only one living L1 speaker); almost no documentation; revitalization activity by speaker community 

\subsection{Urgency of documentation}
MICHA/JOSH: some info on our languages  

\subsection{State of research}
\subsubsection{Language documentation}
grammars, dictionaries, text collections

\subsubsection{The DoBeS archive for Kola Saami}
JOSH/MICHA

\paragraph{Corpus search tools at DoBeS}
\begin{itemize}
\item Annex/Trova: "the Annotation Exploration tool in the MPI web-based framework for archive exploration and enrichement" - see IMDI-Browser!!
\item Arbil: "IMDI Metadata Editor, Browser \& Organizer Tool" only offline, thus only for locally-stored data; search function not yet integrated
\item IMDI-Browser: search entire MPI corpus collection (DoBeS + many more):
	* in "metadata" fields (in the DoBeS sense of cataloging metadata)
	* in "annotation" fields, including display of all hits within context and links to web-based display of ELAN file in ANNEX, inludes list of 8 words on either side of hit, very complex context search using web-based TROVA annotation search; search results can be downloaded/saved
\item ELAN: elaborate searching, including using regular expressions, within a single ELAN file or across several files, including multiple layer search (same software/GUI as with ANNEX/TROVA); search results can be saved.	
\end{itemize}

%KWIC-Suche (key word(s) in context)
%ich glaube diese kexwords kann man über imdi eingeben
%KWIC ist aber für  linguistische und anthropologische Forschung (sowie für Revitalizierung) nur sehr beschränkt nützlich

\subsubsection{The ELAR deposit for Pite Saami}
JOSH

\paragraph{Corpus search tools at ELAR} 
\begin{itemize}
\item very basic character-string search (not even as advanced as a google search!) of a single project's deposit only. I assume more will be possible in the future?
\item ELAN annotations can be searched (off-line) using the DOBES tools
\end{itemize}

\subsubsection{The Saami language technology project}
TROND … write specifically, what is preliminary already there or in the works for Pite and Ume and Kildin and Ter

\subsubsection{Corpus linguistics projects at ELAR, DOBES and ??other}

\subsection{Research aims and methodologies}

\subsubsection{Giellatekno} HERE WE SHOULD MENTION GIELLATEKNO'S GENERAL RESEARCH AIMS AND METHODOLOGIES

… A FEW MORE SPECIFIC THINGS, LIKE:
\paragraph{Two-Level-Morphology}
Zur computerlinguistischen Verarbeitung der sprachlichen Daten wird vorerst die s.g. Zwei-Ebenen-Morphologie (TWOL) angewendet. In TWOL wird die Zuordnung von Morphemen auf zwei Ebenen beschrieben:
\begin{enumerate}
\item morphologische/lexikalische Ebene: Struktur eines Lexems, Morphe, Morphemgrenzen
\item morphographischen Ebene: Oberflächenrealisierung des Lexems
\end{enumerate}
Die erlaubten bzw. verbotenen Zuordnungen von Morphemen auf beiden Ebenen erfolgt mittels (sprachspezifischer) Regeln. Jede Regel wird von einem Transducer in die Realisierung auf der jeweils anderen morphologischen Ebene übersetzt. Somit können morphologische Formen nicht nur analysiert (Übersetzungsrichtung morphographemische Ebene $\rightarrow$ morphologische/lexikalische Ebene), sondern auch generiert werden (Übersetzungsrichtung morphologische/lexikalische Ebene (Stamm+Affixe) $\rightarrow$ morphographemische Ebene (Oberflächenrealisierung)).

Der TWOL-Ansatz eignet sich besonders zur Beschreibung von Sprachen mit einem komplexen und sich durch Irregularitäten und Stammallomorphie auszeichnenden Formensystem – wie z.B. Saamisch. 

\paragraph{Zwei-Ebenen-Morphologie} (two-level morphology, TWOL) ist ein theoretischer Ansatz, in dem die Struktur eines Lexems auf der morphologisch-lexikalischen Ebene in Relation zu dessen Oberflächenrealisierung, d.h. der Kodierung des Lexems auf der morphographischen Ebene, gesetzt wird. Erlaubte bzw. verbotene Korrespondenzen zwischen Zeichen auf den zwei Ebenen werden durch abstrakte Regeln erfasst.

Um Übergeneralisierung bei der durch einen Automaten (Transduktor) angefertigten Übersetzung zu vermeiden, müssen die Regeln durch die Anbindung eines (Morphem-)Lexikons sowie durch Angaben über Morphemkombinatorik auf der morpho-lexikalischen Ebene eingeschränkt werden.

Das Lexikon stellt eine Liste aller gebundenen und ungebundenen Morpheme dar. Ein vorläufiges Morpheminventar wird aus den bereits vorhandenen Toolboxprojekten von KSDP und PSDP importiert. Hauptziel nach abgeschlossener Arbeit mit Toolbox ist es aber weitere Texte mit Hilfe der Automatoren zu parsen und das Morpheminventar damit automatisch zu erweitern.

Die Listen mit Stämmen und der ihnen zugeordneten morphologischen Markierungen werden in TWOL mittels eines mit Indizes versehener Listenverkettungsmodells (indexed concatenation model) miteinander verbunden. Dies bedeutet, dass phonologische oder graphemische Morphemvarianten eines Stamms mit unterschiedlichen Indizes versehen werden, die der Art der zugelassenen Folgekategorien (d.h. möglichen Suffixen für einen gegebenen Stamm) entsprechen. So erhält ein Stamm, dem ein bestimmte Gruppe Suffixe folgen kann, eine andere Indizierung als ein Stamm, dem eine andere Gruppe Suffixe folgen kann. Auch die Suffixe im Lexikon erhalten Indizes, die angeben, mit welcher Art indizierter Stämmen sie kombiniert werden dürfen. Die Aufgabe des Listenverkettungsmodells ist es, die Liste der Wortstämme mit der möglicher Suffixe entsprechend der Indizes einzelner Elemente in diesen Listen miteinander zu verketten.

\textit{Giellatekno} wendet TWOL bei der Erstellung von sprachtechnologischen Anwendungen wie interaktive pädagogische Programme oder Rechtschreibprüfprogramme für morphologisch komplexe Sprachen mit Stamm- und Flexionsvariation an, darunter v.\,a. Nordsaamisch, aber auch andere saamische Sprachen sowie auch Grönländisch, Kveeni, Komi

\paragraph{Syntaktische disambiguierung}

\paragraph{automatic translation} 

\paragraph{the automatic-dictionary-thing} Mit Hilfe des Lexikons und der zu programmierenden morphologischen und später auch auch syntaktischen automaten können am Ende neu eingegebene bzw. importierte Texte automatisch analysiert sowie Paradigmen einzelner Lexeme generieren werden.

\paragraph{Wortformgenerator} Die bereits im Laufe des Pilotprojektes zu erstellenden \textbf{Paradigmen-} und \textbf{Wortformengeneratoren} können eine praktische Funktion im Sprachunterricht übernehmen. In Korpora lassen sich i.d.R. nicht alle Formen eines Paradigmas finden. Mithilfe von TWOL (aber im Unterschied zu Programmen wie Toolbox) können Wortformen jedoch nicht nur analysiert, sondern auch generiert werden. Komplette Paradigmen lassen sich somit für jeden Eintrag in der Datenbank in Echtzeit generieren oder z.B. in das geplante interaktive Lexikon einbinden.

Dementsprechend gibt der Wortformengenerator bei vorgegebener Grundform eines Lexems und der gewünschten morphologischen Markierungen die entsprechende grammatische Form aus (z.B. input: hestur+ACC $\rightarrow$ output: hest).%besser saamisches Beispiel\\

Sobald in einer späteren Projektphase auch ein syntaktischer Automat programmiert worden ist, könnte zusätzlich sogar ein \textbf{Interaktives Syntax-Lernprogramm} erstellt werden. Dabei liefert zuerst die morphologische Analyse auf Wortebene die jeweilige Form. Ein syntaktischer Disambiguator eliminiert daraufhin auf Satzebene die inkorrekten Analysen und versieht die korrekten Analysen mit entsprechenden syntaktischen Funktionen (Subjekt, Objekt usw.). Schließlich könnte eine Phrasenstrukturgrammatik der linearen Darstellung eine hierarchische Struktur zuordnen.

\paragraph{spell checker} use Divvun experience with Northern Saami

\paragraph{hyphenation} use Divvun experience with Northern Saami

\paragraph{Oahpa!}

\subsubsection{DoBeS}

\paragraph{LEXUS} as interface for multimedia annotation (done by assistants) and web-based presentation, export to off-line dictionary applications, export to wikipedia

\paragraph{ELAN}

\paragraph{the dobes-corpus-search-thing}

\subsubsection{ELAR}

\subsubsection{methodological problems}

Wir würden gern Giellateknos Methoden und Worksflows verwenden. Aber unser geplantes Korpus unterscheidet sich von Giellatekno für Nordsaamisch:

-gesprochene Sprache (hohe Variation durch Dialekte und durch language attrition/language loss; sehr viel code-switching; sehr viel hezitations, false starts, self-corrections usw.; Intonationseinheiten anstatt Sätze)

-keine etablierte Orthographie (für Pite und Ter)

-kleinerer Umfang des Korpus (Anzahl der Token)

-komplett mehrsprachiger (paralleler) Korpus: Pite-Schwedisch-Englisch bzw. Kildin/Ter-Russisch-Englisch (weil alle unsere Transkriptionen übersetzt sind)

-Link zu Multimedia ist obligatorisch!


\subsubsection{written vs. spoken coprora} 
CORPUS OF WRITTEN LANGUAGE

we have many and large corpora of written languages

-CORPUS OF SPOKEN LANGUAGE

--we have much less and smaller corpora of spoken language (e.g. dialects, sociolects)

--written representation of text

--different units than in text

--linguistic features characteristic of spoken language (intonation, hesitation, false start, etc.)

---CORPUS OF AN ENDANGERED SPOKEN LANGUAGE

---we have very few and only very small corpora of endangered languages

---written representation of text

---different units than in text

---linguistic features characteristic of spoken language (intonation, hesitation, false start, etc.)

---code switching to other languages

---language attrition

\subsubsection{The “spoken corpus” issue}
…this is mostly an issue regarding annotation conventions

\subsubsection{The “orthography issue”}
… since we are working with the orthographic representation of our texts and we are producing practical tools for language users … we have to discuss in some detail for what languages orthographies are already established (Kildin is established but in several variants; Pite is under development by community members in collaboration with Josh; Ter could use the Kildin orthography but community members have to decide on this; Ume is unclear) … and how we deal with orthography issues like variants and possible later changes originating from inside the communities … I know that this is all no great deal from the Giellatekno perspective, but we have to address this for our referees 

\subsubsection{Different users' portals to our created infrastructure and tools}

\paragraph{Giellatekno} portal in multilingual (Norwegian/Swedish, Russian, English, other) localizations is already existent

\paragraph{Project portal} \url{www.saami.uni-freiburg.de} community friendly multilingual portal to ELAR and DOBES archives, to PSDP, KSDP, Tools, etc. 

\paragraph{ELAR} only links

\paragraph{DOBES} only links

\subsubsection{Work flow}
svn/ichat/subethaedit/etc.

Workflow
-manually annotated texts linked to audio and video (ELAN) already available
-create annotation in standard orthography (+using conventions for features characteristic of spoken language)
-create automators
-parse
-create automatically annotated texts linked to audio and perhaps video (ELAN)

rough plan:
1. Kildin Morpho-Automaten schreiben (alle zusammen)
2. Kildin Syntax-Automaten anfangen zu schreiben (alle zusammen)
3. Pite Morpho-Automaten anfangen zu schreiben (alle zusammen)
4. Kildin Syntax weitermachen (Micha+GT)
5. Pite Morpho weitermachen (Josh+GT)
6. Pite Syntax anfangen (alle)
7. Ter morpho-automaten
8. Ume morpho-automaten

%!!Note DOBES requirements
%-we have to use their tools (ELAN, LEXUS, etc.)
%--[ELAN and LEXUS is used already]
%-we have to use their archive (Nijmegen)
%--[no problem] 
%-we have to involve to speech community
%--[hire and train native speaker assistant for work with LEXUS (e.g. create a monolingual multimedia dictionary after automatic lemmatization]

\subsubsection{Expected outcome of the proposed project}
Automatically annotated corpora for spoken Pite Saami, Kildin Saami, Skolt Saami, Ter Saami languages.

practical products:

-tagged (single and parallel [always 2 translations already available at ELAR/DOBES!]) corpora of Saami languages =useful/neded for linguists

dictionaries =useful/needed for linguists and revitalization:

-gt-Automatic dictionary for Kildin, Pite, Ter, Ume
%note that an preliminary (not yet established) orthography is used in the case of Pite and Ter and Ume

-LEXUS-multimedia lexicon =useful/needed for revitalization

-teaching programs (Oahpa!)=useful/needed for revitalization

Spell checker and hyphenation program for Kildin (not the other languages, because no established orthographies yet)

Automatic translation for Kildin, Pite (not Pite and Ume, because too small corpora)
%note that an preliminary (not yet established) orthography is used in the case of Pite and Ume  

…

…

…

Finally, creation of workflows, tools, conventions which can be useful for other DOBES projects as well


\subsection{Cooperation partners}
Kola Saami: Lovozero

Pite/Ume Saami: Peter Steggo

??DAUM: Umesaamische Aufnahmen

??Divvun: spell checker, hyphenation

??SaamiDocNet

\bibliographystyle{linquiry2.bst}
\bibliography{KolaPiteDobes}

\subsection{Community consent}

\section{Binding indication}

\section{Key project participants, their responsibilities in the project and previous work on the subject}

\subsection{Project participants' previous work on the subject}
The team consists of experienced documentary and computer linguists and programmers familiar with Saamic languages.

\paragraph{Michael Rießler} studied Scandinavian and General Linguistics, among others. His research has focused on both historical and synchronic-typological issues, and has covered a variety of topics including (but not limited to) Areal linguistics in Northern Europe, Saami linguistics, and Documentary linguistics. He delivered his doctoral dissertation in General linguistics in July 2010. Defence at the Uni Leipzig expected November 2010.

During the last six years, Michael Rießler has conducted extensive fieldwork on Kola Saami languages. Due to his work as principle investigator and coordinator of the “Kola Saami Documentation Project”, a DoBeS project, he is familiar with the objectives and methods of contemporary documentary linguistics in general and with the framework of the DoBeS program group in particular. He has also organized a DoBeS Winter School on “Saami Language Documentation and Revitalization” (together with Ulrike Mosel, Jurij Kusmenko, Bruce Morén, and Ida Toivonen) which took place in 2010 in Bodø/Norway.

In connection with the documentation of Kola Saami … in collaboration with the Saami language technology group in Tromsø … corpus and computer linguistics … Giellatekno … workshop Tromsø

\paragraph{Trond Trosterud} … Saami linguist, Computer linguist, leader of Giellatekno, employed in Tromsø

\paragraph{Ciprian Gerstenberger} … Computer linguist, programmer, employed in Tromsø

\paragraph{Joshua Wilbur} … HRELP scholarship until 2012 at Humboldt … PhD at Kiel university (Mosel)

\paragraph{Elena Karvovskaya} is a prospective student of Saamic and General linguistics at Potsdam university. In 2010 she finished her B.Sc. thesis in Kola Saami semantics which was based on fieldwork and corpus research. She has participated at several field expeditions in Russia, working on %??caucasus (with ??)
 Ket, Evenki (both with Olga Kazakevich), Pite Saami (with Joshua Wilbur), and Kola Saami (with Michael Rießler). due to  … internships and student assistantship for KSDP and Krifka-DOBES …  due to her work for the DoBeS projects, DoBeS training courses, DoBeS Winter school, etc. she is familiar with the Documentary linguistics framework in general and with DoBeS framework in particular … Giellatekno … workshop Tromsø 
 %Liebe Lena, ich möchte dich gerne im Antrag haben. Wir schreiben jetzt erstmal etwas schönes rein. Falls du später lieber was anderes machen willst, oder weniger Stunden oder sonstwas, kann man das jederzeit ändern. Micha 

\subsection{Project participants' responsibilities}

\paragraph{Michael Rießler} will administer the project. … employed as principal researcher … Kola Saami 

\paragraph{Joshua Wilbur} employed as principal researcher (half-time position until the completion of his doctoral dissertation planed for 2012) … Ume and Pite Saami 

\paragraph{Trond Trosterud} will coordinate the project … his main responsibility, however, will be setting up, maintaining and coordinating the project's technical infrastructure, including the training of native speaker assistants …

\paragraph{Ciprian Gerstenberger} … computer linguist and programmer

\paragraph{Elena Karvovskaya} will be employed as a \textbf{student research assistant} with 15 hours of work per week for 3 years … M.A. thesis in Saami corpus linguistics.\\

For assistance with … the project will hire … members of the Saamic communities in Sweden and Russia as \textbf{research assistants}. In this way the project hopes to encourage and to stimulate the passing on of linguistic knowledge and language within the Saamic community …

\subsection{Infrastructure}
All team members are based at different places in Freiburg (Rießler, Wilbur), Tromsø (Trosterud, Gerstenberger), Sweden (Pite or Ume assistant), Russia (Kola Saami assistant), Potsdam (Karvovskaya) … svn/iChat/Subethaedit/etc. as it is common practice at Giellatekno …

\section{Work program and time schedule}
\subsection{General work program}

four meetings/workshops of project participants under participation of community assistants and other community members: 1. Tromsø, 2. Lujavv'r, 3. Arjeplog/Arvidsjaur, 4. Nijmegen %Finanzierung nur für Projektmitarbeiter, andere saamische Teilnehmer finden in der Regel selber leicht Mittel

\subsection{Time schedule}

check/unify existent Kola Saami orthographic transcription and Russian/English translation (M. Rießler, E. Karvovskaya)

check/unify existent Pite Saami orthographic transcription and Russian/English translation (J. Wilbur, E. Karvovskaya)

annotate and archive the Ume Saami recordings from DAUM (E. Karvovskya, Saami assistant)

continue programming/testing the morphological parser for Kildin Saami (M. Rießler, T. Trosterud) and Pite Saami (J. Wilbur, T. Trosterud)

start programming/testing the morphological parsers for Ter Saami (M. Rießler, T. Trosterud) and Ume Saami (J. Wilbur, T. Trosterud)

start programming/testing the syntactic parsers for Kildin and Ter Saami (M. Rießler, T. Trosterud) and Pite and Ume Saami (J. Wilbur, T. Trosterud)

automatic lemmatization and dictionary creation (M. Rießler, J. Wilbur, C. Gerstenberger)

dictionary export to Oahpa!

multimedia annotation of lexica (Saami assistants)

lexica export to wikipedia

\begin{longtable}{ l l l }
Oct 2010 & Initial meeting/workshop in Tromsø&\\
\end{longtable}

\section{Budget summary}



\begin{longtable}{| l | r |}
\hline
Cost items & €\\
\hline
\textbf{A. Personnel expenditure}&\\
\hline
Research personnel &\\
\hline
Other personnel &\\
\hline
Scholarships &\\
\hline
\hline
\textbf{B. Running non-personnel costs} & \\
\hline
Travel and accommodation &\\
\hline
Consumables/other &\\
\hline
\hline
\textbf{C. Non-recurring expenses} & \\
\hline
Equipment &\\
\hline
\hline
\textbf{Total A+B+C} & \textbf{~250.00}\\
\hline
\multicolumn{2}{| l |}{\textbf{Please note:}}\\
\multicolumn{2}{| l |}{\textbf{Administration overheads and valuable added tax for personnel}}\\
\multicolumn{2}{| l |}{\textbf{costs are not covered by the Volkswagen Foundation.}}\\
\multicolumn{2}{| l |}{\textbf{Personnal funds for foreign partner should be estimated at the}}\\
\multicolumn{2}{| l |}{\textbf{standard rates of the relevant country}}\\
\hline
\end{longtable}

\section{Budget details and justification}
\subsection*{A Personnel expenditure}
\noindent \textbf{Salaries}\\
\begin{longtable}{| l | l | r |}
\hline
M.\,Rießler&total costs as&\\
13 TV-L&projected by the&\\
3 years&university's administration&\\
\hline
E.\,Karvovskaya&total costs as&\\
60 working ours/month& projected by the&\\
3 years&university's administration&\\
\hline
2 Research assistants&Werkvertrag with&\\
in Sweden/Russia&Freiburg university&\\
&?? €/month each&\\
\hline
\end{longtable}

\subsection*{B Running non-personnel costs}
\paragraph{Travel and accommodation}
We estimate the following average travel costs:\\

\noindent The regular per diem remunerations based on \textit{Verpflegungsmehraufwendungen} and \textit{Übernachtungspauschale} come to € 135 for The Netherlands, € ?? for Norway, € ?? for Sweden, and € ?? for Russia.

%\begin{longtable}{| l | l | r |}
%\hline
%2009 	& 2 x Berlin-Nijmegen&400\\
%fall	& 5 days x 2 pers The Netherlands&1,350\\
%	\hline
%2009–	& 4 x Berlin-Serres&1,200\\
%2010	& 1 x Sliven-Serres&50\\
%	& 14 days x 5 pers Greece&4,200\\
%	\hline
%2010	& 2 x Berlin Sliven&600\\
%spring	& 30 days x 2 pers Bulgaria&3,000\\
%\hline
%2010	& 2x Serres or Sliven-Nijmegen&800\\
%spring	& 1x Berlin-Nijmegen&200\\
%	& 5 days x 3 pers The Netherlands&2,025\\
%\hline
%2010	& 3 x Berlin-Strumica&900\\
%summer	& 1 x Serres or Sliven-Strumica&50\\
%	& 1 x 10 days Macedonia (Rießler)&700\\
%	& 25 days x 3 pers Macedonia&5,250\\
%\hline%hier
%2010	& 2 x Berlin-Serres&600\\
%fall	& 30 days x 2 pers Greece&3,600\\
%\hline
%2011	& 2 x Berlin-Sliven&600\\
%spring	& 30 days x 2 pers Bulgaria&3,000\\
%\hline
%2011	& 1 x Berlin-Sofia (Rießler)&300\\
%spring	& 1 x Serres-Sofia&50\\
%	& 1 x Sliven-Sofia&50\\
%	& 5 days x 5 pers Sofia&1,250\\
%\hline
%2011	& 1 x Berlin-Nijmegen&200\\
%spring	& 5 days x 1 pers The Netherlands&675\\
%\hline
%2011	& 3 x Berlin-Serres&900\\
%summer	& 2 x 10 days Greece (Rießler, Voß)&1,200\\
%	& 2 x 30 days Greece (Leluda-Voß, student)&3,600\\
%\hline
%2011–	& 1 x Berlin-Sliven&300\\
%2012	& 1 x 30 days Bulgaria&1,500\\
%\hline
%2012	& 2 x Berlin-Strumica&600\\
%spring	& 1 x 10 days Macedonia (Voß)&700\\
%	& 1 x 20 days Macedonia (Leluda-Voß)&1,400\\
%	\hline
%2012	& 1 x Berlin-Nijmegen&200\\
%spring	& 1 x Sliven-Nijmegen&400\\
%	& 5 days x 2 pers The Netherlands&1,350\\
%\hline	
%2012	& 3 x Berlin-Sliven&900\\
%summer	& 5 days x 1 pers The Netherlands&675\\
%	\hline
%2012	& 2 x Serres/Sliven-Berlin&600\\
%fall	& 2 x 10 days Germany (assistants)&500\\
%\hline
%2011	& 3 x Berlin-Serres&900\\
%spring	& 1 x Sliven-Serres&50\\
%	& 5 days x 4 pers Serres&1,200\\
%\hline
%\end{longtable}

%\noindent \textbf{Consumables/Other}
%\begin{longtable}{| l |  r | r | r |}
%\hline
%Item	&Price	&Qty.		&Total\\
%	&	&per unit	&price\\
%\hline		
%prepaid internet/phone&20€/month&2x36 months&1,440\\
%for Assistants&&&\\
%\hline
%Mini-DV tapes&&ca. 100&300\\
%\hline
%DVD-RWs&&ca. 100&300\\
%\hline
%Batteries	&&&300\\
%\hline	
%Stationary and photocopy costs&&&	700\\
%\hline
%\end{longtable}

\subsection*{C Non-recurring expenses}
\noindent desktop computer for principal investigators, notebook for assistants; photokameras, videocameras (for multimedia lexicon work done by assistants) are available from PSDP, KSDP
\begin{longtable}{| l | l | r | r | r |}
\hline
Item&Model&Price&Qty.&Total\\
	&&per unit&&price\\
\hline
multimedia&iMac 27”&…&2&…\\
desktop computer&&&&\\
\hline
all-in-one&MacBook Pro&2,000&3&6,000\\
notebook&&&&\\
\hline
\end{longtable}

\newpage
\section*{Appendix}
\subsection*{Curriculum vitae and list of publications for the key participants}

\end{document}
