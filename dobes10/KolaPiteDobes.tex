%!!XeLaTeX (not LaTeX)!!
\documentclass[a4paper,12pt]{article}

\usepackage{fontspec}
\usepackage{xunicode}
\usepackage{textcomp}
\usepackage{graphicx}

\usepackage[english]{babel}

\usepackage{url}

\usepackage{colortbl}
%tags: %\rowcolor[gray]{0.8} %\columncolor[gray]{0.8}

\usepackage{subscript}

\usepackage{linguex}

\usepackage{lscape}

\usepackage{longtable}

\usepackage{tabularx}

\setromanfont{Arial}

\usepackage{natbib}
\bibpunct[: ]{(}{)}{;}{a}{}{;}

\begin{document}
\urlstyle{same}
\newfontinstance\scshape[Letters=SmallCaps,Numbers=Uppercase]{Hoefler Text}

\begin{flushleft}
\begin{Large}
\textit{Funding Initiative\\
\textbf{“Documentation of Endangered Languages”}}\\\bigskip
\end{Large}

\textbf{VolkswagenStiftung}\\
Dr.\,Vera Szőllősi-Brenig\\
Kastanienallee 35\\
30519 Hannover\\
GERMANY
\end{flushleft}

\begin{flushright}
\today
\end{flushright}

\begin{flushleft}
\begin{tabularx}{\textwidth}{ l | X }
\hline
\multicolumn{2}{>{\large\columncolor[gray]{0.8}} c }{Personal data and adresses}\\
\multicolumn{2}{>{\large\columncolor[gray]{0.8}} c }{Applicant(s), cooperation partner, grant recipient}\\
\hline
\multicolumn{2}{>{\large\columncolor[gray]{0.8}} l }{Principal applicant}\\
\hline
\hline
\textbf{Family Name} & {\textbf{Trosterud}}\\
\hline
\textbf{First Name} & {\textbf{Trond}}\\
\hline
Female / Male & {Male}\\
\hline
Titel & {PhD}\\
\hline
Field of Study & {Saami linguistics}\\
\hline
\hline
\textbf{Institution} & {\bf{Universitetet i Tromsø}}\\
\hline
\textbf{Department} & {\textbf{Institutt for språkvitskap}}\\
\hline
Postcode & {9037}\\
\hline
City & {Tromsø}\\
\hline
Country & {Norway}\\
\hline
Phone No & {+47-77644763}\\
\hline
E-Mail Address & {trond.trosterud@uit.no}\\
\hline
Homepage & {http://www.hum.uit.no/a/trond/}\\
\hline
\end{tabularx}
\end{flushleft}

\noindent The application has not been / will not be submitted to other funding institutions.\\

Signature\\
%\begin{figure}[htbp]
%\begin{center}
%\includegraphics[width=3cm]{signature.jpg}
%\end{center}
%\end{figure}

\noindent \textit{\textbf{Please note that the Volkswagen Foundation – in accordance with regulations safeguarding data privaty – records electronically your personal data as well as the project proposal.}}

\newpage

\begin{flushleft}
\begin{tabularx}{\textwidth}{ l | X }
\hline
\multicolumn{2}{>{\large\columncolor[gray]{0.8}} l }{Co-applicant and grant recipient}\\
\hline
\textbf{Family Name} & {\textbf{Rießler}}\\
\hline
\textbf{First Name} & {\textbf{Michael}}\\
\hline
Female / Male & {Male}\\
\hline
Titel & {M.A. (Dr.\,phil expected 2010)}\\
\hline
Field of Study & {Scandinavian linguistics}\\
\hline
\hline
\textbf{Institution} & \textbf{Albrecht-Ludwigs-Universität Freiburg i.\,Br.}\\
\hline
\textbf{Department} & \textbf{Skandinavisches Seminar}\\
\hline
Street & {Platz der Universität 3}\\
\hline
Postcode & {79085}\\
\hline
City & {Freiburg}\\
\hline
Country & {Germany}\\
\hline
Phone No & {+49-761-203-3300}\\
\hline
Mobile Phone No & {+49-179-9441585}\\
\hline
Fax No & {+49-761-203-3366}\\
\hline
E-Mail Address & {michael.riessler@skandinavistik.uni-freiburg.de}\\
\hline
Homepage & \url{www.skandinavistik.uni-freiburg.de/institut/mitarbeiter/riessler/}\\
\hline
\end{tabularx}
\end{flushleft}

\newpage

\begin{flushleft}
\begin{tabularx}{\textwidth}{ l | X }
\hline
\multicolumn{2}{>{\large\columncolor[gray]{0.8}} l }{Co-applicant}\\
\hline
\textbf{Family Name} & {\textbf{Gerstenberger}}\\
\hline
\textbf{First Name} & {\textbf{Ciprian}}\\
\hline
Female / Male & {Male}\\
\hline
Field of Study & {Computer linguistics}\\
\hline
\hline
\textbf{Institution} & {\bf{Universitetet i Tromsø}}\\
\hline
\textbf{Department} & {\textbf{Institutt for språkvitskap}}\\
\hline
Postcode & {9037}\\
\hline
City & {Tromsø}\\
\hline
Country & {Norway}\\
\hline
Phone No & {…}\\
\hline
Fax No & {…}\\
\hline
E-Mail Address & {…}\\
\hline
Homepage & {…}\\
\hline
\end{tabularx}
\end{flushleft}

\newpage

\begin{flushleft}
\begin{tabularx}{\textwidth}{ l | X }
\hline
\multicolumn{2}{>{\large\columncolor[gray]{0.8}} l }{Co-applicant}\\
\hline
\textbf{Family Name} & {\textbf{Wilbur}}\\
\hline
\textbf{First Name} & {\textbf{Joshua Karl}}\\
\hline
Female / Male & {Male}\\
\hline
Field of Study & {Documentary Linguistics}\\
\hline
\hline
\textbf{Institution} & {\bf{Humboldt-Universität zu Berlin}}\\
\hline
\textbf{Department} & {\textbf{Nordeuropa-Institut}}\\
\hline
Postcode & {10099}\\
\hline
City & {Berlin}\\
\hline
Country & {Germany}\\
\hline
Phone No & {+49-30-2093-4850}\\
\hline
Fax No & {+49-30-2093-9626}\\
\hline
E-Mail Address & {wilburjk@staff.hu-berlin.de}\\
\hline
Homepage & {\url{http://www.ni.hu-berlin.de/personal/jwil/jwil_html}}\\
\hline
\end{tabularx}
\end{flushleft}

\newpage

\section*{Project proposal}

\section{Basic information}

\begin{tabbing}
LLLLLLLLinks \= Mitte \= Rechts \kill
Project title: \>\textbf{Language Technology for Small Saamic Languages}\\
%alternative titles:
%Multimedia corpora and language technology for highly endangered Saami languages
%Computer-aided analyses of Saami language corpora
%Spoken corpora and computer linguistics for small Saami languages
Total budget: \>\textbf{250,000 €}\\
Project period: \>\textbf{October 2011 – September 2014}\\
\end{tabbing}

\section{Abstract}%1–2 pages %itemize macht, dass es länger wird

The proposed project aims at creating a state-of-the-art corpus infrastructure as well as language technology tools for four endangered Saami languages: Ume, Pite, Kildin and Ter Saami. This shall involve a variety of aspects relevant to the field of modern documentary linguistics: endangered language documentation and archiving practices, automatic annotation, and computer-aided tools for creating lexica and educational materials.

The main goal of the project is twofold: 1) to apply language technology tools (more precisely: finite-state transducers and constraint grammar) to efficiently supplement corpora of digitized, transliterated and translated texts with annotations, and 2) using the resulting annotated corpora, to then create practical applications such as paradigm and word-form generators, interactive teaching materials and electronic lexica. As a result, the needs of both the academic community and the speech communities concerned will be served.

%The following Saami languages will be focused upon:
%
%\begin{itemize}
%\item Kildin (highly endangered, with an orthography, with some description and documentation)
%\item Ter (moribund, without an [established] orthography, with little description and documentation)
%\item Pite (moribund, without an [established] orthography, with little description and documentation)
%\item Ume (practically extinct, without an [established] orthography, with very little description and documentation)
%\end{itemize}

Specifically, the project shall result in the following outcomes:%\\
\begin{itemize}
\item a searchable corpus of spoken language for these four endangered Saami languages which shall be linked to multimedia files and intended not only %particularly 
for use in linguistic research (similar to the Giellatekno corpus for North Saami in Tromsø, but with the addition of multimedia), but also as a gateway to the corpus for the language communities;
\item automatic annotation of the corpus (as currently possible for North Saami in the Giellatekno infrastructure);
\item creation of language technology tools for both linguists and revitalization activists such as lemmatizers%is this the english word?
, intelligent dictionaries, translation tools, educational materials, etc. (again based on tools currently used by Giellatekno for North Saami); 
\item development of methods, workflows, conventions and best-practice guidelines for such projects;
\item dissemination of these outcomes to be utilized in the future as at least a template or source of inspiration for other DoBeS, ELAR and other endangered language projects for developing corpus linguistics and language technologies.
\end{itemize}
%Lena: Maybe we should mention that some of our tools and methods can be helpful for other dobes projects (thus Giellatekno has OAHPA for different languages, not only Saami). Other projects could definitely use our results 

Pite Saami and Ume Saami are spoken in Sweden, while Kildin Saami and Ter Saami are spoken in Russia; they belong genealogically to distinct subbranches of the Saamic group inside Uralic: Ume is in the southern branch of West-Saamic, Pite in the central branch of West-Saamic, and Kildin and Ter belong to the Peninsula branch of East-Saamic. All four languages are highly endangered or moribund due to ongoing language shift to the regional suprastrate Indo-European languages Swedish and Russian.

For three of the four languages in question, annotated multimedia corpora already exist thanks to the groundwork carried out by principle applicant Michael Rießler (partially with the help of co-applicants Elena Karvovskaya and Joshua Wilbur) %das in Klammern vielleicht besser als Fußnote, sonst wird dieser Satz zu umständlich?
for the Kola Saami Documentation Project (a DoBeS project) and by co-applicant Joshua Wilbur for the Pite Saami Documentation Project (a Hans Rausing Endangered Languages Project). %All annotations from these projects' corpora can be viewed in alignment with the respective audio/video recordings on a local computer using the program ELAN. Furthermore, the materials in the DoBeS archive can be viewed on-line via the IMDI browser. Elaborate searches can also be performed on annotations locally in ELAN or on-line with the help of the ANNEX/TROVA programs. %%JW: I don't think this is relevant for the Introduction (too detailed)
These corpora covering Kildin, Ter and Pite Saami will form the initial core set of resources for the project's work, but will be supplemented with more data and more elaborate annotations in the course of the project, as well as with addition of the first annotated corpus of Ume language materials.

Community-driven revitalization efforts are underway to varying extents for all four languages. Thus, there is local interest in working on these languages while speakers are still living, and the proposed project will work closely with the language communities in producing practical products in tune with the respective communities' wishes and expectations.

%Ultimately, the project should result in the creating and … The survival of the very object of such research, the language itself, is of course crucially dependent on an active community of speakers. In order to support these individuals and their community as well as to promote active use of the endangered Saamic languages, the project will actively contribute its energies and resources to the language communities …

The project will be carried out within the framework of contemporary documentary linguistics. The team consists of experienced documentary and computational linguists as well as programmers familiar with Saamic languages in general and specifically with excellent working knowledge of the Saami languages in question and the relevant \textit{lingua franca}. Team members are also well versed in corpus- and computer linguistics, carrying out documentation fieldwork, and in practical revitalization issues. The project will also profit from the symbioses between researchers familiar with the Giellatekno infrastructure (including tools for other northern languages) and researchers acquainted with the DOBES and ELAR infrastructures (documentary linguistics of spoken language, archiving of annotated multimedia).

%The proposed project will have the support of the {Giellatakno-Projekt} (Tromsø, http://giellatekno.uit.no/), which already is a cooperation parter for M.\,Rießler and J.\,Wilbur in the documentation of Pite and Kola Saami languages. %%JW: not sure this is really relevant here in the Intro, should be clear from previous paragraph

\section{Detailed project description}%approx 15 pages
\subsection{Introduction}

The \textit{Dokumentation bedrohter Sprachen} (DoBeS) and the \textit{Hans Rausing Endangered Language Project} (which ELAR is a part of) programs have the aim to create data-orientated, multifunctional, and generally accessible documentations of endangered languages in order to support languages in danger of extinction and to help protect endangered linguistic and other cultural data from disappearing. In this, archiving recordings of spoken languages alone is not considered sufficient; instead, archived recordings must include metadata and annotations describing their contents. This should at least consist of cataloging metadata (concerning eg. participants, recording location, etc.) and transcriptions of the recordings with translations into a stable \textit{lingua franca}. However, annotations can provide much more detailed information, such as the specifics provided by detailed linguistic analyses\footnote{Detailed linguistic analyses typically contain such information as part of speech, glosses, and/or word or morpheme boundaries.}. Such analyses are interesting and useful in and of themselves for linguistics research, but they also can be utilized to produce lexica, translation tools and teaching materials for use by linguists and, at least as important, by the relevant endangered language community. However, producing such detailed analyses is exceptionally time and resource consuming, and as a result, only a portion %can we be more specific here?
of archived materials typically include such detailed information. Fortunately, with the help of current computer-based technology, the availability of such detailed linguistic analyses in a computer-readable format can result in:
\begin{itemize}
\item easier and more effective searches of documentation corpora for both corpus linguists and language community members, including linking search results to the actual primary recordings;
\item the creation of lexica, translation tools and educational materials utilizing modern web-based technologies.
\end{itemize}
It is thus the logical next step from the perspective of both language and research communities to supplement archived materials with as much detailed linguistic analyses as possible and then, with the help of computers, to create practical products as a result. This task is particularly urgent for endangered languages because 1) access to native speaker knowledge is limited and potentially no longer possible in a relatively short time, and 2) because those involved in revitalization efforts want to take advantage of the practical tools that result in counteracting wide-spread language death at a local level as soon as these become available.

It is precisely these considerations which are at the source of motivation for the proposed project.

\paragraph{what is “corpus linguistics”?} Corpus linguistics=study of language as expressed in "real world" text
%korpuslinguist ist jeder der mit hilfe von korpora etwas erforscht. aber unser hauptziel soll es sein eine _korpuslinguistische infrastruktur_ zu schaffen, damit man später aus DOBES mehr machen kann

\paragraph{what is computer linguistics?} Computer linguistics=maschinelle Verarbeitung natürlicher Sprache
%-Entwicklung von Analyse- und Generierungsverfahren für natürlich-sprachliche Texte
%-Programme zur Sammlung und statistischen Auswertung großer Mengen von Sprachdaten (Lemmatisierung, Häufigkeitswortlisten, Konkordanzen)
%-Praktische Anwendungen: maschinelle Übersetzung, computergestützter Sprachunterricht
%-Giellatekno already has all this for Northern Saami, partly also for Lule and South Saami, very little even for Kildin Saami

\paragraph{what are corpora?}
%DOBES enthält Korpora, in denen man aber inhaltsmäßig nur relativ beschränkt suchen kann, wegen inkonsequenter (manueller) Annotation
%Giellatekno corpus largely derived by automated processes (morphological+syntactic automators)

\paragraph{what are multimedia vs. written corpora?}
%DOBES/ELAR corpora largely derived by hand (ELAN, Toolbox)

\paragraph{what is language technology?}
%TROND??
% hmm, die unterschied zwischen sprachtechnologie und computerlinguistik ist vielleicht das das erste funktioniert, das zweite nicht?

% Was ich hier schreibe ist polemisch, aber wahr. Vielleiht sollen wir nicht so polemisch sein. Ob nicht, nehmen wir ganz einfach alle Referenzen zu konkurrierenden Paradigmen weg.
Language technology is technology aimed at analysing and generating natural language in various ways. Currently, the dominating paradigm within language technology is based upon statistical methods, upon teaching the computer the behaviour of natural language by means of presenting it for vast amount of unanalysed and manually analysed data. This approach achieves results superior to the most common gramamr-based approaches. For the majority of the world's languages, and especially for the languages treated here, this is not a possibility. The amounts of text needed (analysed or not) is simply not available. 

The competing paradigm is grammar-based. 
%Lena: should we mention somewhere the differencies between documentation and  a corpus? Thus what DOBES now has is not a corpus. However this data could be organized and made search-able (this makes a corpus). However some work is required to do it. The methodology is extremely important. Not every corpus is search-able, even if it is called "corpus". We have a good example - giellatekno, we can use the methodology which already exists. 

\paragraph{why Saami?}
%-because computer linguistic infrastructure already available for closely related (linguistically and culturally) written Northern Saami
%-because computer linguistic know-how already available for simple and parallel written corpora of Northern Saami

\paragraph{what Saami?}
%Pite Saami: annotated corpus available at ELAR, linguist (for programming morphological (and syntactic) automators) available
%Kildin, Ter Saami: annotated corpus available at DOBES, linguists (for programming morphological (and syntactic) automators) available
%Ume Saami: the most endangered Saamic language (probably only one living L1 speaker); almost no documentation; revitalization activity by speaker community 

\subsection{Urgency of documentation}
The Saamic languages (Uralic) form a dialect continuum across an area extending from central Scandinavia to the Kola Peninsula in the Russian Federation. Traditionally, the Saami were half-nomadic peoples who migrated annually between summer and winter settlements and survived on hunting and fishing. Herding reindeer was not a typical occupation until relatively recent times, but has become so at least in the central mountainous areas of Sápmi %das stimmte so nicht, was Josh geschrieben hatte: Rentierzucht gibt es erst seit dem ausgehenden Mittelalter und ist auch nur für einen Teil Saami typisch; vor allem waren sie bis vor kurzem halbsesshafte Fischer, Jäger und hatten oft sogar etwas Hausvieh
; nowadays most Saami live modern lives and blend in with the rest of the predominantly non-Saami population. Due to the influence of North Germanic, Finnic and Russian languages and cultures, all ethnic Saami are fluent in their respective contact languages, while younger generations are typically monolingual in these. As a result, the Saami languages are today either extinct, moribund or endangered. The least endangered Saamic language is Northern Saami with approx. ?? speakers in Norway, Sweden and Finland out of ?? ethnic Saami.%MR:I'll check the figures
The proposed project will focus on four particularly endangered Saami languages: Ume, Pite, Kildin and Ter Saami.

Ume is a southern Saamic language of West-Saamic and spoken in central Swedish Lapland in and around the modern towns of Arvidsjaur, Malå, Ammarnäs and Tärna. Today, Ume is practically extinct, as the estimated number of speakers is less than five.\footnote{This is J.\,Wilbur's own estimate, based partly on hear-say; Henrik Barruk (ca.~40 years old) taught himself Ume as a second language, but speaks it with his two kids, who could be considered Ume speakers; according to my contacts in Arjeplog, there are 2–3 elderly non-active speakers in Arvidsjaur.} Although the first Saami language book ever was a bible translation (New Testament) in Ume Saami from 1755, no standardized orthography has been developed, nor is there any significant set of literature in Ume. However, one Ume language activist (Henrik Barruk) has recently developed some educational materials and texts on his own.

The situation for Pite Saami is not quite as desperate, but can hardly be described as positive. Pite (also known as Arjeplog Saami) is in the central branch of West-Saami and is currently spoken by approximately 30 speakers in and around the community of Arjeplog in central Swedish Lapland. Of these speakers, the vast majority are 50 years old or older and have neglected to teach their children Pite. There is one exception, a 33 year old reindeer herder who actively speaks Pite on a daily basis in his work with his family's reindeer, and speaks Pite with his two young sons (6 and 4 years old). As with Ume, Pite does not have an established orthography; however the local Saami association in Arjeplog is currently completing a project called \textit{Insamling av pitesamiska ord}\footnote{Expected completion in December 2010.} to collect a wordlist consisting of several thousand words, and is creating an orthography in the process. Co-applicant J.\,Wilbur has worked closely with the project as an unofficial “linguistics consultant” and in many conversations with Nils-Henrik Bengtsson, the project coordinator, has heard about the project's interest in using the wordlist to create a dictionary and pedagogical materials.

Historically, Ume and Pite were also spoken in the adjacent parts of central Norway, but today, the territory in which they are spoken has been reduced to a small area in Swedish Lapland due to repressive language policies in both Norway and Sweden, particularly during the first half of the 20th century. Today, Swedish language and culture dominate the lives of the Ume and Pite Saami to such an extent that their languages have practically been wiped off the map due to language shift to Swedish. Indeed, \citet[123]{blokland-etal2003} point out that Swedish laws protecting the status of Saami minorities lag behind those in Norway and Finland; this has contributed to a faster decline of Saami languages and culture in Sweden. Current Swedish (and Norwegian) language legislation only applies to South, Lule and North Saami, and ignores Ume and Pite as separate languages (Kulonen et al, 2005: 180), despite the fact that they are considered independent languages by linguists \citep[cf.][]{gordon 2005,sammallahti1998b}. Even a recent language law introduced in Sweden in 2009 and intended to protect and support minority languages only recognizes Saami as one single language, despite the significant linguistic and social divides between the various Saami groups. Effectively, the other three Saami languages spoken in Sweden (North, Lule and South Saami), but particularly North Saami (with the most robust group of speakers in Sweden, Norway and Finland, as well as with radio and television support) block any chances that Ume or Pite might have for gaining public recognition. 

While knowledge of Pite lexical items is common particularly among reindeer herders, the language is rarely used as a means of communication. No classes are taught in school, not even as a foreign language, however there have been sporadic intensive courses over the last two decades, targeted mostly at teaching reading and writing using the Lule Saami orthography.

Ter and Kildin Saami spoken in Russia together form the Peninsula branch of East-Saamic. Ter Saami is nearly extinct and spoken by no more than 30 speakers or semi- speakers living in different locations within and outside of the Murmansk region, such as in Murmansk, Lovozero, Gremicha, Revda, Krasnoščel'e, Umba, and also in Saint Petersburg. The mean age of the youngest speakers of Ter is over 50.


%Lena: the urgency of work with the collected data should not be underestimated. Even though we already have a rich documentation, the languages are still endangered. Together with the the language community we can still make a lot with the texts and interviews which we have recorded. We have the community members who are interested and there are still speakers left. However if the speakers are gone, the work on the recordings becomes extremely difficult. It is problematic with the DoBes archives. They should be analyzed as soon as possible, but most of the data is only annotated. 

In addition to the fact that Ume and Pite are in desperate need of documentation (revitalization seems unlikely at present despite an increase in interest among younger ethnic Saami), from a purely linguistic point of view, they prove to be interesting both cross-linguistically and within Saami studies. Unlike the other Indo-European languages in the area, Saami languages are almost exclusively suffixing and mostly have postpositions. Phrases are generally head-final, although SVO has become grammatical as well, just as postponed relative constructions are acceptable (most likely a result of language contact). Saami case systems consist of between seven and nine cases. Verbs and pronouns are inflected for singular, dual and plural number categories. In Pite and Ume, all inflectional case suffixes have been retained, and accusative and genitive cases are still differentiated morphologically; they are thus considered more archaic than other Saami languages to the north and east that have lost some case endings and have come to lack an accusative-genitive distinction. Phonological and phonetic studies of Pite and Ume from the project's data should reveal more detailed information about another Saami characteristic: preaspiration (a rarity in the world's languages).

Additionally, Saami languages are known for their complicated non-linear morphology realized as umlaut, consonant gradation (phonological alternations in a stem's consonants triggered by the morphological environment) and/or palatalization of noun and verb stems. In general, regressive vowel harmony in Saami languages is found in the initial stressed syllable of a stem with its allophone being adjusted to the vowel in the second and unstressed syllable. Pite is especially interesting in this respect because it has progressive vowel harmony concerning lip rounding in the same environment (Korhonen 2005a:272). Ume is interesting because it has a large variety of umlauts and the least amount of consonant gradation, with the exception of neighboring South Saami, the only Saami language lacking consonant gradation. Whether proto-Saami also had consonant gradation is a controversial issue (cf. Svonni 2006:154-55), and further information about this phenomenon in Ume and Pite could help inform this debate. Furthermore, Ume is unique among Saami languages in having trisyllabic stems in the nominative case (Korhonen 2005b:421); the role nonlinear morphology plays in words of this length is unclear. The resulting corpus from the proposed project could help Saami linguistics to come to a more detailed understanding of the complicated nonlinear morphological processes in Ume and Pite, these processes' history and their place in Saami languages and linguistics in general.%%is this paragraph necessary? ja, sogar noch mehr: fast nichts bis jetzt gemacht zu Informationsstruktur, Fokus, und und und
%corpus linguistic ... blablabla will make more date for future research on these and other topics available to Saami and General linguists

\subsection{State of research}
\subsubsection{Language documentation}
grammars, dictionaries, text collections

%Slightly more linguistic work has been done on Pite than on Ume. Juhani Lehtiranta published a Pite grammar in Finnish in 1992, but in fact this grammar is based on earlier studies done between 1890 and 1950, and thus does not necessarily reflect current usage. There is also an old Pite grammar in German from 1926 written by Eliel Lagercrantz. There is no dictionary of Pite. For Ume, on the other hand, an Ume-German dictionary based on the speech of the village Malå (a dialect which is now extinct) was published by Wolfgang Schlachter in German in 1958. In 1738 a dictionary and the only grammar of Ume appeared (Fjellström 1738a, 1738b). Aside from discussions of Ume and Pite in various articles, that was the extent of linguistic work on these languages. However, the Pite Saami Documentation Project has now archived annotated materials at the Endangered Languages Archive in London; various papers and talks are being worked on, and a PhD-dissertation including a sketch grammar and a description of morpho-phonological phenomena shall also be completed by the end of 2012 by co-applicant Joshua Wilbur. There is a considerably larger amount of material available for Lule and South Saami (the two neighboring languages to the north and south) which may prove useful in analyzing the data.

Slightly more linguistic work has been done on Pite than on Ume. Juhani Lehtiranta published a Pite grammar in Finnish in 1992, but in fact this grammar is based on earlier studies done between 1890 and 1950, and thus does not necessarily reflect current usage. There is also an old Pite grammar in German from 1926 written by Eliel Lagercrantz. There is no dictionary of Pite. For Ume, on the other hand, an Ume-German dictionary based on the speech of the village Malå (a dialect which is now extinct) was published by Wolfgang Schlachter in German in 1958. In 1738 a dictionary and the only grammar of Ume appeared (Fjellström 1738a, 1738b). Aside from discussions of Ume and Pite in various articles, that was the extent of linguistic work on these languages. However, the Pite Saami Documentation Project has now begun archiving annotated materials at the Endangered Languages Archive in London; %muss/soll ich genauer erwähnen, wieviel schon archiviert ist, noch archiviert wird? schwer zu quantifizieren.
various papers and talks are underway, %eine sehr vage Angabe!
and a PhD-dissertation including a sketch grammar and a description of morpho-phonological phenomena shall also be completed by the end of 2012 by co-applicant Joshua Wilbur. There is a considerably larger amount of material available for Lule and South Saami (the two neighboring languages to the north and south) which may prove useful in analyzing the data.

\subsubsection{The DoBeS archive for Kola Saami}
JOSH/MICHA

\paragraph{Corpus search tools at DoBeS}
\begin{itemize}
\item Annex/Trova: "the Annotation Exploration tool in the MPI web-based framework for archive exploration and enrichement" - see IMDI-Browser!!
%Lena: The complex searchh in the archive is not really possible, because different conventions for annotation are being used. Most of the data is only translated and has no grammatical analysis - JOSH: excellent point! this should also be highlighted in the application, and that we will adhere to eg. Leipzig glossing rules and other accepted standards/conventions, and will definitely make very transparent glossing practices of our own and ensure all project transcriptions are glossed consistently.
\item Arbil: "IMDI Metadata Editor, Browser \& Organizer Tool" only offline, thus only for locally-stored data; search function not yet integrated
\item IMDI-Browser: search entire MPI corpus collection (DoBeS + many more):
	* in "metadata" fields (in the DoBeS sense of cataloging metadata)
	* in "annotation" fields, including display of all hits within context and links to web-based display of ELAN file in ANNEX, inludes list of 8 words on either side of hit, very complex context search using web-based TROVA annotation search; search results can be downloaded/saved
\item ELAN: elaborate searching, including using regular expressions, within a single ELAN file or across several files, including multiple layer search (same software/GUI as with ANNEX/TROVA); search results can be saved.	
\end{itemize}

%KWIC-Suche (key word(s) in context)
%ich glaube diese kexwords kann man über imdi eingeben
%KWIC ist aber für  linguistische und anthropologische Forschung (sowie für Revitalizierung) nur sehr beschränkt nützlich

\subsubsection{The ELAR deposit for Pite Saami}
JOSH

In August 2010, the first Pite materials resulting from PSDP went online at the Endangered Languages Archive in London. This initial deposit consists of 6 audio/video recordings and the respective ELAN annotation files, as well as related images and metadata definition files. These recordings cover a variety of genres (from descriptions, traditional handicrafts, reindeer herding and vocabulary, conversations), and in the future more recordings and transcriptions/annotations will be deposited and made available online, including elicitation sessions.

\paragraph{Corpus search tools at ELAR} 
\begin{itemize}
\item very basic character-string search (not even as advanced as a google search!) of a single project's deposit only. I assume more will be possible in the future?
\item ELAN annotations can be searched (off-line) using the DOBES tools
\end{itemize}

\subsubsection{The Saami language technology project}
% TROND … write specifically, what is preliminary already there or in the works for Pite and Ume and Kildin and Ter

The language technology project Giellatekno already has an infrastructure setup for building language analysers. This infrastructure has been set up for Pite and Kildin Saami, and for both languages there is an embryonic analyser, containing some items from each part of speech, and some of the basic morphological and morphophonological processes for nouns and verbs. The Kildin Saami analyser has also been included on the Giellatekno home page, cf. \url{http://giellatekno.uit.no/cgi/index.sjd.eng.html}.  Nothing has been done for Ume or Ter Saami yet.

\subsubsection{Corpus linguistics projects at ELAR, DOBES and ??other}
%Lena: does anybody know what these guyes are doing? http://www.rosettaproject.org/ The Rosetta Project:A meaningful survey and near permanent archive of 1000 languages. JOSH: as far as I can tell, they only store printed/image data so far, so they could in theory have our transcriptions/annotations, but we are also very concerned about linking these to the actual recordings; but the Rosetta Project has serious foresight!
%and http://www.endangeredlanguagefund.org/ Endangered Languages Fund: Devoted to the scientific study of endangered languages and to the support of native efforts in maintaining endangered languages. Helping to stem the tide of language dissapearance in the world. JOSH: as far as I can tell, they fund similar projects like HRELP and Dobes, but the archive is purely symbolic and for any serious comparative or corpora linguistic work essentially useless, at least the online access is, and I can't find any information about future plans for the archive - maybe I missed something, but maybe they don't plan to make archived materials available on-line.

\subsection{Research aims and methodologies}

\subsubsection{Giellatekno} 
%HERE WE SHOULD MENTION GIELLATEKNO'S GENERAL RESEARCH AIMS AND METHODOLOGIES

% Our approach (finite-state transducers and constraint grammar) differs from most grammar-based systems in that it gives robust analyses for unconstrained text input.

With the basic grammatical analysers and generators in place, a wide range of possibilities opens up for language anlysis and development. Lingusitic analysis will benefit from lemmatisation and grammatical analysis: From a given corpus of collected texts, analysers may keep track of which lemmata are covered in a dictionary and which are not. For revitalisation work, 
% hmm, repeating myself here

… A FEW MORE SPECIFIC THINGS, LIKE:

\paragraph{Two-Level-Morphology}
Zur computerlinguistischen Verarbeitung der sprachlichen Daten wird vorerst die s.g. Zwei-Ebenen-Morphologie (TWOL) angewendet. In TWOL wird die Zuordnung von Morphemen auf zwei Ebenen beschrieben:
\begin{enumerate}
\item morphologische/lexikalische Ebene: Struktur eines Lexems, Morphe, Morphemgrenzen
\item morphographischen Ebene: Oberflächenrealisierung des Lexems
\end{enumerate}
Die erlaubten bzw. verbotenen Zuordnungen von Morphemen auf beiden Ebenen erfolgt mittels (sprachspezifischer) Regeln. Jede Regel wird von einem Transducer in die Realisierung auf der jeweils anderen morphologischen Ebene übersetzt. Somit können morphologische Formen nicht nur analysiert (Übersetzungsrichtung morphographemische Ebene $\rightarrow$ morphologische/lexikalische Ebene), sondern auch generiert werden (Übersetzungsrichtung morphologische/lexikalische Ebene (Stamm+Affixe) $\rightarrow$ morphographemische Ebene (Oberflächenrealisierung)).

Der TWOL-Ansatz eignet sich besonders zur Beschreibung von Sprachen mit einem komplexen und sich durch Irregularitäten und Stammallomorphie auszeichnenden Formensystem – wie z.B. Saamisch. 

\paragraph{Zwei-Ebenen-Morphologie} (two-level morphology, TWOL) ist ein theoretischer Ansatz, in dem die Struktur eines Lexems auf der morphologisch-lexikalischen Ebene in Relation zu dessen Oberflächenrealisierung, d.h. der Kodierung des Lexems auf der morphographischen Ebene, gesetzt wird. Erlaubte bzw. verbotene Korrespondenzen zwischen Zeichen auf den zwei Ebenen werden durch abstrakte Regeln erfasst.

Um Übergeneralisierung bei der durch einen Automaten (Transduktor) angefertigten Übersetzung zu vermeiden, müssen die Regeln durch die Anbindung eines (Morphem-)Lexikons sowie durch Angaben über Morphemkombinatorik auf der morpho-lexikalischen Ebene eingeschränkt werden.

Das Lexikon stellt eine Liste aller gebundenen und ungebundenen Morpheme dar. Ein vorläufiges Morpheminventar wird aus den bereits vorhandenen Toolboxprojekten von KSDP und PSDP importiert. Hauptziel nach abgeschlossener Arbeit mit Toolbox ist es aber weitere Texte mit Hilfe der Automatoren zu parsen und das Morpheminventar damit automatisch zu erweitern.

Die Listen mit Stämmen und der ihnen zugeordneten morphologischen Markierungen werden in TWOL mittels eines mit Indizes versehener Listenverkettungsmodells (indexed concatenation model) miteinander verbunden. Dies bedeutet, dass phonologische oder graphemische Morphemvarianten eines Stamms mit unterschiedlichen Indizes versehen werden, die der Art der zugelassenen Folgekategorien (d.h. möglichen Suffixen für einen gegebenen Stamm) entsprechen. So erhält ein Stamm, dem ein bestimmte Gruppe Suffixe folgen kann, eine andere Indizierung als ein Stamm, dem eine andere Gruppe Suffixe folgen kann. Auch die Suffixe im Lexikon erhalten Indizes, die angeben, mit welcher Art indizierter Stämmen sie kombiniert werden dürfen. Die Aufgabe des Listenverkettungsmodells ist es, die Liste der Wortstämme mit der möglicher Suffixe entsprechend der Indizes einzelner Elemente in diesen Listen miteinander zu verketten.

\textit{Giellatekno} wendet TWOL bei der Erstellung von sprachtechnologischen Anwendungen wie interaktive pädagogische Programme oder Rechtschreibprüfprogramme für morphologisch komplexe Sprachen mit Stamm- und Flexionsvariation an, darunter v.\,a. Nordsaamisch, aber auch andere saamische Sprachen sowie auch Grönländisch, Kveeni, Komi

\paragraph{Syntaktische disambiguierung}

\paragraph{automatic translation} 

\paragraph{the automatic-dictionary-thing} Mit Hilfe des Lexikons und der zu programmierenden morphologischen und später auch auch syntaktischen automaten können am Ende neu eingegebene bzw. importierte Texte automatisch analysiert sowie Paradigmen einzelner Lexeme generieren werden.

\paragraph{Wortformgenerator} Die bereits im Laufe des Pilotprojektes zu erstellenden \textbf{Paradigmen-} und \textbf{Wortformengeneratoren} können eine praktische Funktion im Sprachunterricht übernehmen. In Korpora lassen sich i.d.R. nicht alle Formen eines Paradigmas finden. Mithilfe von TWOL (aber im Unterschied zu Programmen wie Toolbox) können Wortformen jedoch nicht nur analysiert, sondern auch generiert werden. Komplette Paradigmen lassen sich somit für jeden Eintrag in der Datenbank in Echtzeit generieren oder z.B. in das geplante interaktive Lexikon einbinden.

Dementsprechend gibt der Wortformengenerator bei vorgegebener Grundform eines Lexems und der gewünschten morphologischen Markierungen die entsprechende grammatische Form aus (z.B. input: hestur+ACC $\rightarrow$ output: hest).%besser saamisches Beispiel\\

Sobald in einer späteren Projektphase auch ein syntaktischer Automat programmiert worden ist, könnte zusätzlich sogar ein \textbf{Interaktives Syntax-Lernprogramm} erstellt werden. Dabei liefert zuerst die morphologische Analyse auf Wortebene die jeweilige Form. Ein syntaktischer Disambiguator eliminiert daraufhin auf Satzebene die inkorrekten Analysen und versieht die korrekten Analysen mit entsprechenden syntaktischen Funktionen (Subjekt, Objekt usw.). Schließlich könnte eine Phrasenstrukturgrammatik der linearen Darstellung eine hierarchische Struktur zuordnen.

\paragraph{spell checker} use Divvun experience with Northern Saami

\paragraph{hyphenation} use Divvun experience with Northern Saami

\paragraph{Oahpa!}

\subsubsection{DoBeS}

\paragraph{LEXUS} as interface for multimedia annotation (done by assistants) and web-based presentation, export to off-line dictionary applications, export to wikipedia

\paragraph{ELAN}

\paragraph{the dobes-corpus-search-thing}

\subsubsection{ELAR}

\subsubsection{methodological problems}

Wir würden gern Giellateknos Methoden und Worksflows verwenden. Aber unser geplantes Korpus unterscheidet sich von Giellatekno für Nordsaamisch:

-gesprochene Sprache (hohe Variation durch Dialekte und durch language attrition/language loss; sehr viel code-switching; sehr viel hezitations, false starts, self-corrections usw.; Intonationseinheiten anstatt Sätze)

-keine etablierte Orthographie (für Pite und Ter)

-kleinerer Umfang des Korpus (Anzahl der Token)

-komplett mehrsprachiger (paralleler) Korpus: Pite-Schwedisch-Englisch bzw. Kildin/Ter-Russisch-Englisch (weil alle unsere Transkriptionen übersetzt sind)

-Link zu Multimedia ist obligatorisch!


\subsubsection{written vs. spoken coprora} 
CORPUS OF WRITTEN LANGUAGE

we have many and large corpora of written languages

-CORPUS OF SPOKEN LANGUAGE

--we have a lot fewer and smaller corpora of spoken language (e.g. dialects, sociolects)

--written representation of text

--different units than in text

--linguistic features characteristic of spoken language (intonation, hesitation, false start, etc.)

---CORPUS OF AN ENDANGERED SPOKEN LANGUAGE

---we have very few and only very small corpora of endangered languages

---written representation of text

---different units than in text

---linguistic features characteristic of spoken language (intonation, hesitation, false start, etc.)

---code switching to other languages

---language attrition

\subsubsection{The “spoken corpus” issue}
…this is mostly an issue regarding annotation conventions

\subsubsection{The “orthography issue”}
… since we are working with the orthographic representation of our texts and we are producing practical tools for language users … we have to discuss in some detail for what languages orthographies are already established (Kildin is established but in several variants; Pite is under development by community members in collaboration with Josh; Ter could use the Kildin orthography but community members have to decide on this; Ume is unclear) … and how we deal with orthography issues like variants and possible later changes originating from inside the communities … I know that this is all no big deal from the Giellatekno perspective, but we have to address this for our referees 

\subsubsection{Different users' portals to our created infrastructure and tools}

\paragraph{Giellatekno} portal in multilingual (Norwegian/Swedish, Russian, English, other) localizations is already existent

\paragraph{Project portal} \url{www.saami.uni-freiburg.de} community friendly multilingual portal to ELAR and DOBES archives, to PSDP, KSDP, Tools, etc. 

\paragraph{ELAR} only links

\paragraph{DOBES} only links

\subsubsection{Work flow}
svn/ichat/subethaedit/etc.

Workflow
-manually annotated texts linked to audio and video (ELAN) already available
-create annotation in standard orthography (+using conventions for features characteristic of spoken language)
-create automators
-parse
-create automatically annotated texts linked to audio and perhaps video (ELAN)

rough plan:
1. Kildin Morpho-Automaten schreiben (alle zusammen)
2. Kildin Syntax-Automaten anfangen zu schreiben (alle zusammen)
3. Pite Morpho-Automaten anfangen zu schreiben (alle zusammen)
4. Kildin Syntax weitermachen (Micha+GT)
5. Pite Morpho weitermachen (Josh+GT)
6. Pite Syntax anfangen (alle)
7. Ter morpho-automaten
8. Ume morpho-automaten

%!!Note DOBES requirements
%-we have to use their tools (ELAN, LEXUS, etc.)
%--[ELAN and LEXUS is used already]
%-we have to use their archive (Nijmegen)
%--[no problem] 
%-we have to involve to speech community
%--[hire and train native speaker assistant for work with LEXUS (e.g. create a monolingual multimedia dictionary after automatic lemmatization]

\subsubsection{Expected outcome of the proposed project}
Automatically annotated corpora for spoken Pite Saami, Kildin Saami, Skolt Saami, Ter Saami languages.

practical products:

-tagged (single and parallel [always 2 translations already available at ELAR/DOBES!]) corpora of Saami languages =useful/neded for linguists

dictionaries =useful/needed for linguists and revitalization:

-gt-Automatic dictionary for Kildin, Pite, Ter, Ume
%note that an preliminary (not yet established) orthography is used in the case of Pite and Ter and Ume

-LEXUS-multimedia lexicon =useful/needed for revitalization

-teaching programs (Oahpa!)=useful/needed for revitalization

Spell checker and hyphenation program for Kildin (not the other languages, because no established orthographies yet)

Automatic translation for Kildin, Pite (not Pite and Ume, because too small corpora)
%note that an preliminary (not yet established) orthography is used in the case of Pite and Ume  
%Lena: using the spell-checkers we  could also annotate the existing Saami texts which have been published earlier. Like Образцы саамской речи. We could use an OCR (Optical character recognition) programm like Fine Reader. Including older texts in the corpora will be very helpful for future researchers and for the community; JOSH: in theory, definitely a good idea, but practically the question is whether it is realistic to include legacy - I originally planned to work on legacy materials for Ume in my original HRELP application, but I have my hands full with the recordings I've done on my own (of course, I'm also not the most industrious/productive linguist alive) LENA: you are right. but I thought  if you really create the automates you can add legacy materials later.
…

…

…

Finally, creation of workflows, tools, conventions which can be useful for other DOBES projects as well


\subsection{Cooperation partners}
Kola Saami: Lovozero

??Pite/Ume Saami: Peter Steggo%still waiting for his response! maybe include Nisse Bengtsson, too?

??DAUM: Umesaamische Aufnahmen%still waiting for response from arkivchefen

??Divvun: spell checker, hyphenation

??SaamiDocNet%has anyone contacted Bruce? or who do we need to contact?

\bibliographystyle{linquiry2.bst}
\bibliography{KolaPiteDobes}

\subsection{Community consent}

\section{Binding indication}

\section{Key project participants, their responsibilities in the project and previous work on the subject}

\subsection{Project participants' previous work on the subject}
The team consists of experienced documentary and computer linguists and programmers well familiar with Saamic languages.

\paragraph{Michael Rießler} completed a \textit{Magister} degree in Scandinavian Linguistics at Humboldt-Universität in Berlin. His research has focused on both historical and synchronic-typological issues, and has covered a variety of topics including (but not limited to) areal linguistics in Northern Europe, Saami linguistics, and Documentary linguistics. He submitted his doctoral dissertation in General Linguistics in July 2010, and his defense at the Universität Leipzig is expected in November 2010.

During the last six years, Michael Rießler has conducted extensive fieldwork on Kola Saami languages. Due to his work as principle investigator and coordinator of the “Kola Saami Documentation Project”, a DoBeS project, he is familiar with the objectives and methods of contemporary documentary linguistics in general and with the framework of the DoBeS program group in particular. He has also organized a DoBeS Winter School on “Saami Language Documentation and Revitalization” (together with Ulrike Mosel, Jurij Kusmenko, Bruce Morén, and Ida Toivonen) which took place in 2010 in Bodø/Norway.

In connection with the documentation of Kola Saami M.\,Rießler has extensively collaborated with the Saami language technology group (Giellatekno) at Tromsø university and has helped creating Web-based teaching tools for Kildin and Skolt Saami. He is thus well familiar with practical computer linguistics applications and the methods and workflows applied at Giellatekno.

\paragraph{Trond Trosterud} … Saami linguist, Computer linguist, leader of Giellatekno, employed in Tromsø

\paragraph{Ciprian Gerstenberger} … Computer linguist, programmer, employed in Tromsø

\paragraph{Joshua Wilbur} completed a \textit{Magister} degree in General Linguistics and American Studies at the Universität Leipzig in March 2008. Since April 2008 he has been a doctoral student in the General Linguistics department at Humboldt-Universtität in Berlin, but will be transferring to Christian-Albrechts-Universtität in Kiel to complete his PhD in Documentary Linguistics under the supervision of Ulrike Mosel. His initial contact with endangered languages took place as a student assistant in the Chintang and Puma Documentation Project (DoBeS), as well as during an internship fieldwork trip to collect an initial wordlist for the Gurung dialect (Tibeto-Burman) spoken in the Manang District of Nepal. Joshua's first contact with Saami languages came as a student research assistant in the Kola Saami Documentation Project (DoBeS) and included two field trips to the Russian Federation to work on Kildin and Ter Saami; as a result of these trips and his work on the project, he completed his MA thesis on syllable structures and stress patterns in Kildin Saami. He has also written an article (together with M.\,Rießler) on “Documenting the endangered {K}ola {S}aami languages” \citep{riesler-etal2007}.

Since June 2008, he has been coordinator of the Pite Saami Documentation Project (PSDP; funded by the Hans Rausing Endangered Languages Project) and recipient of an Individual Graduate Studentship to carry out the project as part of the HRELP funding (extended funding through August 2011). For the project, he has spent 14 months doing fieldwork in and around Arjeplog to collect data for the project. His PhD dissertation will involve the documentation corpus being collected, a sketch grammar of Pite Saami (the first grammatical description in English) and a detailed analysis of morpho-phonological phenomena in Pite Saami (expected completion in summer 2012).

As part of PSDP and his own PhD project, he has compiled an extensive database for his recordings and results from elicitation sessions, and has developed Kildin and Pite keyboards and maintains a webpage on the project website concerning useful project resources for documentary linguists. He has taught a university class on documentary linguistics in the northern European area, and led a fieldwork course and a Toolbox tutorial at the DoBeS Winter School on “Saami Language Documentation and Revitalization”.

\paragraph{Elena Karvovskaya} is a student of General linguistics at Potsdam university. In 2010 she finished her B.Sc. thesis in Kola Saami semantics which was based on fieldwork and corpus research. She has also co-authored conference presentations and a paper on Kola Saami semantics.

Beside her extensive field work on Kildin and Ter Saami (with M.\,Rießler) and Iźva-Komi (with R.\,Blokland and M.\,Rießler) on the Kola Peninsula she has done field work on Pite Saami (with J.\,Wilbur) and earlier also participated at several other field expeditions in the Caucasus (with Yakov Testelets) and Siberia (both with Olga Kazakevich). As the result of her internships and student assistantship for the Kola Saami Documentation Project and the Languages of West Ambrym, her participation at DoBeS training courses and a DoBeS Winter School in Saami Language Documenation she is familiar with the Documentary linguistics framework in general and with the DoBeS workflows in particular.

Due to her studies and her research E.\,Karvovskaya has acquainted herself with general corpus linguistic methods and tools. She has also participated at a computer lexicography workshop organized by Giellatekno in Tromsø and has been working with the creation of Web-based teaching tools for Kildin and Skolt Saami. He is thus also well familiar with practical computer linguistics applications and the methods and workflows applied at Giellatekno.

\subsection{Project participants' responsibilities}

\paragraph{Trond Trosterud} will coordinate the project work including the setting up and maintaining the project's computer linguistics infrastructure at Giellatekno in Tromsø. He will also work (together with other team members) on ... Saami data and tools%, including the training of native speaker assistants …

\paragraph{Michael Rießler} will administer the project and working as principal researcher with special focus on the Kildin and Ter Saami data and tools.%noch etwas mehr muss hier hin, z.B. LEXUS, DOBES archive infrastructure oder sowas

\paragraph{Joshua Wilbur} will work as second principal researcher with a half-time position. His contribution will focus on the Ume and Pite Saami data and tools. Beside his collaboration in the project he will complete the Pite Saami multimedia archive at ELAR and his doctoral dissertation (planed for 2012).

\paragraph{Ciprian Gerstenberger} will work as computer linguist and programmer for the 

\paragraph{Elena Karvovskaya} will be employed as a \textbf{student research assistant} with 10 hours of work per week for 3 years … M.A. thesis in Saami corpus linguistics.\\

For assistance with … the project will hire … members of the Saamic communities in Sweden and Russia as \textbf{research assistants}. In this way the project hopes to encourage and to stimulate the passing on of linguistic knowledge and language within the Saamic community …

\subsection{Infrastructure}
All team members are based at different places in Freiburg (Rießler, Wilbur), Tromsø (Trosterud, Gerstenberger), Sweden (Pite or Ume assistant), Russia (Kola Saami assistant), Potsdam (Karvovskaya) … svn/iChat/Subethaedit/etc. as it is common practice at Giellatekno …

\section{Work program and time schedule}
\subsection{General work program}

SaamiDocNet meeting Dec Murmansk bevor Projekt eigentlich losgeht und vollständig durch SaamiDocNet finanziert
Language technology for East Saami
Presentation of KSDK archive
Giellatekno training
LEXUS training
all team members will meet in Murmansk

it is planned to copy the Murmansk meeting (if successful) and organize a similar event in spring 2011 in Arvidsjaur or Arjeplog for the SW Saami languages South, Ume, Pite
this would than also be the second project meeting (also completely financed by SaamiDocNet; das Geld ist definitiv vorhanden)

all in all we plan four meetings/workshops of project participants under participation of community assistants and other community members: 1. Lujavv'r (after or before SaamiDocNet meeting Dec 2010), 2. Arjeplog/Arvidsjaur (after or before SaamiDocNet meeting Spring 2011), 3. London (nach Absprache mit ELAR technical staff, ideally connected to training course for assistants), 4. Nijmegen (nach Absprache mit DOBES technical staff, ideally connected to training course for assistants), 5. Tromsø (project completion)
die ersten beiden treffen (von SaamiDocNet finanziert, das ersten beiden Treffen stehen schon fest unabhängig vom erfolg dieses antrags) dienen vor allem zur Besprechung des Arbeitsplans, Aufgabenverteilung, Training der Assistenten
in London und Nijmegen sollen unsere beiden Assistenten sowie je ein weiterer nicht-regulär bezahlter Konsultant aus allen vier saamischen Sprachgemeinschaften dabeisein, das Archiv kennenlernen und gemeinsam mit Archivaren/Dokumentationslinguisten Probleme von collaborative documentation/archiving diskutieren, idealerweise finden die treffen in london/nijmegen in verbindung mit training statt an dem dann die saami auch teilnehmen können
letztes treffen tromsø zur präsentation der ergebnisse, projektabschluss im zusammenhang mit einer konferenz zu saamischer sprachdokumentation und sprachtechnologie (diese konferenz wird von uni tromsø organisiert)

\subsection{Time schedule}

Dec 2010
meeting Murmansk (SaamiDocNet, finanzierung bereits zugesichert)
all team members including assistants and main native speaker consultants from all four Saami cummunities

Spring 2011
official project start
meeting in Arjeplog/Arvidsjaur (SaamiDocNet, finanzierung bereits zugesichert)
(why not Jokkmokk? Josh könnte dann dort auch sein Archiv übergeben?)
all team members including assistants and main native speaker consultants from all four Saami cummunities

Spring–Summer 2011
checking/completing/unifying existent Kildin Saami orthographic transcriptions and Russian/English translations at DOBES archive 
(M. Rießler, with help of E. Karvovskaya and Kildin assistant)
checking/completing/unifying existent Pite Saami orthographic transcriptions and Swedish/English translations
(J. Wilbur, with help of E. Karvovskaya and Pite assistant)
agree on special conventions for spoken texts
(all team members)
processing Kildin and Pite multi-tier annotations [MR: ich weiß nicht ob das "korpustechnisch" so heißt] and transferring to Giellatekno corpus structure (es wie die nordsaamischen Texte auf den gtsvn server packen)
(M. Rießler, J. Wilbur, E. Karvovskaya, with help of T. Trosterud and C. Gerstenberger)

Fall-Winter 2011
programming/testing morphological parsers for Kildin and Pite (die sind jetzt schon angefangen!), test parsing and creation and making available of automators for Kildin and Pite (Kildin ist jetzt schon angefangen und im Internet!)
(T. Trosterud, M. Rießler, J. Wilbur)
start annotating and archiving Ume Saami recordings from DAUM (E. Karvovskya, Saami assistant)

Winter–Spring 2012
checking/completing/unifying existent Ter Saami orthographic transcriptions and Russian/English translations at DOBES archive 
(M. Rießler, with help of E. Karvovskaya and Saami assistant)
start programming/testing the syntactic parsers for Kildin Saami (M. Rießler, T. Trosterud) and Pite Saami (J. Wilbur, T. Trosterud)

Spring–Summer 2012

Summer–Fall 2012

Fall–Winter 2012

Winter–Spring 2013
meeting Tromsø
finalize project
…

automatic lemmatization and dictionary creation Kildin Pite (M. Rießler, J. Wilbur, C. Gerstenberger)

dictionary export to Oahpa!

multimedia annotation of lexica (Saami assistants)

lexica export to wikipedia

\begin{longtable}{ l l l }
Oct 2010 & Initial meeting/workshop in Tromsø&\\
\end{longtable}

\section{Budget summary}



\begin{longtable}{| l | r |}
\hline
Cost items & €\\
\hline
\textbf{A. Personnel expenditure}&\\
\hline
Research personnel &\\
\hline
Other personnel &\\
\hline
Scholarships &\\
\hline
\hline
\textbf{B. Running non-personnel costs} & \\
\hline
Travel and accommodation &\\
\hline
Consumables/other &\\
\hline
\hline
\textbf{C. Non-recurring expenses} & \\
\hline
Equipment &\\
\hline
\hline
\textbf{Total A+B+C} & \textbf{~250.00}\\
\hline
\multicolumn{2}{| l |}{\textbf{Please note:}}\\
\multicolumn{2}{| l |}{\textbf{Administration overheads and valuable added tax for personnel}}\\
\multicolumn{2}{| l |}{\textbf{costs are not covered by the Volkswagen Foundation.}}\\
\multicolumn{2}{| l |}{\textbf{Personnal funds for foreign partner should be estimated at the}}\\
\multicolumn{2}{| l |}{\textbf{standard rates of the relevant country}}\\
\hline
\end{longtable}

\newpage
\section{Budget details and justification}
\subsection*{A Personnel expenditure}
\noindent \textbf{Salaries}\\
\begin{longtable}{| l | l | r |}
\hline
M.\,Rießler&total costs as&\\
13 TV-L&projected by the&\\
3 years&university's administration&176,400\\
\hline
J.\,Wilbur&total costs as&\\
13 TV-L&projected by the&\\
3 years (50\%)&university's administration&88,200\\
\hline
E.\,Karvovskaya&total costs as&\\
40 working hours/month& projected by the&\\
3 years&university's administration&16,497\\
%\hline
%1 Main native speaker consultant&Werkvertrag with&\\
%in Sweden&Freiburg university&\\
%&250 €/month&9,000\\
%\hline
%1 Main native speaker consultant&Werkvertrag with&\\
%in Russia&Freiburg university&\\
%&150 €/month each&5,400\\
\hline
\end{longtable}

\subsection*{B Running non-personnel costs}
\paragraph{Travel and accommodation}
We estimate the following average travel costs:\\

\noindent The regular per diem remunerations based on \textit{Verpflegungsmehraufwendungen} and \textit{Übernachtungspauschale} come to € 135 for The Netherlands, € ?? for Norway, € ?? for Sweden, and € ?? for Russia.

%\begin{longtable}{| l | l | r |}
%\hline
%2009 	& 2 x Berlin-Nijmegen&400\\
%fall	& 5 days x 2 pers The Netherlands&1,350\\
%	\hline
%2009–	& 4 x Berlin-Serres&1,200\\
%2010	& 1 x Sliven-Serres&50\\
%	& 14 days x 5 pers Greece&4,200\\
%	\hline
%2010	& 2 x Berlin Sliven&600\\
%spring	& 30 days x 2 pers Bulgaria&3,000\\
%\hline
%2010	& 2x Serres or Sliven-Nijmegen&800\\
%spring	& 1x Berlin-Nijmegen&200\\
%	& 5 days x 3 pers The Netherlands&2,025\\
%\hline
%2010	& 3 x Berlin-Strumica&900\\
%summer	& 1 x Serres or Sliven-Strumica&50\\
%	& 1 x 10 days Macedonia (Rießler)&700\\
%	& 25 days x 3 pers Macedonia&5,250\\
%\hline%hier
%2010	& 2 x Berlin-Serres&600\\
%fall	& 30 days x 2 pers Greece&3,600\\
%\hline
%2011	& 2 x Berlin-Sliven&600\\
%spring	& 30 days x 2 pers Bulgaria&3,000\\
%\hline
%2011	& 1 x Berlin-Sofia (Rießler)&300\\
%spring	& 1 x Serres-Sofia&50\\
%	& 1 x Sliven-Sofia&50\\
%	& 5 days x 5 pers Sofia&1,250\\
%\hline
%2011	& 1 x Berlin-Nijmegen&200\\
%spring	& 5 days x 1 pers The Netherlands&675\\
%\hline
%2011	& 3 x Berlin-Serres&900\\
%summer	& 2 x 10 days Greece (Rießler, Voß)&1,200\\
%	& 2 x 30 days Greece (Leluda-Voß, student)&3,600\\
%\hline
%2011–	& 1 x Berlin-Sliven&300\\
%2012	& 1 x 30 days Bulgaria&1,500\\
%\hline
%2012	& 2 x Berlin-Strumica&600\\
%spring	& 1 x 10 days Macedonia (Voß)&700\\
%	& 1 x 20 days Macedonia (Leluda-Voß)&1,400\\
%	\hline
%2012	& 1 x Berlin-Nijmegen&200\\
%spring	& 1 x Sliven-Nijmegen&400\\
%	& 5 days x 2 pers The Netherlands&1,350\\
%\hline	
%2012	& 3 x Berlin-Sliven&900\\
%summer	& 5 days x 1 pers The Netherlands&675\\
%	\hline
%2012	& 2 x Serres/Sliven-Berlin&600\\
%fall	& 2 x 10 days Germany (assistants)&500\\
%\hline
%2011	& 3 x Berlin-Serres&900\\
%spring	& 1 x Sliven-Serres&50\\
%	& 5 days x 4 pers Serres&1,200\\
%\hline
%\end{longtable}

%\noindent \textbf{Consumables/Other}
%\begin{longtable}{| l |  r | r | r |}
%\hline
%Item	&Price	&Qty.		&Total\\
%	&	&per unit	&price\\
%\hline		
%prepaid internet/phone&20€/month&2x36 months&1,440\\
%for Assistants&&&\\
%\hline
%Mini-DV tapes&&ca. 100&300\\
%\hline
%DVD-RWs&&ca. 100&300\\
%\hline
%Batteries	&&&300\\
%\hline	
%Stationary and photocopy costs&&&	700\\
%\hline
%\end{longtable}

\subsection*{C Non-recurring expenses}
\noindent desktop computer for principal investigators, notebook for assistants; photokameras, videocameras (for multimedia lexicon work done by assistants) are available from PSDP, KSDP %WÄRE ES NICHT SCHÖN, EINE TOLLE NEUE KAMERA ZU KAUFEN?
\begin{longtable}{| l | l | r | r | r |}
\hline
Item&Model&Price&Qty.&Total\\
	&&per unit&&price\\
\hline
multimedia&iMac 27”&2,000&2&4,000\\
desktop computer&&&&\\
\hline
all-in-one&MacBook Pro&2,000&1&2,000\\
notebook&&&&\\
\hline
software&??&??&3&??\\
\hline
\end{longtable}

\newpage
\section*{Appendix}
\subsection*{Curriculum vitae and list of publications for the key participants}

\end{document}
