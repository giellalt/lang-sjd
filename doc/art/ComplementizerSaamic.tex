\documentclass[a4,12pt]{scrartcl}

\usepackage{xltxtra,tabularx,authblk,qtree}

\usepackage[english]{babel}

\usepackage{hyperref}
\hypersetup{colorlinks=true, citecolor=red, linkcolor=blue, urlcolor=green, breaklinks=true}

\usepackage[style=authoryear-icomp]{biblatex}
\bibliography{bibliographyNeu}
\DefineBibliographyStrings{english}{
}

\usepackage{gb4e}%compiles only if this package comes sist?

%%%%%%%%%%%%%%%%%%%%%%%%%%%%%%%%%%%%%%%%%%%
%
%				                              GUIDELINES
%
%%%%%%%%%%%%%%%%%%%%%%%%%%%%%%%%%%%%%%%%%%%
%
%GUIDELINES for the contributions to the volume “Semantic functions of complementizers in European languages”
%
%	COMPLEMENTIZER: “a word, particle, clitic or affix, one of whose functions it is to identify [a clause] as a complement” (Noonan 2007: 55).
%	COMPLEMENT MARKER: other kinds of expressions – e.g. whole constructions, word orders, prosodical featurs, etc. – which identify clauses as complements.
%	CANONICAL COMPLEMENTIZER (optional term): expressions that conform to Noonan’s definition (see above) AND occur only in finite complements.
%	Complementation Marker (identify finite+non-finite clauses as complements) > Complementizer (word/particle/clitic/affix which identify finite+non-finite clauses as complements) > Canonical Complementizer (Complementizer in *finite* clauses)
%
%General requirements
%Focus
%-	If your languages have equivalents of English that and if in Bob doesn’t know that/if Janet loves him, focus at least on these.
%-	If your languages DO NOT have equivalents of English that and if, focus at least on other finite-clause complementizers or complementation markers that are NOT identical to question words (e.g. English how, what).
%-	If your languages DO NOT have finite-clause complementizers or complementation markers, focus at least on complementizers or complementation markers which 1) are found in complements of propositional attitude predicates (‘think’, ‘believe’), knowledge predicates (‘know’, ‘learn’), perception predicates (‘see’, ‘hear’), and/or utterance predicates (‘say’, ‘tell’), and 2) are NOT identical to question words (e.g. English how, what).
%
%	Note:	You are free to discuss also question-word like complementizers, but the focus of the paper must include complementizers or complementation markers that are not like question words (provided, of course, such complementizers or complementation markers exist).
%
%Number of languages covered
%Contributions should preferably cover more than one language from the language family studied. This does not concern Basque, Maltese, Hungarian, Albanian, Greek, Romani and Kalmyk.
%%%%%%%%%%%%%%%%%%%%%%%%%%%%%%%%%%%%%%%%%%%%

\title{Complementizers in three Saamic languages}
\author[1]{Kristina Kotcheva}
\author[2]{Michael Rießler}
\affil[1]{Department of Linguistics, University of Konstanz}
\affil[2]{Scandinavian Department, University of Freiburg}

\setmainfont[Mapping=tex-text]{Charis SIL}

\begin{document}
 %%Contributions should be between 12,500 and 17,500 words long, including references.%KK: ca. 10–15 Seiten%

%%%only preliminary for keeping track of the content
\tableofcontents 
\newpage
%%%

\maketitle

\begin{abstract}
The chapter describes the the syntactic structure and semantic functions of complementizer constructions in three endangered Saamic (Uralic) languages.\\%needs some more elaboration

{\bf Keywords:} semantics, syntax, etymology, typology, subordination, complementizer\\

{\bf ISO 639-3:} sjd, sme, sms 
\end{abstract}


\section{Introduction}\label{intro}

\subsection{Aim of the investigation}

\paragraph{Prospect I}  Our paper aims at a detailed description of the syntactic structure and semantic functions of complementizer constructions in North Saami, illustrated with minimal pairs (in concordance with the project description). North Saami data will be taken from existing descriptions \cite[e.g.]{nickel1994, sammallahti1998b, nielsen1979-1} and corpora (\url{http://giellatekno.uit.no/text.en.html}).  %Are there restrictions on the co-occurrence of complementizers and e.g. modal markers?\\

\paragraph{Prospect II} We will also try to compare North Saami with Kildin Saami (E-Saamic<Uralic), for which syntactic and semantic descriptions are virtually non-existent. Kildin Saami data will be taken from a spoken language corpus (Kola Saami Documentation Project, DoBeS Archive. Nijmegen: Max Planck 
Institut for Psycholinguistics, \url{http://www.mpi.nl/DOBES/}) and from elicitations and experiments with native speakers.

\subsection{Chapter outline}
The present section \ref{intro} serves as an introduction to our chapter on complementizers in Saamic. It includes a brief overview on the Saamic branch of languages and the three investigated languages North, Skolt and Kildin Saami as well as a presentation of the data sources used for our investigation.

Section \ref{syntax} is rather descriptive. Based on the available descriptive sources and using an ontological form-to-function approach it describes the the syntactic environments of different formatives identifying clausal complements in North Saami. North Saami is used here because available descriptions are most complete for this Saamic language.

Section \ref{semantics} is the main part of the present chapter. Based on the functional definition of complementation marking (??SEE INTRODUCTION TO THIS VOLUME??) and the previous descriptive section \ref{syntax} on North Saami it uses a function-to-form approach for  presenting a complete inventory of complementizers and other complement markers attested in Kildin and Skolt Saami and compares these two languages to North Saami.

Section \ref{history} consists of a rather short historical-comparative description of etymological sources for the discussed formatives in all three languages as well as possible changes in their form and function.

Section \ref{summary} summarizes our findings.

\subsection{Saamic languages}
%%1. briefly introduce the languages covered (among other things, mention possible characteristics of the languages as well as their genetic and geographical affiliation and – if possible – the number of speakers of the languages).

\begin{figure}
\begin{center}
\caption{Sápmi and the Saamic languages. The languages of the project are shaded.} %shadow!
\label{SaamiLgs}
\includegraphics[width=0.7\textwidth]{SaaLgs.jpg}
\end{center}
\end{figure}
\begin{figure}
\caption{Saamic language tree \cite[according to][6–34]{sammallahti1998b}.} \label{tree}
\qtreecenterfalse
\Tree [.Saamic [.{West-Saamic} [.South South Ume ] [.Central Pite Lule {\bf North} ] ] [.{East-Saamic} [.Mainland Inari {\bf Skolt} Akkala ] [.Peninsula {\bf Kildin} Ter ] ] ]
\end{figure}

The Saamic\footnote{Other common spellings are Sámi(c) or Sami(c).} languages belong to the Finnougric branch of the Uralic language family and form a dialect continuum across an area extending from central Scandinavia to the Kola Peninsula in the Russian Federation. Typologically, Saamic languages are unlike the Indoeuropean languages in the area in being exclusively suffixing languages and exhibiting predominantly postpositions. Phrases are generally head-final, although SVO has become grammatical as well, just as postponed relative constructions occur. Saami case systems consist of between seven and nine cases and nouns additionally have possessive inflection (this feature is lost in Kildin). In most Saamic languages, verbs and pronouns are inflected for dual (in Skolt dual is preserved only in pronominal inflection, in Kildin the feature is lost completely), in addition to singular and plural number categories. Additionally, all Saamic languages are known for their complicated non-linear morphology realized as ablaut, consonant gradation (phonological alternations in a stem's consonants triggered by the morphological environment) and/or palatalization of noun and verb stems (only in East-Saamic). A syntactic feature common for all Saamic languages is negation expressed by means of an inflected negation auxiliary followed by the non-finite main verb in a special connegative form.

Due to the influence of Norwegian, Swedish, Finnish and Russian languages and cultures, all ethnic Saami are fluent in their respective contact languages, while younger generations are typically monolingual in these. As a result, all Saamic languages are today either extinct, moribund or endangered. The present chapter focuses on Kildin, Skolt and North Saami, which exhibit different stages of endangerness due to divers language sociological conditions and official support in the respective countries where these languages are spoken.

\paragraph{North Saami} (Endonym {\it sámegiella} or {\it davvisámegiella}) is the least endangered Saamic language today. The language belongs to the West-Saamic subbranch and is spoken by about 17,000 speakers in Norway, Sweden and Finland \cite[cf., e.g.,][1]{sammallahti1998b}. North Saami is devided into several dialect groups and subgroups \cite[e.g.][9–20]{sammallahti1998b}. The three main dialects are (from North to South) Sea North Saami, spoken along the northernmost Norwegian coast, Finnmark North Saami spoken in the inland of northernmost Norway and adjacent north-westernmost Finland, and Torne North Saami spoken in northernmost Sweden and adjacent areas in Norway and Finland. North Saamic is relatively well documented and described. Standard written North Saami is based on Latin script. The language is used in written media of all kind today.

Main contact languages of contemporary North Saami are Finnish, Norwegian and Swedish.

\paragraph{Skolt Saami} (Endonym {\it (nuõrtt)sääˊmǩiõll}) is an East-Saamic language spoken in Finland, Russia and Norway. Skolt Saami used to be spoken predominantly on the western Kola Peninsula and the adjacent mainland in the borderland area between Russia, Norway and Finland. However, most members of the original Skolt Saami villages in Russia resettled in Finland after the area was ceded to the Soviet Union in 1945. As a result, the language is now spoken in a relatively compact area by approximately 300 Saami in Finland \cite[??PAGE??]{salminen2007}. In Russia there are only about 30 Skolt Saami speakers left today \cite[??PAGE??]{scheller2011}. The traditional Skolt Saami dialect of Sør-Varanger in Norway died out in the 1970s.

Resettlement to Finland is also the explanation for the fact that Skolt Saami has been more extensively documented and described by Finish linguists, has been sucessfully standardized only in Finland and is probably also somewhat less endangered than the closely related Kildin Saami language of Russia due to support from the Finnish state. 

Skolt Saami has a standardized orthography, based on Latin script, which was developed in Finland in the 1970s. This orthography could in principle cover all Skolt Saami dialects in the different countries (just as the standard North Saami orthography is valid in Finland, Norway and Sweden) and a project is currently going on at the Skolt Saami museum in Neiden/Norway to introduce materials written in Standard Skolt Saami in Norway and Russia. However, the Skolt Saami written language has so far only been used in Finland. 

Main contact languages of contemporary Skolt Saami are Finnish (in Finland) and Russian (in Russia) until recently there were also very close contacts with speakers of Karelian.

\paragraph{Kildin Saami} (Endonym {\it (kīllt)sāmʼkīllʼ}) is not only the most endangered, but also the least documented and described among the three languages investigated in the present chapter. The language is actively spoken by no more than 100 native speakers \cite[??PAGE??]{scheller2011}. Kildin Saami used to be spoken in the central inland parts and the central coastal parts of the Kola Peninsula, but the original Kildin dialect areas have fragmented chiefly as the result of forced centralization to larger towns. As a result, today most of the Kola Saami population live in the town of Lovozero, which is nowadays usually regarded as the ‘Saami capital’ of Russia.

A Cyrillic-based orthography for Kildin was developed during the 1980s and has been used in text books for elementary schools and in other pieces of teaching materials, including a few dictionaries. Also a considerable amount of literary texts have been published, although most of these are short or written for learners of Kildin Saami.

The main contact language of contemporary Kildin Saami is Russian until recently there were also very close contacts with speakers of Karelian.

\subsection{Data sources}
North Saami
\begin{itemize}
\item Grammatical descriptions
\item Written corpora (GT)
\end{itemize}

Skolt Saami
\begin{itemize}
\item Grammatical descriptions
	\subitem Feist
	\subitem School grammar
\item Written corpora (Textsammlung mit CD)
\end{itemize}

Kildin Saami
\begin{itemize}
\item Grammatical descriptions
	\subitem Kert?
\item Spoken corpora (KSDP)
\end{itemize}

\section{Syntax of complementizers in three Saamic languages}\label{syntax}
%2.	give an overview of complementizers and complementation markers, including also e.g. equivalents of English how, what, etc.
KK: Tabellen am besten zusammenfügen

\subsection{North Saami}
\begin{table}[!ht]
\begin{tabular}{l | l l}
\hline
\hline
Nsaa & Gloss & Translation\\
\hline
{\bf ahte} & {\sc COMP} & 'that'\\
{\bf jos/jus} & {\sc COMP} & 'if'\\
{\bf go} & {\sc advCOMP} & 'when/if/because'\\
{\bf go} & {\sc attrCOMP} & 'that/which/than'\\
sahte & & 'if only'\\
vaikko/vaikke & & 'even if, even though'\\
\hline
\hline
\end{tabular}
\label{KompNSaami}
\caption{Complementizers in North Saami}
\end{table}

In our paper we shall concentrate on {\it ahte}, {\it jos} and {\it go}. {\it Ahte} and  {\it jos}  function basically like English {\it that} (neutral) and {\it if} (uncertainty). {\it Ahte} can be combined with other subordinators (which shows that its meaning is neutral).  
{\it Go} is a polyfunctional adverbial subordinator (or “adverbalizer”) introducing temporal (\ref{goTemporal}) or causal (\ref{goCausal}) adverbial clauses.


\subsection{Kildin Saami}


\begin{table}[!ht]
\begin{tabular}{l | l l l l}
\hline
\hline
KSa 		& Gloss			&					&NSa translation	& English translation\\
\hline
šte		&{\sc COMP} ?		&< Russian {\it što}		&ahte			&'that'\\
%bydte	&				&					&ahte\\
jesli		&{\sc	 COMP} ?		&< Russian {\it jesli}		&jos				&'if'\\
%poke	&				&< Russian {\it poka}		&go\\
gū		&{\sc advCOMP} ??	&$\leftarrow$Proto Saamic&go				& 'XXX'\\
\hline
\hline
\end{tabular}
\label{KildinComps}
\caption{Complementizers in Kildin Saami}
\end{table}


\subsection{Skolt Saami}
%3.	mention which complementizers and/or complementation markers will be in focus in the remainder of the paper.
%4.	if relevant, consider the caveat that what is intuitively a complementizer need not be what actually identifies the complement as a complement.


\section{Semantics of complementizers in three Saamic languages}\label{semantics}% AT LEAST HALF OF THE PAPER
%1.	What are the semantic functions of – and contrasts between – the complementizers or complementation markers in focus (in addition to the function of embedding propositions/states of affairs in other propositions/states of affairs)? 
%		Note a:		Consider e.g. the following possibilities, some of which are discussed in the workshop description: i) contrasts between proposition (truth-valued) and state-of-affairs (non-truth-valued), ii) contrasts related to information structure, iii) contrasts related to knowledge, modality and adjacent phenomena (volitionality, implicativity, polarity), iv) temporal contrasts, v) aspectual (and Aktionsart) contrasts, vi) voice/valency-related contrasts.
%		Note b:	   Not only binary semantic contrasts may be found (for instance, between certainty and uncertainty), but also more complex semantic contrasts.
%		Note c:	   Complementizers or complementation markers may be semantically complex. For instance, we might find a complementizer or a complementation marker with both modal and aspectual meanings.
%2.	Are there minimal pairs which support the analysis of the semantic contrast? If yes, please exemplify, and please include examples with propositional attitude predicates, knowledge predicates, perception predicates, and utterance predicates, if relevant.
%3.	Are there other facts (co-occurrence restrictions, harmonic combinations, etc.) which support the analysis of the semantic contrast?
%
\subsection{Types of CTPs}%complement taking predicates
%%KK: von hier nehmen wir die liste mit prädikaten und chechen mit daten aus wörterbüchern und den korpora
classes of CTPs \cite{noonan2007}
\paragraph {\bf propositional attitude predicates} {\it think, believe, assume, suppose; doubt}
% ''express an attitude regarding the truth of the proposition expressed as their complement'' \cite[3.2.2]{noonan2007}
\begin{itemize}
\item positive attitude: believe, think, suppose, assume\dots
\item negative attitude: not believe, doubt, deny\dots
\end{itemize}
%animate subjects are experiencers/identical with speaker it's certain that\dots
%''tendency across languages'' – CTP expresses ''the subject's propositional attitude, while adverbials, choice of complementizer and complement type normally express the speaker's propositional attitude''\cite{noonan2007}

\begin{exe}
\ex %think
% árvalit, gáddit, jurddaSit, jurddahit; (schnell, einzelnen gedanken) jurdilit, smiehttat

\ex %believe
%jáhkkit

\ex %assume
%gáddit, doaivut

\ex %suppose
%navdit, gáddit,, oaivvildit, oskut, ; doaivut

\ex %doubt
%eahpidit (ahte ii), vávjit, 
%mun eahpidan ahte sus ii leat riekta 'ich bezweifle, dass er recht hat' (egtl. ? 'ich bezweifle, dass er/sie *nicht* recht hat?)

\ex %not believe


\ex %deny
%Siitit 'bestreiten'
%biehtadit, biehttalit, Siitit 'abstreiten'


\end{exe}


\paragraph {\bf knowledge predicates} predicates of knowledge and acquisition of knowledge ({\it know, learn, discover, realize, find out, forget} = semifactives; {\it see, hear, dream})  \cite[3.2.5]{noonan2007}
%subject is experiencer, predicates ''describe the state or the manner of acquisition of knowledge'' 
%complements presupposed to be true, but may be part of new information, of what is being asserted ((not proposition itself presupposed but rather the truth of the proposition))
%complement of {\it know} is background information

\begin{exe}

\ex %know
%diehtit, 

\ex %learn
%boahtit, diehtit; muosáhit, vásihit 'erfahren'


\ex %discover
%gávdnat, gávnnahit; áicat, fuorbmát, fourrbmaSit 'entdecken'

\ex %realize
%fuorbmát, fuomáSahttit, fuormaStit, fuopmáSit 'feststellen'
%áicat, fuorbmát, fuomaSit, merkot 'bemerken (wahrnehmen)'

\ex %find out
%boahtit,diehtit, fuobmát, fuomáSit, fuopmaSit, gávnnahit, geahCái beassat, oaZZut, Cielgasa 'herausfinden'

\ex %forget
%vajálduhtti, vajáldahttit 'vergessen'
%mun vajálduvven grijji lohkat, vaikko livCCen galgan málistit 'ich war in mein Buch versunken und vergass darüber, dass ich Essen kochen sollte' 


\ex %see
%oaidnit; geahCCAt

\ex %hear
%gullat

\ex %dream
%niegadit + acc / birra


\end{exe}



\paragraph {\bf perception predicates} ({\it see, hear} – \cite[3.2.5]{noonan2007} regards them as {\it knowledge predicates} when not used in ''immediate perception sense'')

\paragraph {\bf utterance predicates} {\it say, tell, report, promise, ask} 
%''utterance predicates are used in sentences describing a simple transfer of information initiated by an agentive subject'' \cite[3.2.1]{noonan2007}
%animate subjects are agents
%except {\it promise} these predicates have complements with independent time reference (ITR)%KK>MR: time reference of complement clause does not depend upon the point/period of time referred to in the main sentence

\begin{exe}
\ex %say, tell
%cealkit, dadjat, lohkat; muitalit; namuhit; (vom hörensagen) cuoigut

\ex %report
%máinnastit, dieDihit, muitalit, Cilget, raporteret, (mehrmals, über versch. dinge, in vollem ernst) beakkuhit

\ex %promise
%lohpidit

\ex %ask
%jearrat 'fragen'

\end{exe}

\paragraph {\bf desiderative predicates} \cite[3.2.7]{noonan2007} {\it want, wish, desire, hope}
%subject is experiencer ''expressing ad desire that the complement proposition be realized'', ''oppisite of predicates of fearing'', ''expressing a positive \dots feeling about the ultimative realization of the complement proposition''
{\bf three usage classes}
	\begin{enumerate}
		\item {\it hope}-class: %complements with independent time reference; emotional attitude toward the complement proposition; status of complement proposition unknown, ''but which can turn out to be true''
		
\begin{exe}

\ex %hope
%sávvat, doaivvut

\end{exe}
		\item {\it wish}-class: %complements with independent time reference; emotional attitude toward the complement proposition; status of complement proposition unknown, ''but which can turn out to be true''; {\bf normally} a contrafactive interpretation; complement in subjunctive
		
\begin{exe}


\ex %wish
%sávvat, váinnuhit; doaivvuhit

\end{exe}
		\item {\it want}-class: %complements with dependent time reference, ''express a  desire that some state or event may be realized in the future''; ''complements may refer to unrealizable state of affairs''
		
\begin{exe}

\ex %want
%áigut, (starkes verlangen:) háliidit, hálidit, sihtat

\ex %desire als 'verlangen'
%gáibidit, geahCCat 'fordern'
%goardit 'aufdringlich verlangen'

\end{exe}
	\end{enumerate}


\paragraph {\bf pretence predicates} \cite[3.2.3]{noonan2007} %''the world described by the proposition embodies in the complement is not the real world''
{\it imagine; pretend, fool (into thinking), trick (into thinking), make believe}
%subjects are experiencers ({\it imagine; pretend, make believe}) or agents ({\it fool (into thinking), trick (into thinking), pretend, make believe}
%''a very general implication that the proposition [of the complement] is false''
%complements have ITR
%''complements of pretence predicates are normally interpreted as hypothetical nonevents''–nevertheless, complements are coded not as subjunctives (in Lgs which have the distinction indicative/subjunctive)

\begin{exe}
\ex 
%oskkuhit, jáhkilit 'glauben machen'

\ex %imagine

\ex %pretend


\ex %trick into thinking


\ex %fool into thinking

\end{exe}

\paragraph {\bf commentative predicates} = {\bf factives} {\it regret, be sorry, be sad; be odd, be significant, be importatnt} \cite[3.2.4]{noonan2007}
%if overt human subject, then subject=experiencer; CTP ''provide a comment on the complement proposition which take the form of an emotional reaction or evaluation\dots or a judgement''
%emotional evaluations/judgements ''are normally made on events or states that people take to be real (Rosenberg 1975)'' (–> complements are presupposed)
%complements are discourse dependent, have idependent time reference (depend not upon the time of the main clause)
%in Lgs with verb vs. adjective distinction, commentative predicates tend to be adjectives


\begin{exe}
\ex %regret


\ex %be sorry


\ex %, be sad



\ex %be odd


\ex %be significant



\ex %be importatnt


\end{exe}

\paragraph {\bf predicates of fearing} \cite[3.2.6]{noonan2007} {\it be afraid, fear, worry, be anxious} 
%take an experiencer subject, express ''an attitude of fear or concern that the complement proposition will be or has been realized''
%complement has independent time reference
%''languages differ in the asignement of negation to such complements'' : a positive statement in the complement is interpreted affirmatively (e.g. Eng. I fear that the sky will fall on my head) vs. positive complement interpreted negatively (e.g. Latin {\it Vereor.{\sc 1sg} ne.{\sc neg} accidat.{\sc 3sg}} 'I fear that it may not happen' – {\it Vereor.{\sc 1sg} ut.{\sc compl} accidat.{\sc 3sg}} 'I fear that it may not happen')
%Lgs ''possess devices to indicate the degrree of certainty for the realization of the complement proposition'' (indicative vs. subjunctive, change of complementizer, presence or absence of negation) 

\begin{exe}
\ex %be afraid; %fear
%balahit, ballat


\ex %, worry
%áittardit, fuolahit, fuola atnit, moraStit, vára váldit
%!geahCCat/bearrái ahte

\ex % be anxious
%s.o. be afraid


\end{exe}



\subsection{Distribution}
\paragraph{Types of complements}%1.	In which types of complements are the complementizers or complementation markers in focus found?
\begin{exe}
	\ex  commentative predicate \label{ahteSubject} \cite[194]{nickel1994} 
	\gll 	Imaš 	lei, 	{\bf ahte} 	rámbuvrriin 	oaččui 	viine 	oastit.\\
	odd 	be:{\sc 3sg.pst} {\sc comp} shop:{\sc loc:pl} get:{\sc 3sg:pst} wine:{\sc acc:sg} buy:{\sc inf}\\
	\glt 	‘It was odd that one could buy wine in the shops.’
	
	\ex preception predicate \label{ahteObject} \cite[194]{nickel1994}
	\gll 	Mun 	oainnán, 	{\bf ahte} 	áddjá 	lea 	boahtán.\\
	{\sc 1sg} see:{\sc 1sg} {\sc comp} grand\_father be:{\sc 3sg.prs} come:{\sc pastpart}\\%past participle
	\glt 	‘I see that grandfather has come.’
	
		 \ex knowledge predicate? \textbf{jos} \label{josObjekt}
	 	\begin{xlist}
		\ex 	
		\gll Mon in dieđe {\bf jos} mon gearggan boahtit.\\
		{\sc 1sg} {\sc neg:1sg} know:{\sc conneg} {\sc comp} {\sc 1sg} get\_ready:{\sc 1sg} come:{\sc inf}\\
		\glt ‘I don't know if I will manage to come.’ %’Jag vet inte om jag hinner komma.’ 
		
		\ex perception predicate
		\gll 	Mun 	oainnán, 	{\bf jos}  son áigu njuovvat sávzza.\\
		{\sc 1sg} see:{\sc 1sg} {\sc comp} {\sc 3sg} will:{\sc ind.prs} butcher:{\sc inf} sheep:{\sc acc.sg}\\
		\glt 	‘I shall see if s/he will butcher a sheep.’
		\end{xlist}	

%
%KK>MR: wat machemer mitn relativsätzen? es gibt einen artikel über die konzeptionelle/gramm. nähe von COMP-sätzen und REL-sätzen. aber wir müssen uns überlegen, ob wir's brauchen/wollen
%
%	 \ex Relative clause \label{ahteRelativsatz} \cite[194]{nickel1994}
%	\gll Sus lei jou geasset leamaš dat jurdda, ahte dáidá mus oažžut veahki.\\
%	{\sc 3sg.loc} be:{\sc 3sg.ind.pst} already in\_summer be:{\sc pastpart} DEM:{\sc 3sg.nom} thought:{\sc nom.sg} REL may\_be{\sc 3sg.ind.prs} {\sc 1sg.loc} get:{\sc inf} help:{\sc acc.sg}\\
%	'S/he had already in summer the idea that I might help him/her.'
%
%	\ex \cite[195]{nickel1994}
%	\gll De dáhpáhuvai, {\bf ahte} rievvár bođii.\\
%	so happen:{\sc 3sg.ind.pst} {\sc advCOMP} robber:{\sc nom.sg} come:{\sc 3sg.ind.pst}\\
%	\glt 'It happened that a robber came.'
	
	\ex Causal adverbial clause \label{goCausal}
%reviewer's note:  If one looks at the translation, the "final" clause introduced by /go/ seems rather a "causal" clause. *fertig*
\gll 	In 	sáhte 	boahtit 	{\bf go} 	lean 	buohcci.\\
	{\sc neg:1sg} can:{\sc conneg} come:{\sc inf} {\sc advCOMP} be:{\sc 1sg.prs} sick:{\sc pred}\\
\glt 	‘I cannot come because I am sick.’\\ (literally: 'I cannot come when I am sick.')%'Jeg kan ikke komme fordi jeg er syk.'%??for
\end{exe}

\begin{exe}

	\ex \label{goAttr} Attributive clause of (subject) pronoun \cite[439]{nickel1994}% (…{\it dat go…} ‘…that which…’)
	\gll 	Dasa 	lei 	sivvan dat 	{\bf go} 	son 	čorbmagođii 	mu 	reaŋgabártáža.\\
	{\sc dem:ill} be:{\sc 3sg.pst} reason {\sc dem}  {\sc attrCOMP} {\sc 3sg} fist\_hit:{\sc 3sg:pst} {\sc 1sg:gen} servant\_boy:{\sc dim:acc}\\
	\glt 	‘The reason for that was such that he hit my boy servant.’ %'Årsaken var at han begynte å slå gutten som var dreng hos mig.'

	\ex \label{goComp} Comparative clause \cite[199]{nickel1994} %(…{\it riggát go…} ‘…bigger than…’) 
	\gll 	Dat 	dáidá 	riggát 	{\bf go} 	mii 	jáhkkit.\\
	{3sg}	may\_be:{\sc 3sg} rich:{comp:pred} {\sc attrCOMP} {\sc 1pl} believe:{\sc 1pl}\\
	\glt 	‘He is perhaps richer than we believe.’ %'Han er kanskje rikere enn vi tror.'


\end{exe}


\paragraph{Types of complement-taking elements}%2.	With which types of complement-taking elements are the complementizers or complementation markers in focus found? Please include a discussion of positive and negative propositional attitude predicates (‘think’, ‘believe’, ‘doubt, ‘be uncertain’), knowledge predicates (‘know’, ‘learn’, ‘forget’), perception predicates (‘see’, ‘hear’), and utterance predicates (‘say’, ‘tell’),  if relevant. (If relevant, you may also consider pretence predicates (‘pretend’), factive emotion predicates (‘be sad’, ‘rejoice’), desiderative predicates (‘hope’, ‘want’, ‘wish’), and predicates of fearing (‘be afraid’).
\begin{exe}
\ex	Knowledge predicate
	\begin{xlist}
	\ex 
	\gll Mun in dieđe, {\bf ahte} son áigu njuovvat sávzza.\\
	1{\sc sg} {\sc neg:1sg.prs} know:{\sc conneg} {\sc comp} {\sc 3.sg} will:{\sc 3sg:ind:prs} butcher:{\sc inf} sheep:{\sc acc:sg} \\
	\glt 'I don't know that s/he will butcher a sheep.'
	
	\ex 
	\gll Mun dieđán, {\bf ahte} son áigu njuovvat sávzza.\\
	1{\sc sg} know:{\sc 1sg.ind.prs} {\sc comp} {\sc 3.sg} will:{\sc 3sg. ind.prs} butcher:{\sc inf} sheep:{\sc acc.sg} \\
	\glt 'I know that s/he will butcher a sheep.' 

	\ex
	\gll Mun in dieđe, áigu-{\bf go} son njuovvat sávzza.\\
	{\sc 1sg} {\sc neg:1sg} know:{\sc conneg} will:{\sc 3sg.ind.prs}-{\sc qm} 3{\sc sg} butcher:{\sc inf} sheep:{\sc acc.sg}\\
	\glt 'I don't know if/whether s/he will butcher a sheep.'\\
	(literally: 'I don't know: will s/he butcher a sheep?')
	
	\ex 
	\gll Mun in dieđe, {\bf jos} son áigu njuovvat sávzza.\\
	{\sc 1sg} {\sc neg:1sg} know:{\sc conneg} {\sc comp} will:{3sg.ind.prs} 3{\sc sg} butcher:{\sc inf} sheep:{\sc acc.sg}\\
	\glt 'I don't know if s/he will butcher a sheep.'
	\end{xlist}
\end{exe}



\paragraph{A SYSTEM of COMPs?}
%3.	Do the complementizers under consideration make up a “system”, i.e. a distributionally delimited set of expressions? If yes, how many members does the system have?

\paragraph{Triggers?}
%4.	Are there elements (e.g. negation or modal verbs) that trigger one specific complementizer or complementation marker?


\subsection{Complementizer omission}
%1.	Can any of the complementizers or complementation markers be omitted (similarly to English that in I know (that) the butler did it)? If yes, which complementizers, under which circumstances, and with which effect?
However, the structure with omitted {\it ahte} seems to represent an apposed main clause instead of subordination.

\begin{exe}
	\ex \label{ahteOmitted} {\it ahte} can be omitted \cite[439]{nickel1994}%(Nickel:439)
	\gll 	Máhtte 	muitalii 	{\bf (ahte)} 	áddjá 	boahtá \\
 	Máhtte 	tell:{\sc 3sg:pst} {\sc comp} grandfather come:{\sc 3sg}\\
	\glt ‘Máhtte said (that) grandfather is coming.’
 \end{exe}

%
\subsection{Combinability issues}
%1.	Can any of the complementizers or complementation markers be combined with other complementizers or complementation markers? If yes, which complementizers, under which circumstances, and with which effect?
%2.	Can any of the complementizers or complementation markers be combined with other subordination markers (relativizers, adverbializers)? If yes, which complementizers, which other subordination markers, under which circumstances, and with which effect?
%
%Non-complementizing functions of complementizer forms
%1.	Can any of the complementizers or complementation markers be used as relativizers (similarly to English that in the book that I read two days ago) or adverbializers? If yes, which complementizers, and under which circumstances?
%2.	Can any of the complementizers or complementation markers be used as e.g. adverbs, particles, verbs or nouns? If yes, which complementizers, under which circumstances, and with which effect?
%
{\it Ahte} can also be combined, e.g. with another subordinator {\it ahte go} ‘that when’, with a conjunction {\it ahte vai} ‘that so that’ or with an interrogative pronoun {\it ahte maid} [that what:{\sc acc}] ‘that which'. In (\ref{ahteGo}–\ref{ahteMaid}) {\it ahte} seems to indicate the subordinative relation alone while the second part of the subordinating formative {\it go, vai, maid} defines the nature of this relation: temporal adverbial in (\ref{ahteGo}), final adverbial in (\ref{ahteVai}) and a complement in (\ref{ahteMaid}):% Nickel:196 "I enkelte setninger finnes et såkalt pleonastisk (overflødig) ahte":

\begin{exe}
	\ex \label{ahteGo} Temporal clause \textbf{ahte go} \cite[196]{nickel1994}
	\gll Roađđi lea {\bf ahte} {\bf go} ruoksadin šaddá beaivvi badjánemiin ja ruokset fas beaivvi luoitádettiin.\\
	redness:{\sc nmlz} be:{\sc 3sg} {\sc comp} {\sc advCOMP} red:{\sc ess} become:{\sc 3sg} sun:{\sc gen.sg} ascent:{\sc gerund:com} and be\_red again sun:{\sc gen.sg} descent:{\sc gerund:com}\\
	\glt ‘{\it Roađđi} is when it gets red with the sun rising and is red again when the sun is setting.'
	% 'Det er "roaDDi" når det blir rødt når sola stiger opp, og rødt når den går ned igjen.'
	%(kk: AHTE + GO 'att när'; AHTE als reiner COMP?)


\ex \textbf{go} temporal \label{goTemporal}
\begin{xlist}
\ex %Temporal adverbial clause: present \label{goTempContext} 
\cite[196]{nickel1994}
\gll 	Mun 	boađán 	{\bf go} 	gearggan.\\
	{\sc 1sg} come:{\sc 1sg} {\sc advCOMP} get\_ready:{\sc 1sg}\\
\glt 	‘I shall come when / if I am ready.’%'Jeg kommer når jeg er ferdig'
%
%################# go ist hier ambig, vgl. Bsp {\ref{ContextDismbig}): ####################
%% 
\ex \label{goContextDisambig} \cite[196–7]{nickel1994}
\gll Boađe {\bf go} dáhtut.\\ %nickel:196
come-2SG.IMP.PRS AdvCOMT want-2SG.PRS\\
\glt 'Come when / if you want.'
%
% ###### Disambiguierung durch JOS? #################
%############################################################################


\ex %Temporal adverbial clause: past \label{goTemporalPST} 
\cite[436]{nickel1994}%nickel:439, sammallahti:105 -- subjektsatz
\gll 	Buorre 	lei 	{\bf go} 	bohtet\\
	good:{\sc pred} be:{\sc 3sg.pst} {\sc advCOMP} come:{\sc 2sg.pst}\\
\glt 	‘It was good that you came.’
\end{xlist}


\ex \label{goTemporal}
\begin{xlist}
\ex %Temporal adverbial clause: present \label{goTempContext} 
\cite[196]{nickel1994}
\gll 	Mun 	boađán 	{\bf go} 	gearggan.\\
	{\sc 1sg} come:{\sc 1sg} {\sc advCOMP} get\_ready:{\sc 1sg}\\
\glt 	‘I shall come when / if I am ready.’%'Jeg kommer når jeg er ferdig'
%
%################# go ist hier ambig, vgl. Bsp {\ref{ContextDismbig}): ####################
%% 
\ex \label{goContextDisambig} \cite[196–7]{nickel1994}
\gll Boađe {\bf go} dáhtut.\\ %nickel:196
come-2SG.IMP.PRS AdvCOMT want-2SG.PRS\\
\glt 'Come when / if you want.'
%
% ###### Disambiguierung durch JOS? #################
%############################################################################
\ex %Temporal adverbial clause: past \label{goTemporalPST} 
\cite[436]{nickel1994}%nickel:439, sammallahti:105 -- subjektsatz
\gll 	Buorre 	lei 	{\bf go} 	bohtet\\
	good:{\sc pred} be:{\sc 3sg.pst} {\sc advCOMP} come:{\sc 2sg.pst}\\
\glt 	‘It was good that you came.’
\end{xlist}
	
	

	\ex \label{ahteVai} Causal clause \textbf{ahte vai} \cite[196]{nickel1994}
	\gll 	De 	son 	riemai 	gillut 	beatnagiiddis 	ealu 	ala 	{\bf ahte} {\bf vai} 	mii 	eat 	beasa 	rátkit.\\
	so {\sc 3sg} begin:{\sc 3sg:pst} sick:{\sc inf} dog:{\sc poss:3sg:acc:pl} reindeer\_flock:{\sc gen} on {\it comp} {\sc advCOMP} {\sc 1pl} {\sc neg:1pl} be\_able:{\sc conneg} separate:{\sc inf}\\
	\glt ‘So s/he began to sick his/her dogs on the reindeer flock so that we were unable to separate (the reindeer).’
	%'Så begynte hun å hisse hundene sine på reinflokken for at vi ikke skulle få skilt ut (reinsdyra våre)'
	%(kk: AHTE+VAI 'att för att'; AHTE als reiner COMP?)

	\ex \label{ahteMaid} Final clause \textbf{ahte maid} \cite[196]{nickel1994}
	\gll 	Muhto 	mii 	dušše 	geahčastalaime 	vuorrasii 	{\bf ahte} {\bf maid} 	dal 	vuoras 	dahká.\\
	but {\sc 1pl} just look\_a\_bit:{\sc 1pl:pst} old\_person:{\sc ill} {\sc comp} what:{\sc acc.sg} {\sc part} old\_person do:{\sc 3sg}\\
	\glt 	‘But we just looked at the old man, what is the old man doing.'%'Men vi bare tittet på gamlingen; hva ville han nå gjøre?'
	%(kk: AHTE + MAID :'att vad': AHTE als reiner COMP?)

\end{exe}


\section{History of complementizers in three Saamic languages}\label{history}
%1.	What are the diachronic sources of the complementizers or complementation markers under investigation, and how did they develop?
%2.	How did the semantic functions of the complementizers or complementation markers under investigation develop?
NSaa. {\it ahte} < Fin. {\it että} 'that' < FS pronominal stem {\it *e}- + modal suffix {\it -kta/ktä} 'so that'\bigskip\\
%\pause 
NSaa. {\it go} < PSaa {\it *ko-} < FA {\it *ku-} 'as; than'\bigskip\\
%\pause 
NSaa. {\it jos/jus} < Fin. {\it jos} 'if'\bigskip\\
%\pause
$\rightarrow$ {\it ahte} and {\it jos/jus} are Finnish loanwords in NSaa while {\it go} is inherited from pre-Protosaami \cite[226;245;251]{sammallahti1998b}.

 {\it Jos} is borrowed from Finnish {\it jos} ‘if’. {\it Ahte} is also borrowed from Finnic (cf. Finnish {\it että} ‘that’), but this took place already in Common West-Saamic. The marker is thus inherited into North Saami from an earlier stage. {\it Go} is inherited from pre-Proto-Saamic.
%Itkonen 1956:20,28 ahte auch im Ostsaamischen
%
%KK: August2011 ergänzt die Grammatiken und Korpora für NSaa und KilSaa; war im Review angemahnt.

%\section{Freestyle part} (not obligatory)
%Freestyle
%In this part you are free to focus on whatever aspect of complementizer semantics (or complementation marker semantics) you wish to (for instance, you might wish to give a unified account of some of the findings reported in the descriptive part), or to simply include additional points which you consider important. Alternatively, you may entirely omit the freestyle part of the paper.

\section{Summary}\label{summary}

\printbibliography

\end{document}